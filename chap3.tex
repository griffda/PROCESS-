%%-*-Latex-*-
\mychapter{Physics and Engineering Models}
\label{chap:models}

There are a great number of individual models within \PSC characterising many
different aspects of a fusion power plant. Several of these will always be
used by the code and so require no input by the user to activate them.
However, in many cases there is a choice of model available, and each of these
has its own user-controlled switches or flags. This chapter summarises these
models, and indicates their location and interaction within the code, together
with the relevant switch settings and required parameter values. The
nomenclature used, and instructions on how to set switches, etc., are
explained fully in Chapter~\ref{chap:run}.

\section{Physics Models}

\setlength{\parskip}{0mm}
\subsection{Fusion Reactions}

By default, the code assumes that the deuterium-tritium (D-T) fusion reaction
is utilised by the power plant being modelled:
\begin{equation}
\mathrm{D + T} \Longrightarrow \mathrm{^{4}He + n + 17.6 \,MeV}
\label{eq:d-t}
\end{equation}
20\% of the energy produced is given to the alpha particles ($^4$He), which
remain within the plasma and thermalise (slow down) due to collisions, thus
heating the plasma. The remaining 80\% is carried away by the neutrons, which
deposit their energy within the blanket and shield. The tritium fraction of
the D-T fuel is controlled using variable \texttt{ftr} in input block
\texttt{PHYDAT}.

\setlength{\parskip}{5mm}
\PS can also model D-$^3$He power plants, which utilise the following primary
fusion reaction:
\begin{equation}
\mathrm{D + \mbox{$^3$He}} \Longrightarrow \mathrm{^{4}He + p + 18.3 \,MeV}
\label{eq:dhe3}
\end{equation}
Note that no neutrons are produced by this reaction, only charged particles,
allowing the whole of the fusion power to be used to heat the plasma. Useful
energy is extracted from the plasma since the radiation power produced is very
high, and this can be converted to electricity in a number of ways.

Since the temperature required to ignite the D-$^3$He reaction is considerably
higher than that for D-T, it is necessary to take into account the following
D-D reactions, which have significant reaction rates at such temperatures:
\begin{eqnarray}
\mathrm{D + D} & \Longrightarrow & \mathrm{^{3}He + n + 3.27 \,MeV} \\
\mathrm{D + D} & \Longrightarrow & \mathrm{T + p + 4.03 \,MeV}
\end{eqnarray}
Also, as tritium is produced by the latter reaction, D-T fusion is also
possible. As a result, there is still a small amount of neutron power
extracted from the plasma.

The use of the D-$^3$He coding within \PS is controlled using the switch
\texttt{idhe3} in input block \texttt{PHYDAT}. If \texttt{idhe3 = 1} then the
D-$^3$He fusion reaction is assumed, otherwise the D-T reaction is
assumed. The ratio of the ``fuel'' ions (D, $^3$He and T) may be modified if
\texttt{idhe3 = 1} using the three variables \texttt{fdeut}, \texttt{fhe3} and
\texttt{ftrit}, respectively (all in input block \texttt{PHYDAT}).

\subsection{Plasma Profiles}
\setlength{\parskip}{0mm}
All the plasma profiles are assumed to be parabolic, that is, of the form
\begin{eqnarray}
\mbox{Density : } n(r) & = & n_0 \left( 1- \left(\frac{r}{a}\right)^2 \right)
^{\alpha_n} \\
\mbox{Temperature : } T(r) & = & T_0 \left( 1- \left(\frac{r}{a}\right)^2
\right) ^{\alpha_T} \\
\mbox{Current : } J(r) & = & J_0 \left( 1- \left(\frac{r}{a}\right)^2  \right)
^{\alpha_J}
\end{eqnarray}
where $r$ varies from 0 to $a$, the plasma minor radius. This gives
volume-averaged values $\langle n \rangle = n_0 / (1+\alpha_n)$, etc.  These
volume averages are used throughout the code, along with the profile indices
$\alpha$, thus making the code ``$\frac{1}{2}$-D''.  The relevant profile
index variables {\tt alphan}, {\tt alphat} and {\tt alphaj} in the code are
set in input block {\tt PHYDAT}.

\subsection{Beta Limits}

\subsubsection{Troyon limit \protect\cite{IPDG,172}}

The Troyon beta limit is given by
\begin{equation}
\langle \beta \rangle < \frac{gI(\mbox{MA})}{a(\mbox{m})B_0(\mbox{T})}
\label{eq:troyon}
\end{equation}
where $B_0$ is the axial vacuum toroidal field, and $\beta$ is defined with
respect to the total equilibrium ${\bf B}$-field~\cite{172}. The Troyon
coefficient $g$ is set using input parameter {\tt dnbeta}. To apply the beta
limit, constraint equation no.\ 24 should be turned on with iteration variable
no.\ 36 ({\tt fbetatry}). The limit can be applied to either the total plasma
beta, in which case switch {\tt iculbl} in input block {\tt PHYDAT} should be
set to zero, to only the thermal component of the plasma beta, in which
case {\tt iculbl} should be set to one, or to the thermal plus neutral beam
components, in which case {\tt iculbl} should be set to two.

\subsubsection{Limit on $\epsilon.\beta_p$ value}

To apply a limit to the value of $\epsilon.\beta_p$, where $\epsilon = a/R$ is
the inverse aspect ratio, constraint equation no.\ 6 should be turned on with
iteration variable no.\ 8 ({\tt fbeta}). The limiting value of
$\epsilon.\beta_p$ should be set using input parameter {\tt epbetmax} in input
block {\tt PHYDAT}.

\subsection{Density Limits}

Six density limit models~\cite{172} are available in \PSD These are calculated
in routine {\tt CULDLM}, which is called by {\tt PHYSICS}.  To enforce any of
these limits, set {\tt iculdl = 1} in input block {\tt PHYDAT}, and turn on
constraint equation no.~5 with iteration variable no.~9 ({\tt fdene}).  In
addition, switch {\tt idensl} in input block {\tt PHYDAT} must be set to the
relevant value, as follows:-
\begin{description}
\item [{\tt idensl = 1} :] ASDEX model
\item [{\tt idensl = 2} :] Borrass model for ITER, I
\item [{\tt idensl = 3} :] Borrass model for ITER, II
\item [{\tt idensl = 4} :] JET edge radiation model
\item [{\tt idensl = 5} :] JET simplified model
\item [{\tt idensl = 6} :] Hugill-Murakami $M.q$ model
\end{description}

\subsection{Plasma Current Scaling Laws}

A number of plasma current scaling laws exploiting the inverse relationship
between plasma current and edge safety factor $q_{\psi}$~\cite{172} are
available in \PSD These are calculated in routine {\tt CULCUR}, which is
called by {\tt PHYSICS}.  Flag {\tt iculcr} must be set to {\tt 1} in input
block {\tt PHYDAT}. In addition, flag {\tt icurr} in input block {\tt PHYDAT}
must be set to the relevant value, as follows:-
\begin{description}
\item [{\tt icurr = 1} :] Peng analytic fit
\item [{\tt icurr = 2} :] Peng double null divertor scaling (TART)
\item [{\tt icurr = 3} :] Simple ITER scaling
\item [{\tt icurr = 4} :] Revised ITER scaling
\item [{\tt icurr = 5} :] Todd empirical scaling, I
\item [{\tt icurr = 6} :] Todd empirical scaling, II
\item [{\tt icurr = 7} :] Connor-Hastie model
\end{description}

\subsection{Confinement Time Scaling Laws}

More than twenty tokamak-relevant energy confinement time scaling
laws~\cite{IPDG,172,H90P,iter93h} are present within \PSD These are calculated
in routine {\tt PCOND}. The value of {\tt isc} in input block {\tt PHYDAT}
determines which of the scalings is used in the plasma energy balance
calculation.  Table~\ref{tab:scaling_laws} summarises the available scaling
laws.

% Table summarising confinement time scaling laws in PROCESS.

\begin{table}[ht]
\begin{center}

\begin{tabular}{||c||l||} \hline
{\tt isc} & scaling law \\ \hline
\tt 1  & \rm Neo-Alcator (ohmic) \\
\tt 2  & \rm Mirnov (H-mode) \\
\tt 3  & \rm Merezhkin-Muhkovatov (L-mode) \\
\tt 4  & \rm Shimomura (H-mode) \\
\tt 5  & \rm Kaye-Goldston (L-mode) \\
\tt 6  & \rm ITER 89-P (L-mode) \\
\tt 7  & \rm ITER 89-O (L-mode) \\
\tt 8  & \rm Rebut-Lallia (L-mode) \\
\tt 9  & \rm Goldston (L-mode) \\
\tt 10 & \rm T10 \\
\tt 11 & \rm JAERI-88 \\
\tt 12 & \rm Kaye-Big Complex \\
\tt 13 & \rm ITER H90-P (H-mode) \\
\tt 14 & \rm ITER Mix (minimum of {\tt 6} and {\tt 7}) \\
\tt 15 & \rm Riedel (L-mode) \\
\tt 16 & \rm Christiansen et al. (L-mode) \\
\tt 17 & \rm Lackner-Gottardi (L-mode) \\
\tt 18 & \rm Neo-Kaye (L-mode) \\
\tt 19 & \rm Riedel (H-mode) \\
\tt 20 & \rm ITER H90-P (amended) \\
\tt 21 & \rm Large Helical Device (stellarator) \\
\tt 22 & \rm Gyro-reduced Bohm (stellarator) \\
\tt 23 & \rm Lackner-Gottardi (stellarator) \\
\tt 24 & \rm ITER-93H (H-mode) \\
\tt 25 & \rm TITAN (RFP) \\ \hline
\end{tabular}
\end{center}
\caption[TABLE_TAU]{{\it
Summary of the energy confinement time scaling laws in \PSD These laws are all
cited in ~\cite{172,IPDG} and references therein, with the exception of
laws~20~\cite{H90P}, 21~\cite{LHD}, 22~\cite{GRB}, 23~\cite{LG}, 
24~\cite{iter93h}, and 25~\cite{titan1}. }}
\label{tab:scaling_laws}
\end{table}
 % Table summarising confinement time scaling laws

\subsection{Bootstrap Current Scalings}

The fraction of the plasma current provided by the so-called bootstrap effect
can be either input into the code directly, or calculated using one of three
methods, as summarised here.

\subsubsection{Direct input}
To input directly the bootstrap current fraction, set {\tt bscfmax} in input
block {\tt PHYDAT} to $(-1)$ times the required value.

\subsubsection{ITER scaling \protect\cite{IPDG}}

To use the ITER scaling method for the bootstrap current fraction, set {\tt
ibss = 1} in input block {\tt PHYDAT}, and {\tt bscfmax} in input block {\tt
PHYDAT} to the maximum required bootstrap current fraction ($\leq 1$). This
method is valid at high aspect ratio only.

\subsubsection{General scaling \protect\cite{Nevins}}

To use a more general scaling method, set {\tt ibss = 2} in input block {\tt
PHYDAT}, and {\tt bscfmax} in input block {\tt PHYDAT} to the maximum required
bootstrap current fraction ($\leq 1$).

\subsubsection{Numerically fitted scaling \protect\cite{172}}

To use a numerically fitted scaling method, valid for all aspect ratios, set
{\tt ibss = 3} in input block {\tt PHYDAT}, and {\tt bscfmax} in input block
{\tt PHYDAT} to the maximum required bootstrap current fraction ($\leq 1$).

\subsection{Current Drive}

In addition to inductive current drive, eight non-inductive current drive
efficiency models~\cite{172} are present in \PSD The fraction of the
volt-seconds to be produced by non-inductive means, {\tt fvsbrnni}, should be
set in input block {\tt PHYDAT}, and flag {\tt irfcd} in input block {\tt
CDDAT} should be set to {\tt 0} for purely inductive scenarios, or {\tt 1}
otherwise. The current drive efficiency model to be used in this latter case
is defined by the value of switch {\tt iefrf} in input block {\tt CDDAT}:-

\begin{description}
\item [{\tt iefrf = 1} :] Fenstermacher Lower Hybrid model
\item [{\tt iefrf = 2} :] Ion cyclotron model~\cite{IPDG}
\item [{\tt iefrf = 3} :] Fenstermacher electron cyclotron resonance model
\item [{\tt iefrf = 4} :] Ehst Lower Hybrid model
\item [{\tt iefrf = 5} :] ITER neutral beam model~\cite{IPDG,172}
\item [{\tt iefrf = 6} :] Culham Lower Hybrid model~\cite{172}
\item [{\tt iefrf = 7} :] Culham electron cyclotron model~\cite{172}
\item [{\tt iefrf = 8} :] Culham neutral beam model~\cite{172}
\item [{\tt iefrf = 9} :] Oscillating Field current drive (RFPs only --- see
Section~\ref{sec:rfpcd})
\end{description}

It is sometimes useful to adjust artificially the current drive efficiency
values produced by these routines. This can be achieved by setting the scaling
coefficient {\tt feffcd} in input block {\tt CDDAT}. The wall plug to plasma
efficiencies can also be adjusted, by changing the relevant variable ({\tt
etaech}, {\tt etalh}, {\tt etanbi} or {\tt etaof}) in input block {\tt CDDAT}.

\subsection{Safety and Environment Models}

The models comprising the safety and environmental calculations~\cite{FISPACT}
within the code are all called from routine {\tt FISPAC}. They are only
performed once, at the end of each run, as they take a relatively long time to
evaluate, and the results are only used for diagnostic purposes --- no
constraints are imposed at present to minimise doses, for instance. No
switches need to be set to perform these calculations.

\subsection{Other Physics Switches}

\subsubsection{Plasma cross-sectional shape}

Switch \texttt{ishape} in input block \texttt{PHYDAT} controls whether the
input values for the plasma elongation (\texttt{kappa}) and triangularity
(\texttt{triang}) should be used (\texttt{ishape = 0}), or whether they should
be scaled with the plasma aspect ratio (\texttt{ishape = 1}). The latter case
should only be used with a TART machine.

\subsubsection{Fusion power calculations}

Switch {\tt iiter} in input block {\tt PHYDAT} controls which model for the
fusion power calculations should be used. If {\tt iiter = 1}, the ITER
model~\cite{IPDG} is used.

\subsubsection{Neo-classical correction effects}

Switch {\tt ires} in input block {\tt PHYDAT} controls whether neo-classical
correction effects~\cite{Uckan} are included in the calculation of the plasma
resistance and ohmic heating power in routine {\tt POHM}, which is called by
routine {\tt PHYSICS}. If {\tt ires = 1}, these effects are included. Note
that the scaling used is only valid for aspect ratios between 2.5 and 4, and
it is possible for the plasma resistance to be wrongly calculated as negative
if {\tt ires = 1} and the aspect ratio is too high.

\subsubsection{Aspect ratio scaling of Troyon $g$ coefficient}

Switch {\tt gtscale} in input block {\tt PHYDAT} determines whether the Troyon
$g$ coefficient {\tt dnbeta} (Equation~\ref{eq:troyon}) should scale with
aspect ratio ({\tt gtscale }$\not=~${\tt 0}), or be fixed at the input value
({\tt gtscale = 0}).

\subsubsection{Inverse quadrature in $\tau_E$ scaling laws}

Switch {\tt iinvqd} in input block {\tt PHYDAT} determines whether the energy
confinement time scaling laws due to Kaye-Goldston ({\tt isc = 5}) and
Goldston ({\tt isc = 9}) should include an inverse quadrature scaling with the
Neo-Alcator result ({\tt isc = 1}). A value {\tt iinvqd = 1} includes this
scaling.

\subsubsection{Scrape-off width}

Switch {\tt iscrp} in input block {\tt PHYDAT} determines whether the
scrape-off widths should be calculated as 10\% of the plasma minor radius
({\tt iscrp = 0}), or set equal to the input values {\tt scrapli} and {\tt
scraplo} ({\tt iscrp = 1}).

\subsubsection{Wall load calculation}

Switch {\tt iwalld} in input block {\tt PHYDAT} determines whether the neutron
wall load should be calculated using the plasma surface area ({\tt iwalld =
1}) or the first wall area ({\tt iwalld = 2}) as the denominator.

\subsection{Stellarator Physics}

To activate the stellarator coding, it is necessary to create a file {\tt
device.dat}, containing the single character {\tt 1} in the first row, in the
working directory. If the file is absent, or the character is set to the digit
{\tt 0} or {\tt 2}, the stellarator model is not used. Switch {\tt istell} in
{\tt INCLUDE} file {\tt stell.h} is set to {\tt 1} if the stellarator model is
to be used, otherwise {\tt istell = 0}, and the tokamak or RFP model is used
instead.  It is suggested that the plasma geometry, constraint equations and
iteration variables detailed in~\cite{PWF}, Note~0252 are used.
\setlength{\parskip}{5mm}

The following models are used automatically if the stellarator model is
activated:
\setlength{\parskip}{0mm}

\subsubsection{Beta limit}
The beta limit is assumed to be 5\%, based on 3-D MHD calculations
\cite{Nuhrenberg}. To apply the beta limit, constraint equation no.\ 24 should
be turned on with iteration variable no.\ 36 ({\tt fbetatry}).

\subsubsection{Density limit}
The density limit relevant to stellarators has been proposed to be~\cite{LHD}
\begin{equation}
n_{\mbox{\scriptsize max}} = 0.25\left( P B_0 / R_0 a_p^2\right)^\frac{1}{2}
\end{equation}
where $n$ is the line-averaged electron density in units of
$10^{20}$~m$^{-3}$, $P$ is the absorbed heating power (MW), $B_0$ is the
on-axis field (T), $R_0$ is the major radius (m), and $a_p$ is the plasma
minor radius (m). To enforce the density limit, turn on constraint equation
no.\ 5 with iteration variable no.\ 9 ({\tt fdene}).

\subsubsection{$\tau_E$ scaling laws}
Three confinement time scaling laws relevant to stellarators are present
within \PSD The value of {\tt isc} in input block {\tt PHYDAT} determines
which of these is used in the plasma energy balance calculation.
\begin{eqnarray}
\tau_E\; (\mbox{Large Helical Device~\cite{LHD}: {\tt isc=21}})
& = & 0.17 R_0^{0.75} a_p^2 n^{0.69} B_0^{0.84} P^{-0.58} \\
\tau_E\; (\mbox{Gyro-reduced Bohm~\cite{GRB}: {\tt isc=22}})
 & = & 0.25 B_0^{0.8} n^{0.6} P^{-0.6} a_p^{2.4} R_0^{0.6} \\
\tau_E\; (\mbox{Lackner-Gottardi~\cite{LG}: {\tt isc=23}})
& = & 0.17 R_0 a_p^2 n^{0.6} B_0^{0.8} P^{-0.6} \iota^{0.4}
\end{eqnarray}
Here, $\iota$ is the rotational transform, which is equal to the reciprocal
of the tokamak safety factor $q$.

\subsubsection{Heating power options}
The method by which auxiliary heating power is supplied to the stellarator is
determined by the switch {\tt isthtr} in input block {\tt STELLA}:
\begin{description}
\item [{\tt isthtr = 1} :] electron cyclotron resonance heating
\item [{\tt isthtr = 2} :] lower hybrid heating
\item [{\tt isthtr = 3} :] neutral beam injection
\end{description}
The value of variable {\tt pheat} in input block {\tt CDRIV} determines the
actual amount of auxiliary heating power (in Watts) to be applied to the
plasma. This variable may be used as an iteration variable (no.\ 11).

\subsection{Reversed Field Pinch Physics}

To activate the RFP coding, it is necessary to create a file {\tt device.dat},
containing the single character {\tt 2} in the first row, in the working
directory. If the file is absent, or the character is set to the digit {\tt 0}
or {\tt 1}, the RFP model is not used. Switch {\tt irfp} in {\tt INCLUDE} file
{\tt rfp.h} is set to {\tt 1} if the RFP model is to be used, otherwise {\tt
irfp = 0}, and the tokamak or stellarator model is used instead.
\setlength{\parskip}{0mm}

\subsubsection{Beta limit}
The poloidal beta is limited to a maximum value given by input parameter
\texttt{betpmx} (default = 0.19) using constraint equation~48 and iteration
variable~79 (\texttt{fbetap}).

\subsubsection{Density limit}
No density limit is explicitly coded for RFPs, other than by simply
constraining the upper bound of the electron density variable \texttt{dene}
(iteration variable 6).

\subsubsection{$\tau_E$ scaling law}
One confinement time scaling law relevant to RFPs is present
within \PSD The value of {\tt isc} in input block {\tt PHYDAT} determines
the scaling to be used in the plasma energy balance calculation.
\[
\tau_E\; (\mbox{TITAN RFP~\cite{titan1}: {\tt isc=25}})
 = 0.05 a^2 I_p(\mathrm{MA})
\]

\subsubsection{$F$-$\Theta$ plot}
The pinch parameter $\Theta$ is set using iteration variable~78:
\texttt{rfpth} in input block \texttt{PHYDAT}. The corresponding value of the
reversal parameter $F$ (\texttt{rfpf}) is calculated using routine
\texttt{FTHETA}. $F$ is constrained to be negative using constraint
equation~49 with f-value \texttt{frfpf} (iteration variable~80).

\subsubsection{Current drive}
\label{sec:rfpcd}
The RFP oscillating field current drive option is turned on by setting
\texttt{iefrf = 9} in input block \texttt{PHYDAT}. The wall plug to injector
efficiency is set using input parameter \texttt{etaof} in input block
\texttt{CDDAT}, which has a default value of 0.5, and the unit cost is set
using input parameter \texttt{ucof} in input block \texttt{UCSTINP}. The
default value for \texttt{ucof} is 3.3~\$/W of injected power.

\section{Engineering Models}

\subsection{TF Coil Options}

The TF coils can be either resistive (copper) or superconducting. Switch {\tt
itfsup} in input block {\tt TFC} should be set to {\tt 1} for superconducting
coils, or {\tt 0} for purely copper coils. The following options are available
within the superconducting TF coil model ({\tt itfsup = 1}):

\subsubsection{Superconducting materials}

Switch {\tt isumattf} in input block {\tt TFC} specifies which superconducting
material is to be used:
\begin{description}
\item [{\tt isumattf = 1} :] binary Nb$_3$Sn superconductor
\item [{\tt isumattf = 2} :] ternary Nb$_3$Sn superconductor
\item [{\tt isumattf = 3} :] NbTi superconductor
\item [{\tt isumattf = 4} :] generic Nb$_3$Sn superconductor
\item [{\tt isumattf = 5} :] generic NbTi superconductor
\end{description}

\subsubsection{Stress model}

Switch {\tt itfmod} in input block {\tt TFC} controls whether a simple stress
model ({\tt itfmod~=~0}) or more complex stress model ({\tt itfmod = 1})
should be used. To enforce the stress limits calculated using either of these
models, constraint equation no.\ 31 (case stress) and/or constraint equation
no.\ 32 (conduit stress) should be turned on with iteration variables no.\ 48
({\tt fstrcase}) and/or no.\ 49 ({\tt fstrcond}), respectively. The stress
limit can be adjusted using input parameters {\tt csutf} and {\tt csytf} in
input block {\tt TFC}.

\subsubsection{Current density limits}

% ADD jcrit_model switch

Two limits can be applied to the current density $J$ in the (superconducting)
TF coils. To ensure that $J$ does not exceed the critical value, constraint
equation no.\ 33 should be turned on with iteration variable no.\ 50 ({\tt
fiooic}). To ensure that $J$ does not exceed the current density protection
limit, constraint equation no.\ 35 should be turned on with iteration variable
no.\ 53 ({\tt fjprot}).
\setlength{\parskip}{5mm}

(A similar constraint on the TF coil current density exists if resistive coils
are being used ({\tt itfsup = 0}). In this case, constraint equation no.\ 23
should be turned on with iteration variable no.\ 28 ({\tt fjtfc}).)
\setlength{\parskip}{0mm}

\subsubsection{Reversed Field Pinch TF coils}

The TF coils for the RFP option are derived from the TITAN-II coil set. The
radial thickness is set using \texttt{tfcth} as usual, but the toroidal
thickness may also be set, using iteration variable~77 (\texttt{tftort}) in
input block \texttt{TFC}. This is constrained to be no larger than is
geometrically possible using constraint equation~47 with iteration variable~76
(\texttt{frfptf}).

\subsection{PF Coil Options}

\subsubsection{PF coil position}

The PF coil locations are controlled using a set of switches stored in array
{\tt ipfloc} (see Figure~\ref{fig:build1}), and are calculated in routine {\tt
PFCOIL}. Input block {\tt PFC} contains all the variables associated with the
PF coils. The coils are (usually) organised into groups containing two PF
coils placed symmetrically above and below the midplane, and each group {\tt
j} has an element {\tt ipfloc(j)} assigned to it. Input parameter {\tt ngrp}
should be set to the number of groups, and {\tt ncls(j)} should be assigned
the number of coils in each group --- which should be {\tt 2} in each case.
\setlength{\parskip}{5mm}

In the following, all variables are defined in the variable descriptor file
{\tt var.des}. The values for {\tt rpf1}, {\tt rpf2}, {\tt zref(j)} and {\tt
routr} should be adjusted by the user to locate the PF coils accurately.

The three possible values of {\tt ipfloc(j)} correspond to the following PF
coil positions:

\setlength{\parskip}{0mm}
\begin{description}
\item [{\tt ipfloc(j) = 1} :]  PF coils are placed above the OH coil (one
group only);
\footnotesize
\begin{eqnarray*}
R & = & \mbox{\tt rohc + rpf1} \\
Z & = & \pm
\mbox{\tt ( hmax*ohhghf + 0.1 + 0.5*( hmax*(1.0D0-ohhghf)+tfcth+0.1) )}
\end{eqnarray*}
\normalsize
\item [{\tt ipfloc(j) = 2} :]  PF coils are placed above the TF coils (one
group only);
\footnotesize
\begin{eqnarray*}
R & = & \mbox{\tt rmajor + rpf2*triang*rminor} \hspace{62mm} \\
Z & = & \pm (\mbox{\tt hmax + tfcth + 0.86})
\end{eqnarray*}
\normalsize
\item [{\tt ipfloc(j) = 3} :]  PF coils are placed radially outside the TF
coils (any number of groups);
\footnotesize
\begin{eqnarray*}
R & = & \mbox{\tt rtot + tfthko/2.0D0 + routr} \hspace{63mm}\\
Z & = & \pm(\mbox{\tt rminor*zref(j)})
\end{eqnarray*}
\normalsize
\end{description}

\subsubsection{PF coil resistance}

The PF coils can be either resistive or superconducting. This is determined
from the value of {\tt ipfres} in input block {\tt PFC}. If {\tt ipfres = 0},
the PF and OH coils are assumed to be superconducting. If {\tt ipfres = 1},
they are assumed to be resistive, with resistivity given by the value of
variable {\tt pfclres} in input block {\tt PFC}.

\subsubsection{Superconducting materials}

Switch {\tt isumatpf} in input block {\tt PFC} specifies which superconducting
material is to be used for the PF and OH coils if {\tt ipfres = 0}:
\begin{description}
\item [{\tt isumatpf = 1} :] binary Nb$_3$Sn superconductor
\item [{\tt isumatpf = 2} :] ternary Nb$_3$Sn superconductor
\item [{\tt isumatpf = 3} :] NbTi superconductor
\end{description}

\subsection{OH Coil Options}

Switch {\tt iohcl} in input block {\tt BLD} controls whether an OH coil is
present. A value of {\tt 1} denotes that this coil is present, and should be
assigned a non-zero thickness {\tt ohcth} in input block {\tt BLD}. A value of
{\tt iohcl = 0} denotes that no OH coil is present, in which case the
thickness {\tt ohcth} should be set to zero. No PF coils should be located
at positions defined by {\tt ipfloc(j)~=~1} if no OH coil is present.

\subsubsection{OH coil swing time}
\label{sec:tohs}

In the steady-state power plant scenario ({\tt lpulse} $\not=$~{\tt 1}), the
length of time taken for the OH coil current to reverse (see
Figure~\ref{fig:current_vs_time}) is determined from the value of switch {\tt
tohsin} in input block {\tt TIME}. If {\tt tohsin = 0.0D0}, then the swing
time {\tt tohs} is given by $\mbox{\tt tohs} = I_p / 0.5$, where $I_p$ is the
plasma current in MA\@. Furthermore, the PF coil ramp time {\tt tramp} and
shutdown time {\tt tqnch} are set equal to {\tt tohs}.  If {\tt tohsin}
$\not=$~{\tt 0.0D0}, the swing time {\tt tohs = tohsin}, and the ramp and
shutdown times are as defined in input block {\tt TIME}.
\setlength{\parskip}{5mm}

If, however, a pulsed power plant is being modelled ({\tt lpulse = 1}), the OH
coil swing time {\tt tohs} is either set in routine {\tt INITIAL} or via input
block {\tt TIME}, or it can be iterated by using iteration variable 65. The
ramp and shutdown times in the pulsed case are always set equal to {\tt
tohs}. To ensure that the plasma current ramp rate during start-up is
prevented from being too high, as governed by the requirement to maintain
plasma stability in $l_i$-$q_\psi$ space, constraint equation no.\ 41 should
be turned on with iteration variable no.\ 66 ({\tt ftohs}).
\setlength{\parskip}{0mm}

\subsubsection{Current density limits}

The current density in the OH coil can be limited at the beginning-of-pulse
(BOP) and at the end-of-flat-top (EOF --- see
Figure~\ref{fig:current_vs_time}). The limiting value is dependent on the
maximum allowable stress in the coil as given by the value of {\tt sigpfalw}
in input block {\tt PFC}. To limit the current density at the BOP, constraint
equation no.\ 27 should be turned on with iteration variable no.\ 39 ({\tt
fjohc0}). To limit the current density at the EOF, constraint equation no.\ 26
should be turned on with iteration variable no.\ 38 ({\tt fjohc}).

\subsection{Blanket Model}

Switch {\tt lblnkt} in input block {\tt FWBLSH} determines whether the blanket
is to be simulated using a full thermodynamic model~\cite{Panos} ({\tt lblnkt
= 1}) or simply assumed to be made up of relevant materials (see
Section~\ref{sec:fwbs}) in user-defined proportions. The former model also
performs a self-consistent calculation of the thermal-to-electric conversion
efficiency, whereas the latter simply uses the input value {\tt etath}.
\setlength{\parskip}{5mm}

The following switches (all accessible via input block {\tt FWBLSH}) control
the details of the full thermodynamic model of the blanket.
\setlength{\parskip}{0mm}

\subsubsection{Cooling channel geometry}

The value of switch {\tt astr} determines whether the cooling channels have a
circular cross-section ({\tt astr = 1}) or an annular cross-section ({\tt astr
= 2}). The latter case is the default.

\subsubsection{Boundary condition}

The value of switch {\tt bstr} determines whether the coolant output
temperature is to be fixed ({\tt bstr = 1}) or whether the maximum blanket
temperature is to be fixed ({\tt bstr~=~2}). The former case is the default.
The desired coolant output temperature for {\tt bstr = 1} is set using input
parameter {\tt xtfo} in input block {\tt FWBLSH}, and the required maximum
blanket temperature is set using input parameter {\tt xtb} in input block {\tt
FWBLSH}.

\subsubsection{Coolant type}
\label{sec:costr}

The value of switch {\tt costr} determines the type of coolant used in the
first wall, blanket and shield. If {\tt costr = 1}, the coolant is assumed to
be gaseous helium. If {\tt costr = 2}, the coolant is assumed to be
pressurised water/steam, which is the default. This switch is used whether or
not the full blanket model is used, i.e.\ is independent of the setting of
switch {\tt lblnkt}.

\subsubsection{Cooling channel orientation}

The value of switch {\tt estr} determines whether the cooling channels lie in
the radial direction ({\tt estr = 1}) or in the poloidal direction ({\tt estr
= 2}). The former case is the default.

\subsubsection{Blanket material}

Switch {\tt smstr} determines whether a solid blanket of Li$_2$O-Be ({\tt
smstr = 1}), or a liquid blanket of LiPb-Li ({\tt smstr = 2}) is used. The
former is the default, and is the type assumed if {\tt lblnkt} $\not=$~{\tt
1}.

\subsection{Other Engineering Models}

\subsubsection{First wall, blanket and shield materials}
\label{sec:fwbs}

The various material fractions making up these components are all available to
be changed in input block {\tt FWBLSH}. The first wall coolant fraction is
given by the value of {\tt fwclfr}, which is calculated if a pulsed power
plant is being modelled ({\tt lpulse = 1}), and assumed to be the input value
otherwise. The shield coolant fraction is stored in input parameter {\tt
vfshld}. The structural material for both the first wall and shield is
stainless steel. The blanket is a more complicated structure, and can contain
either solid or liquid material as the neutron multiplier or tritium
synthesiser. Table~\ref{tab:blanket} summarises the possible options for the
blanket materials.

% Table summarising blanket materials

\begin{table}
\begin{center}

\begin{tabular}{||l||c||c||c||} \hline
material & {\tt lblnkt$\not=$1}: & \multicolumn{2}{c||}{{\tt lblnkt=1}: full
thermodynamic model} \\ \cline{3-4}
 & simple model & {\tt smstr=1}: solid blanket & {\tt smstr=2}: liquid blanket
\\ \hline
stainless steel & {\tt fblss}   & {\tt fblss}   & {\tt fblss}   \\
vanadium        & {\tt fblvd}   & {\tt fblvd}   & {\tt fblvd}   \\
Li$_2$O         & {\tt fblli2o} & {\tt fblli2o} &     ---       \\
beryllium       & {\tt fblbe}   & {\tt fblbe}   &     ---       \\
LiPb            &     ---       &     ---       & {\tt fbllipb} \\
lithium         &     ---       &     ---       & {\tt fblli}   \\
coolant         & {\tt vfblkt}  & {\tt vfblkt}  & {\tt vfblkt}  \\ \hline
\end{tabular}
\end{center}
\caption[TABLE_BKT]{{\it
Summary of the materials comprising the blanket in the various scenarios
available. The fractions given are all available to be modified, and should
of course add up to 1 for any given model. The type of coolant used is given
by the value of switch {\tt costr} (Section~\ref{sec:costr}).
}}
\label{tab:blanket}
\end{table}
 % Table showing blanket materials

\subsubsection{Divertor model}
\label{sec:divmod}

Switch {\tt idivrt} in input block {\tt PHYDAT}, controls the overall plasma
configuration. At present, it is only possible to use {\tt idivrt = 2}, which
corresponds to an up-down symmetric, double null configuration, as the PF coil
current scaling algorithms only allow for this model.
\setlength{\parskip}{5mm}

The Harrison-Kukushkin-Hotston divertor model~\cite{IPDG} developed for ITER
is used (except for the case of tight aspect ratio tokamaks --- see
later). This is appropriate for conventional aspect ratio machines, but care
should be taken in inputting the divertor magnetics for this model, and
projections far from the ITER CDA machine parameters are likely to be
unreliable.

The divertor calculations are carried out in routines {\tt DIVCALL} and {\tt
DIVERT}.
\setlength{\parskip}{0mm}

\subsubsection{Heat transport and power conversion}

Many of the power conversion efficiencies shown in Figure~\ref{fig:pwrconv}
can be adjusted by the user. The primary coolant is controlled by switch {\tt
costr} in input block {\tt FWBLSH}.  If {\tt costr = 1}, gaseous helium is
used; if {\tt costr = 2}, pressurised water coolant is assumed.

\subsubsection{Vacuum pump}

Switch {\tt ntype} in input block {\tt VACCY} controls whether a turbopump
({\tt ntype = 0}) or a cryopump ({\tt ntype = 1}) is used in the vacuum
system.

\subsection{Pulsed Plant Operation}

Switch {\tt lpulse} in input block {\tt PULSE} determines whether the power
plant is assumed to be based on steady-state ({\tt lpulse = 0}) or
pulsed ({\tt lpulse = 1}) operation.
\setlength{\parskip}{5mm}

The following sections describe the switches that control the use of the
pulsed power plant option, all of which are accessible via input block {\tt
PULSE}.
\setlength{\parskip}{0mm}

\subsubsection{First wall axial stress calculations}

Switch {\tt itcycl} activates the desired model for the first wall axial
stress calculations. If {\tt itcycl = 1} (the default), the wall is fully
constrained axially, and no bending can occur. If {\tt itcycl = 2}, there is
no constraint on the axial motion, but no bending can occur. Finally, if {\tt
itcycl = 3}, again there is no axial constraint, and bending is allowed to
occur.

\subsubsection{Thermal storage options}

During every cycle there is a period when no fusion power is produced. The net
electric output from the plant must, however, be maintained, and this is
achieved using thermal storage. There are three types of thermal storage
available within \PSC and the value of switch {\tt istore} determines which is
to be used. If {\tt istore = 1} (the default), option~1 of
Ref.~\cite{ELECTROWATT} is assumed, which utilises the thermal storage
inherent in the machine's steam cycle equipment. This should be used if the
machine down time is less than 100~seconds. If {\tt istore = 2}, option~2 of
Ref.~\cite{ELECTROWATT} is assumed, which uses the same method as before, but
augments it with an additional boiler. This may be used for machine down times
of up to 300~seconds. Finally, if {\tt istore = 3}, a large stainless steel
block acts as the thermal storage medium.

\subsubsection{First wall coolant temperature rise limit}

The rise in temperature of the first wall coolant can be limited to be no more
than the value of {\tt dtmpmx} by turning on constraint equation no.\ 38 with
iteration variable no.\ 62 ({\tt fdtmp}).

\subsubsection{First wall peak temperature limit}

The maximum first wall temperature can be limited to be no more than the value
of variable {\tt tpkmax} by turning on constraint equation no.\ 39 with
iteration variable no.\ 63 ({\tt ftpeak}).

\subsubsection{Minimum auxiliary power}

The auxiliary power reaching the plasma can be forced to be more than the
minimum allowable value {\tt auxmin} by turning on constraint equation no.\ 40
with iteration variable no.\ 64 ({\tt fauxmn}). The value of {\tt auxmin} is
determined by the code if the start-up model is activated, otherwise it may be
initialised via input block {\tt PULSE}.

\subsubsection{OH coil swing time}

The OH coil swing time can be constrained as described in
Section~\ref{sec:tohs}.

\subsubsection{Cycle time limit}

Due to fatigue induced by the cyclic stresses on the first wall, there is a
maximum number of pulse cycles that can be withstood by this component.
Therefore, for a given first wall lifetime (calculated by the code) there is a
minimum allowable time per cycle. The cycle time can be constrained to be at
least the minimum value by turning on constraint equation no.\ 42 with
iteration variable no.\ 67 ({\tt ftcycl}).

\section{Tight Aspect Ratio Tokamak Switches}

As stated earlier, there are many different models that are relevant only to
tight aspect ratio tokamaks. Switch \texttt{itart} in input block
\texttt{PHYDAT} provides overall control of the TART switches within the code,
and subroutine \texttt{CHECK} ensures that no conflicting values are
inadvertently set by the user in the input file. Table~\ref{tab:tart}
summarises the switch values relevant to each aspect ratio regime.
\begin{table}
\begin{center}
\begin{tabular}{||l|c|c||} \hline
 & conventional aspect ratio & tight aspect ratio \\
switch & \texttt{itart = 0} & \texttt{itart = 1} \\ \hline
ibss & 1, 2, 3 & 2, 3 \\
icurr & 1, 3, 4, 5, 6, 7 & 2 \\
ishape & 0 & 0, 1 \\
itfsup & 0, 1 & 0 \\ \hline
\end{tabular}
\end{center}
\caption{\textit{Summary of the switch values in PROCESS that relate to
conventional aspect ratio and tight aspect ratio machines.}}
\label{tab:tart}
\end{table}
In addition, \texttt{itart} controls the following features relevant to tight
aspect ratio:
\begin{enumerate}
\item The relative radial positions of the TF coil inboard legs and the OH coil
are different for a tight aspect ratio machine. If \texttt{itart = 1}, the
radial build is altered so that, starting from the centreline ($R = 0$), the
component order is: bucking cylinder, TF coil, gap, OH coil, cryostat, and
then continuing as in Figure~\ref{fig:build1}.
\item If \texttt{itart = 1}, a gaseous divertor concept~\cite{storac} is
assumed and a simple divertor heat load calculation is employed, rather than
the ITER-CDA like divertor assumed for conventional aspect ratio tokamaks.
\item Tight aspect ratio tokamaks have resistive TF coils that combine into a
single centrepost at the centre of the machine. If \texttt{itart = 1}, routine
\texttt{CNTRPST} is called to calculate various parameters relevant to the
centrepost, and these may be limited using constraint equations~43 to~46 if
required:
\begin{itemize}
\item Equation~43 is a consistency equation for the average centrepost
temperature.
\item Equation~44 can be used to limit the peak centrepost temperature to a
maximum value (\texttt{ptempalw}) using iteration variable no.\ 68
(\texttt{fptemp}).
\item Equation~45 can be used to force a lower limit to the edge safety
factor, using iteration variable no.\ 71 (\texttt{fq}).
\item Equation~46 can be used to apply an upper limit to the ratio of plasma
current to TF coil (``rod'') current, using iteration variable no.\ 72
(\texttt{fipir}).
\end{itemize}
\item A simple PF coil current scaling algorithm is used if \texttt{itart = 1}.
\end{enumerate}

\section{Cost Models}

The cost accounting used by \PS combines methods~\cite{cost1} used in the
TETRA code~\cite{tetra} and the Generomak~\cite{generomak} scheme.  The costs
are split into the standard accounting categories~\cite{cost2} generally used
in the reporting of power plant costs. The best references for the algorithms
used are~\cite{storac}, and source file {\tt costs.f} in the code itself.
\setlength{\parskip}{5mm}

The majority of the costed items have a unit cost associated with them. These
values scale with (for example) power output, volume, component mass etc., and
many are available to be changed in input block {\tt COSTINP}. All costs and
their algorithms correspond to 1990 dollars.

\setlength{\parskip}{0mm}
\subsection{Cost Options}

\subsubsection{$N^{th}$ of a kind costs}

The unit costs of the components of the fusion power core are relevant to
``first-of-a-kind'' items. That is to say, the items are assumed to be
relatively expensive to build as they are effectively prototypes and
specialised tools and machines have perhaps been made specially to create
them. However, if a ``production line'' has been set up, and R \& D progress
has allowed more experience to be gained in constructing the power core
components, the costs will be reduced as a result. Variable \texttt{fkind} in
input block \texttt{COSTINP} may be used to multiply the raw unit costs of the
fusion power core items (sensibly by a factor of less than one) to simulate
this cost reduction for an $N^{th}$-of-a-kind device. In other systems studies
of fusion power plants~\cite{galambos}, values for this multiplier have ranged
from 0.5 to 0.8.

\subsubsection{Level of safety assurance}

Many of the unit costs have four possible choices, relating to the level of
safety assurance~\cite{lsa} flag {\tt lsa} in input block {\tt COSTINP}. A
value {\tt lsa = 1} corresponds to a plant with a full safety credit (i.e.\ is
truly passively safe). Levels {\tt 2} and {\tt 3} lie between the two
extremes, and level {\tt 4} corresponds to a present day fission reactor, with
no safety credit.

\subsubsection{Replaceable components}

The first wall, blanket, divertor, centrepost (if present) and current drive
system have relatively short lifetimes because of their hostile environment,
after which they must be replaced. Because of this frequent renewal they can
be regarded as though they are ``fuel'' items, and can be costed
accordingly. Switch {\tt ifueltyp} in input block {\tt COSTINP} is used to
control whether this option is used in the code. If {\tt ifueltyp = 1}, the
costs of the first wall, blanket, divertor and a fraction {\tt fcdfuel} of the
cost of the current drive system are treated as fuel costs. If {\tt ifueltyp =
0}, these are treated as capital costs. Variable {\tt fcdfuel} is contained in
input block {\tt COSTINP}.

\subsubsection{Cost of electricity calculations}

Switch {\tt ireactor} in input block {\tt COSTINP} determines the type of cost
of electricity calculation that is performed. If {\tt ireactor = 0}, no cost
of electricity calculation is performed. If {\tt ireactor = 1}, then this
calculation {\em is}\/ performed, with the value quoted in units of
m\$/kWh.

\subsubsection{Net electric power calculation}

Related to the cost of electricity is the net electric power calculation
performed in routine {\tt POWER}. It is possible that the net electric power
can become negative due to a high recirculating power. Switch {\tt ipnet} in
input block {\tt COSTINP} determines whether the net electric power is scaled
to always remain positive ({\tt ipnet = 0}), or whether it is allowed to
become negative ({\tt ipnet = 1}), in which case no cost of electricity
calculation is performed.

\section{Other Switches and Models}

\subsection{Output Control}

Since the user may only be interested in a small proportion of the code's
output, a set of switches exist in input block {\tt OSECTS} that control
whether a given section of the output file is produced.
Table~\ref{tab:osections} indicates how these switches affect the output.

% Table summarising output section controllers

\begin{table}[ht]
\begin{center}

\begin{tabular}{||l|l||} \hline
switch      & relevant output section    \\ \hline
\tt sect01  & power plant   costs        \\
\tt sect02  & detailed costings          \\
\tt sect03  & plasma                     \\
\tt sect04  & current drive system       \\
\tt sect05  & divertor                   \\
\tt sect06  & machine build              \\
\tt sect07  & TF coils                   \\
\tt sect08  & PF coils                   \\
\tt sect09  & volt second consumption    \\
\tt sect10  & support structure          \\
\tt sect11  & PF coil inductances        \\
\tt sect12  & shield / blanket           \\
\tt sect13  & power conversion           \\
\tt sect14  & heat transport             \\
\tt sect15  & vacuum system              \\
\tt sect16  & plant buildings            \\
\tt sect17  & AC power                   \\
\tt sect18  & neutral beams              \\
\tt sect19  & electron cyclotron heating \\
\tt sect20  & Lower Hybrid heating       \\ 
\tt sect21  & times                      \\ \hline
\end{tabular}
\end{center}
\caption[TABLE_OSECT]{{\it
Summary of the switches in \PS that control the format of the output
file. If a switch has a value {\tt 0}, the relevant output section
does not appear in the output file. If its value is {\tt 1}, the
output section is included in the output file.
}}
\label{tab:osections}
\end{table}
 % Table showing output section controllers

\subsection{Code Parameters Affecting Other Models}

This chapter has summarised the methods by which several of the models in the
code can be activated. There are many others present, however, and it is
suggested that the user refers to the variable descriptor file, {\tt
var.des}. As stated earlier, this contains details of all the parameters
within the code that can be changed by the user, in order to customise the
machine modelled by \PSD

\setlength{\parskip}{5mm}

