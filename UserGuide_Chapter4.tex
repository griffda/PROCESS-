\mychapter{Execution of the Code}
\label{chap:run}

The intention of this chapter is to provide a comprehensive prescription for
setting up and performing runs with the code.  Firstly, the input file's
structure and format is described. The user is then taken through the
procedure for setting up the code to model a new machine, and finally an
attempt is made to indicate and solve the problems that the user will face
whilst trying to achieve a feasible solution.

\textbf{Please note:} There is now a prototype Graphical User Interface (GUI)
for people to use to run \process; see Section~\ref{sec:gui} for details.

\section{The Input File}
\label{sec:infile}

The input file \indat\ is used to change the values of the physics,
engineering and other code parameters from their default values, and to set up
the numerics (constraint equations, iteration variables etc.) required to
define the problem to be solved.

\subsection{Tokamak, stellarator, RFP or IFE?}
\label{sec:device}

In addition to the main input file \indat, a second input file
\texttt{device.dat} is used to signal to the code whether a tokamak,
stellarator, reversed field pinch or inertial fusion enery plant is to be
modelled. (If the file does not exist in the working directory, the standard
tokamak model is used.)

File \texttt{device.dat} should contain a single character in the first
line, which is interpreted as follows:
\begin{tabbing}
\hspace{15mm}\= \texttt{0} : use tokamak model \\
\> \texttt{1} : use stellarator model \\
\> \texttt{2} : use reversed field pinch model \\
\> \texttt{3} : use inertial fusion energy model
\end{tabbing}

\subsection{File structure}

The input file comprises a number of lines reminiscent of the Fortran
\texttt{NAMELIST} format used by some programs to read in data. Except for
comment lines that start with a \texttt{*} character, each line is of the form
\begin{verbatim}
variable = value
\end{verbatim}
where \texttt{variable} is the name of one of the input parameters or
iteration variables listed in the variable descriptor file, and \texttt{value}
is the (usually numerical) initial value required for that variable. (Arrays,
as opposed to scalar quantities, are treated differently --- see below.)
Variables can be specified in any order in the input file.

The routines that read in and parse the data from the input file are in source
file \texttt{input.f90}. They allow a great deal of error-trapping to be
carried out at the input stage.  All input data are screened for non-sensible
values directly; this is a useful feature of the code, since without such
intervention modern computers are notorious at not terminating programs when
an arithmetically impossible or undefined operation (``\texttt{NaN}'' error)
is encountered.

\subsection{Format rules}

The following rules must be obeyed when writing an input file:

\begin{enumerate}

\item Each variable must be on a separate line.

\item Variable names can be upper case, lower case, or a mixture of both.

\item Spaces may not appear within a variable name or data value.

\item Other spaces within a line, and trailing spaces, are ignored.

\item Commas are not necessary between variables (but see below).

\item Data can extend over more than one line.

\item One-dimensional arrays can be explicitly subscripted, or unscripted, in
  which case the following element order is assumed: \texttt{A(1), A(2),
    A(3),...}

\item At present, multiple dimension arrays can only be handled without
  reference to explicit subscripts, in which case the following element order
  is assumed: \texttt{B(1,1), B(2,1), B(3,1),...} The use of the input file to
  specify multiple dimension array elements is prone to error.

\item Unscripted array elements must be separated by commas.

\item Blank lines are allowed anywhere in the input file.

\item Lines starting with a \texttt{*} are assumed to be comments.

\item Comment lines starting with five or more asterisks (i.e.\
  \texttt{*****}) are reproduced verbatim in the output file. These should be
  used copiously to give a great deal of information about the run being
  performed, and should be updated before every single run of the code, as it
  is very easy to lose track of what is being attempted.

\item In-line comments are \textit{usually}\/ ignored, but there can be
  problems if one contains a comma (\texttt{,}). If this is the case, there
  must also be a comma after the variable's value and before the comment.

\end{enumerate}

It is useful to divide the input file into sections, using suitable comment
lines, to help the user keep related variables together.

The following is a valid fragment of an input file (the vertical lines are
simply to help show the column alignment):
\begin{center}
\begin{tabular}{||l}
$\!\!$\texttt{* This line is a comment that will not appear in the output} \\
$\!\!$\texttt{***** This line is a comment that will appear in the output} \\
$\!\!$\texttt{boundl(1) = 2.5,} \\
$\!\!$\texttt{BOUNDU(10) = 3.,} \\
$\!\!$\texttt{BOUNDU(45) = 1,} \\
$\!\!$\texttt{* Another comment... Note that real values can be entered as if} \\
$\!\!$\texttt{* they were integers, and vice versa (but it's not recommended...)} \\
$\!\!$\texttt{epsfcn = 10.e-4,} \\
$\!\!$\texttt{Ftol = 1.D-4,} \\
$\!\!$\texttt{* The next line sets the first five elements of array icc:} \\
$\!\!$\texttt{ICC =   2, 10, 11, 24, 31} \\
$\!\!$\texttt{* The next line sets the first ten elements of array ixc:} \\
$\!\!$\texttt{ixc =   10, 12, 3, 36, 48,} \\
$\!\!$\hspace{15mm}\texttt{1, 2, 6, 13, 16,} \\
$\!\!$\texttt{IOPTIMZ = 1,} \\
$\!\!$\texttt{maxcal = 200} \\
$\!\!$\texttt{ nsweep = 7} \\
$\!\!$\texttt{NEQNS = 5,    This is an in-line comment} \\
$\!\!$\texttt{NVAR = 10,    Another, but successfully containing a comma!} \\
\end{tabular}
\end{center}

The following are \textit{invalid}\/ entries in the input file
(Q: Why?!):
\begin{center}
\begin{tabular}{||l}
$\!\!$\texttt{boundl(1,1) = 2.5,} \\
$\!\!$\texttt{BOUNDU(N) = 3.,} \\
$\!\!$\texttt{A line of `random' characters like this will clearly wreak havoc} \\
$\!\!$\texttt{eps fcn = 10.e-4, ftol = 1.D-4} \\
$\!\!$\texttt{epsvmc = 1.0 e-4} \\
$\!\!$\texttt{ICC =   2  10  11  24  31} \\
$\!\!$\texttt{IOPTIMZ = 1.0,  This will in fact be okay - but is not recommended} \\
$\!\!$\texttt{NEQNS = 5    An in-line comment on a line with only one comma (,) character} \\
\end{tabular}
\end{center}

If the code encounters a problem reading the input file, it will stop immediately
with a (hopefully) useful error message. It may be worth looking at the
contents of the output file as well, to help narrow down on which line of the
input file the problem might lie.

\section{The Output File}

The main output from the code is sent to file \outdat\ in the working
directory.

A second file, \mfile, is also produced in the working directory.  This file
contains most of the same data as \outdat\ but in a different format, and has
been designed to be ``machine-readable'' by some of the utility programs
described in Chapter~\ref{chap:utilities}, to allow simple post-processing and
graphical output to be produced easily.

\section{Running the Code}

This section will attempt to guide the user through the actual running of the
code in its various modes. In most cases only minor changes to the input file
are necessary to change the code's mode of operation --- usually the physics
and engineering variables, etc.\ remain unchanged, with the major differences
occurring in the numerical input only.

\subsection{Environment set-up}
\label{sec:run_environment}

Under normal circumstances, the code is only available for execution from the
CCFE Fusion Unix Network, on which you must have an account.

To set up your environment to be able to run the code and the associated
Python utilities (see Chapter~\ref{chap:utilities}), add the following lines
to your \texttt{.bashrc} file:
\begin{quote}
\begin{verbatim}
module use /home/pknight/modules
module swap python
module load process/master
\end{verbatim}
\end{quote}
(If you want to use --- with care! --- the latest draft version of the code
instead of the latest verified release version, replace \texttt{module load
  process/master} with \texttt{module load process/develop} above.)

After re-logging in, execute \process\ by simply typing \texttt{process} on
the command line.

\subsection{Non-optimisation mode}

Non-optimisation mode is used to perform benchmark comparisons, whereby the
machine size, output power etc.\ are known and one only wishes to find the
calculated stresses, beta values and fusion powers, for example. When starting
to model a new machine, \process\ should always be run first in non-optimisation
mode, before any attempt is made to optimise the machine's parameters.

The first thing to do is to add to the input file all the known details about
the machine to be modelled. This may include some or all of the following:
\begin{itemize}
\item machine build
\item plasma aspect ratio
\item PF coil locations
\item type of current drive to be used
\item net electric power
\item various physics parameters, e.g.
  \begin{itemize}
  \item toroidal field on axis
  \item electron density
  \item electron temperature
  \item elongation
  \item triangularity
  \item beta $g$ coefficient
  \item edge safety factor
  \end{itemize}
\end{itemize}

In addition, some of the switch values summarised in Chapter~\ref{chap:models}
may have to be altered from their default values.

Next, the relevant numerics information must be entered. Switch
\texttt{ioptimz} must be set to \texttt{-1} for non-optimisation mode. Then
the user must decide which constraint equations and iteration variables to
activate --- this choice is dictated partly by the information required by the
user, and partly by the machine being modelled itself.

As stated earlier, all the relevant consistency equations must be activated,
together with the corresponding iteration variables. A number of limit
equations can also be activated, to investigate how the calculated values
compare with the physics or engineering limits.  The following is part of an
example non-optimisation input file:
\begin{verbatim}
IOPTIMZ = -1
NEQNS = 8
NVAR = 8
ICC =  1,  2, 10, 11,  7, 16,  5, 24,
IXC =  5, 10, 12, 29,  7,      9, 36, 4,

FPNETEL = 1.0
PNETELIN = 1200.0
\end{verbatim}
\vspace{-6mm}
$\vdots$

Consistency equations 1, 2, 10, 11 and 7 are activated, together with limit
equations~16, 5 and 24 (refer to Table~\ref{tab:eqns}). This example assumes
that neutral beam current drive is present (equation~7 with variable~7), and
that the net electric power is to be fixed at 1200~MW\@. Note the optional
(but beneficial) practice of vertically aligning corresponding equations and
variables --- constraint equation~16 has no corresponding iteration variable
(which would normally be no.\ 25, \texttt{fpnetel}), as we want the net
electric power to be fixed at the value given by \texttt{pnetelin}. Since in
non-optimisation mode, the number of variables must be equal to the number of
equations, we have scope to add a ``free'' iteration variable (refer to
Tables~\ref{tab:itvars1} and~\ref{tab:itvars2}), in this case no.\ 4 ---
electron temperature, to help raise the fusion power sufficiently to obtain
the required net electric power. Finally, note the use of the density and beta
limit equations (\texttt{5} and \texttt{24}, respectively); the final values
of the corresponding f-values will indicate if the limits are exceeded and by
how much.

On running \process\ and (hopefully) achieving a feasible result, examination
of the output may well show up discrepancies between some of the parameter
values produced and their known values (if available). Remember that, of all
the variables shown in the variable descriptor file with a default value, only
those declared as \textit{active iteration variables}\/ can change from their
initial values, whether they are set in the input file or in the relevant
module. However some of the calculated parameters may be wrong, the most
common of which are as follows:
\begin{itemize}

\item Plasma current; this can be adjusted using the edge safety factor
  \texttt{q}: $I_p \propto 1/q$

\item Fusion power; this scales roughly with the density profile factor
  \texttt{alphan}.

\item Build parameters; it may be necessary to change non-critical thicknesses
  to achieve the correct machine build.

\end{itemize}

It may still be difficult, if not impossible, to reconcile the fusion power
and the net electric power with the required values. This may well be due to
the power conversion efficiency values being used --- refer to
Figure~\ref{fig:powerflow3}.

With luck, a few iterations of this process will produce an adequate benchmark
case. A typical input file for use with \process\ in non-optimisation mode is
contained in Appendix~\ref{app:infile1}.

\subsection{Optimisation mode}
\label{sec:optim}

Running \process\ in optimisation mode requires few changes to be made to the
input file from the non-optimisation case. The main differences between
optimisation mode and non-optimisation mode are:

\begin{enumerate}

\item Optimisation mode applies lower and upper bounds to all active iteration
  variables.

\item There is no upper limit to the number of active iteration variables in
  optimisation mode.

\item A figure of merit must be specified in optimisation mode.

\item Scans can be performed in optimisation mode.

\end{enumerate}

Switch \texttt{ioptimz} must be set to \texttt{0} or \texttt{1} for
optimisation mode. If \texttt{ioptimz = 0}, a non-optimisation pass is
performed first, to provide a (hopefully) feasible set of initial conditions;
if \texttt{ioptimz = 1}, this is skipped and the code runs in optimisation
mode from the start. However, it is recommended to use \texttt{ioptimz = 1}
for optimisation runs, as using the HYBRD step before a \vmcon\ iteration
tends to cause \process\ to fail to converge more often.

As before, the user must decide which constraint equations and iteration
variables to activate. Again, the choice depends largely on the information
required by the user and the extent of the freedom that the code may have with
the machine's parameters.

The following is part of an example optimisation input file:
\begin{verbatim}
IOPTIMZ = 1
NEQNS = 16
NVAR = 19
ICC =  1,  2, 10, 11,  7, 16,  5, 24, 14,  8, 31, 32, 33, 34, 35, 36,
IXC =  5, 10, 12, 29,  7,      9, 36, 19, 14, 48, 49, 50, 51, 53, 54,
4, 6, 1, 18,
BOUNDL(1) = 2.5
BOUNDU(10) = 2.0
MINMAX = 6

FPNETEL = 1.0
PNETELIN = 1200.0
WALALW = 4.4

ISWEEP = 3
NSWEEP = 11
SWEEP = 3.5, 3.7, 3.9
\end{verbatim}
\vspace{-6mm}
$\vdots$

The figure of merit in this example is the (minimum) cost of electricity
(\texttt{minmax = 6}). Note that additional limit equations are now active,
along with a second consistency equation related to the neutral beam current
drive --- the number of decay lengths to the plasma centre is constrained to
be equal to the input value (\texttt{tbeamin}, which is not shown here).
Furthermore, there are now more iteration variables than constraint equations,
to aid the minimisation process.  Finally, note that a three-point scan in the
beta $g$ coefficient \texttt{dnbeta} --- scanning variable~11, is to be
performed.

A useful practice in optimisation mode is to perform ``stationary'' scans,
whereby the same value is given to the scanning variable on successive
iterations. This provides a check as to how well converged the solution has
become. If scans of a given variable are to be made over a large range of
values, it is often a good idea to start the scan in the middle of the desired
range, and to split the scan in two --- one going downwards from the initial
value, and the other upwards.  This ensures that the whole range of the scan
produces well-converged machines (assuming a ``good'' initial point), without
sharp changes in gradient in the parameter values.

It should be remembered that the value of the scan variable is set in the
array \texttt{sweep}, and this overrules any value set for the variable
elsewhere in the input file. For instance, in the example above, the values of
\texttt{dnbeta} set in the \texttt{sweep} array would overrule any value for
\texttt{dnbeta} set elsewhere in the file.

The output from an optimisation run contains an indication as to which
iteration variables lie at their limiting values. On the whole there is a
greater chance of unfeasible solutions being found whilst in optimisation
mode, and Section~\ref{sec:problems} will hopefully be of some use in this
situation. A typical input file for use with \process\ in optimisation mode
is contained in Appendix~\ref{app:infile2}.

\section{Problem Solving}
\label{sec:problems}

Experience has shown that the first few attempts at running \process\ with a
new input file tends to produce unfeasible results --- that is, the code will
not find a consistent set of machine parameters. The highly non-linear nature
of the numerics of \process\ is the reason for this difficulty, and it often
requires a great deal of painstaking adjustment of the input file to overcome.

\subsection{Error handling}
\label{sec:errors}

In general, errors detected during a run are handled in a consistent manner,
with the code producing (hopefully) useful diagnostic messages to help the
user understand what has happened.

There are three levels of ``error'' that may occur:
\begin{description}

\item{Level 1:} An \textit{informational}\/ message is produced under certain
  conditions, for example if the code has modified the user's input choice for
  some reason.

\item{Level 2:} A \textit{warning}\/ message is produced if a non-fatal situation has
  occurred that may result in an output case that is inaccurate or
  unreliable in some way.

\item{Level 3:} An \textit{error}\/ message will occur if a severe or fatal
  error has occurred and the program cannot continue.

\end{description}

These messages are printed on the screen during the course of a run, and those
still active at the final (feasible or unfeasible) solution point are also
written to the end of the output file (messages encountered during the
iteration process are not copied to the output file, as the convergence to a
valid solution might resolve some of the warnings produced earlier in the
solution process).

The \texttt{error\_status} variable returns the highest severity level that
has been encountered (or zero if no abnormal conditions have been found); if a
severe error (level 3) is flagged at any point the program is terminated
immediately. The final message number encountered during a run is returned via
output variable \texttt{error\_id}. In addition, with certain messages, a
number of diagnostic values may also be given; these can be used to provide
extra diagnostic information if the source code is available.

\subsection{General problems}

A code of the size and complexity of \process\ contains myriads of equations
and variables. Virtually everything depends indirectly on everything else
because of the nature of the code structure, so perhaps it is not surprising
that it is often difficult to achieve a successful outcome.

Naturally, problems will occur if some of the parameters become unphysical.
For example, if the aspect ratio becomes less than or equal to one, then we
must expect problems to appear. For this reason, the default bounds on the
iteration variables and the allowed ranges of all the input variables have
been selected with great care.

The code contains a large (though probably not exhaustive) number of error
traps to try and prevent problems from propagating. These include tests for
unphysical values, and checks to prevent divisions by zero, and non-sensible
arguments for logarithms and square roots, etc. However, occasionally
arithmetic (``\texttt{NaN}'') errors still occur, although their incidence is
low. They now usually only occur due to unfeasibility problems (see later).

The error messages produced by the code attempt to provide diagnostic
information, telling the user where the problem occurs, and also suggest a
possible solution. These messages are out of necessity brief, and so cannot
promise to lead to a more successful outcome.

For expert users, there is the option to turn on extra debugging output; to do
this, set \texttt{verbose = 1} in the input file.

\subsection{Optimisation problems}

On reflection it is perhaps surprising that \process\ ever does manage to
find the global minimum figure of merit value, since if there are
\texttt{nvar} iteration variables active the search is over
\texttt{nvar}-dimensional parameter space, in which there may be many shallow
minima of approximately equal depth. Remember that \texttt{nvar} is usually of
the order of twenty.

The machine found by \process\ may not, therefore, be the absolutely optimal
device. It is quite easy to have two or more solutions, with results only a
few per cent different, but a long way apart in parameter space. The technique
of ``stationary'' scans described in Section~\ref{sec:optim} above can often
help in this situation, which is why this method is recommended at all times.

Scans should be started in the middle of a range of values, to try to keep the
scan within the same family of machines. The optimum machine found may
otherwise suddenly jump to a new region of parameter space, causing the output
variables to seem to vary unpredictably with the scanning variable.

It should be noted that in general the machine produced by \process\ will
always sit against one or more operation limits. If, during a scan, the limit
being leant upon changes (i.e.\ if the machine jumps from leaning on the beta
limit to leaning on the density limit) the output parameters may well become
discontinuous in gradient, and trends may suddenly change direction.

\subsection{Unfeasible results}

In the numerics section of the output file, the code indicates whether the run
produced a feasible or unfeasible result.

The former implies a successful outcome, although it is always worth checking
that the estimate of the constraints (\texttt{sqsumsq}) is small ($\sim
10^{-3}$ or less); the code will issue a warning if the run seems feasible but
the value of \texttt{sqsumsq} exceeds $10^{-2}$. If this occurs, reducing the
value of the HYBRD tolerance \texttt{ftol} or \vmcon\ tolerance
\texttt{epsvmc} (as appropriate) should indicate whether the result is valid
or not; the output can usually be trusted if (1) the constraint
residues\footnote{The constraint residues are the final values of $c_i$ in the
  constraint equations --- see Section~\ref{sec:constraints}. The value
  \texttt{sqsumsq} is the square root of the sum of the squares of these
  residuals.} fall as the tolerance is reduced to about $10^{-8}$, and (2) the
code indicates that a feasible solution is still found.

An unfeasible result occurs if \process\ cannot find a set of values for the
iteration variables which satisfies all the given constraints. In this case,
the values of the constraint residues shown in the output give some indication
of which constraint equations are not being satisfied --- those with the
highest residues should be examined further. In optimisation mode, the code
also indicates which iteration variables lie at the edge of their allowed
range.

Unfeasible runs are caused either by ill-defining the problem to be solved, or
by starting the problem in an unfavourable region of parameter space. The
latter can be checked simply by changing the initial values of the
\textit{active}\/ iteration variables in the input file, but the former
requires some extra work. This situation arises if there are insufficient
iteration variables for the given constraint equations. It is important to
choose the right number of \textit{useful}\/ iteration variables for the
problem to be solved --- it is possible to activate too many iteration
variables as well as too few, some of which may be redundant.

Both optimisation and non-optimisation runs can fail with an error message
suggesting that the iteration process is not making good progress. This is
likely to be due to the code finding itself unable to escape a region of the
parameter space where the minimum in the residuals is significantly above
zero. In this situation, there is either no solution possible (the residuals
can therefore never approach zero), or the topology of the local minimum makes
it difficult for the code to escape to the global minimum. Again, a helpful
technique is to either change the list of iteration variables in use, or to
simply modify their initial values to try to help the code avoid such regions.

A technique that occasionally removes problems due to unfeasible results,
particularly if an error code \texttt{ifail = 3} is encountered during an
optimisation run, is to adjust slightly one of the limits imposed on the
iteration variables, even if the limit in question has not been reached. This
subtly alters the gradients computed by the code during the iteration process,
and may tip the balance so that the code decides that the device produced is
feasible after all. For instance, a certain component's temperature might be
400~K, and its maximum allowable temperature is 1000~K\@. Adjusting this limit
to 900~K (which will make no difference to the \textit{actual}\/ temperature)
may be enough to persuade the code that it has found a feasible solution.

Similarly, the order in which the constraint equations and iteration variables
are stored in the \texttt{icc} and \texttt{ixc} arrays can make the difference
between a feasible and unfeasible result. This seemingly illogical behaviour
is, sadly, typical of the way in which the code works.

Another technique in such situations may be to change the finite difference
step length \texttt{epsfcn}, as this might subtly change the path taken in the
approach towards a solution.

Unfeasible cases often produce unrealistic machines, so one should not believe
the output values from these runs. Unfortunately, the stationary scan method
sometimes, though not always, fails to help these cases, since it will tend to
keep starting the run at the same point. Ill-defined problems sometimes
produce arithmetic errors, for obscure reasons.

Though a great deal of work has been performed on the code to improve its
standard, there can be no guarantee that \process\ is entirely bug-free,
simply because of its large size. Rarely, then, it may be that an unfeasible
result indicates that the code has encountered a programming error, although
its precise location will be almost impossible to find by simply examining the
output file.

It may be the case that the act of satisfying all the required constraints is
impossible. No machine can exist if the allowed operating regime is too
restrictive, or if two constraint equations require conflicting
(non-overlapping) parameter spaces. In this case some relaxation of the
requirements is needed for the code to produce a successful machine design.

\subsection{Hints}

The above sections should indicate that it is the complex interplay between
the constraint equations and the iteration variables that determines whether
the code will be successful at producing a useful result. It can be a somewhat
laborious process to arrive at a working case, and (unfortunately, perhaps)
experience is often of great value in this situation.

It should be remembered that sufficient iteration variables should be used to
solve each constraint equation. For instance, a particular limit equation may
be $A \leq B$, i.e.\ $A = fB$, where the f-value $f$ must lie between zero and
one for the relation to be satisfied.  However, if none of the iteration
variables have any effect on the values of $A$ and $B$, and $A$ happens to be
\textit{greater}\/ than $B$, then \process\ will clearly not be able to solve
the constraint.

The lower and upper bounds of the iteration variables are all available to be
changed in the input file. Constraints can be relaxed in a controlled manner
by moving these bounds, although in some cases care should be taken to ensure
that unphysical values cannot occur.  The code indicates which iteration
variables lie at the edge of their range.

It is suggested that constraint equations should be added one at a time, with
sufficient new iteration variables activated at each step.  If the situation
becomes unfeasible it can be helpful to reset the initial iteration variable
values to those shown in the output from a previous feasible case, and rerun
the code.

Finally, it should be borne in mind that the machine that is envisaged may not
be a valid solution to the constraints being imposed, no matter how many
degrees of freedom (i.e.\ iteration variables) are available. In this case,
and many others, the user has to relax the constraints slowly until a feasible
result is found.

