\documentclass[]{article}
\usepackage{lmodern}
\usepackage{amssymb,amsmath}
\usepackage{ifxetex,ifluatex}
\usepackage{fixltx2e} % provides \textsubscript
\ifnum 0\ifxetex 1\fi\ifluatex 1\fi=0 % if pdftex
  \usepackage[T1]{fontenc}
  \usepackage[utf8]{inputenc}
\else % if luatex or xelatex
  \ifxetex
    \usepackage{mathspec}
  \else
    \usepackage{fontspec}
  \fi
  \defaultfontfeatures{Ligatures=TeX,Scale=MatchLowercase}
\fi
% use upquote if available, for straight quotes in verbatim environments
\IfFileExists{upquote.sty}{\usepackage{upquote}}{}
% use microtype if available
\IfFileExists{microtype.sty}{%
\usepackage[]{microtype}
\UseMicrotypeSet[protrusion]{basicmath} % disable protrusion for tt fonts
}{}
\PassOptionsToPackage{hyphens}{url} % url is loaded by hyperref
\usepackage[unicode=true]{hyperref}
\hypersetup{
            pdftitle={Variable Descriptor},
            pdfborder={0 0 0},
            breaklinks=true}
\urlstyle{same}  % don't use monospace font for urls
\IfFileExists{parskip.sty}{%
\usepackage{parskip}
}{% else
\setlength{\parindent}{0pt}
\setlength{\parskip}{6pt plus 2pt minus 1pt}
}
\setlength{\emergencystretch}{3em}  % prevent overfull lines
\providecommand{\tightlist}{%
  \setlength{\itemsep}{0pt}\setlength{\parskip}{0pt}}
\setcounter{secnumdepth}{0}
% Redefines (sub)paragraphs to behave more like sections
\ifx\paragraph\undefined\else
\let\oldparagraph\paragraph
\renewcommand{\paragraph}[1]{\oldparagraph{#1}\mbox{}}
\fi
\ifx\subparagraph\undefined\else
\let\oldsubparagraph\subparagraph
\renewcommand{\subparagraph}[1]{\oldsubparagraph{#1}\mbox{}}
\fi

% set default figure placement to htbp
\makeatletter
\def\fps@figure{htbp}
\makeatother


\title{Variable Descriptor}
\date{}

\begin{document}
\maketitle

\subsection{PROCESS Variable Descriptor File : dated
20190509}\label{process-variable-descriptor-file-dated-20190509}

\begin{center}\rule{0.5\linewidth}{\linethickness}\end{center}

Variables labelled with FIX are initialised with the given default value
(shown between / / characters), but currently are not available to be
changed in the input file.

All other variables shown with a default value (including arrays boundl,
boundu and sweep) can be changed in the input file.

Variables not shown with a default value are calculated within PROCESS,
so need not be initialised.

\begin{center}\rule{0.5\linewidth}{\linethickness}\end{center}

\subsubsection{\texorpdfstring{\href{global_variables.html}{global\_variables}}{global\_variables}}\label{global_variables}

\begin{itemize}
\tightlist
\item
  icase : power plant type
\item
  runtitle /Run Title/ : short descriptive title for the run
\item
  verbose /0/ : switch for turning on/off diagnostic messages:

  \begin{itemize}
  \tightlist
  \item
    = 0 turn off diagnostics
  \item
    = 1 turn on diagnostics
  \end{itemize}
\item
  run\_tests /0/ : Turns on built-in tests if set to 1
\item
  maxcal /200/ : maximum number of VMCON iterations
\end{itemize}

\subsubsection{\texorpdfstring{\href{constants.html}{constants}}{constants}}\label{constants}

\begin{itemize}
\tightlist
\item
  degrad FIX : degrees to radians, = pi/180
\item
  echarge FIX : electron charge (C)
\item
  mproton FIX : proton mass (kg)
\item
  pi FIX : famous number
\item
  rmu0 FIX : permeability of free space, 4.pi x 10\^{}(-7) H/m
\item
  twopi FIX : 2 pi
\item
  umass FIX : unified atomic mass unit (kg)
\end{itemize}

\subsubsection{\texorpdfstring{\href{physics_variables.html}{physics\_variables}}{physics\_variables}}\label{physics_variables}

\begin{itemize}
\tightlist
\item
  ipnlaws /46/ FIX : number of energy confinement time scaling laws
\item
  abeam : beam ion mass (amu)
\item
  afuel : average mass of fuel portion of ions (amu)
\item
  aion : average mass of all ions (amu)
\item
  alphaj /1.0/ : current profile index; calculated from q0, q if
  iprofile=1
\item
  alphan /0.25/ : density profile index
\item
  alphap : pressure profile index
\item
  alpharate : alpha particle production rate (particles/m3/sec)
\item
  alphat /0.5/ : temperature profile index
\item
  aspect /2.907/ : aspect ratio (iteration variable 1)
\item
  beamfus0 /1.0/ : multiplier for beam-background fusion calculation
\item
  beta /0.042/ : total plasma beta (iteration variable 5) (calculated if
  ipedestal =3)
\item
  betaft : fast alpha beta component
\item
  betalim : allowable beta
\item
  betanb : neutral beam beta component
\item
  betap : poloidal beta
\item
  normalised\_total\_beta : normaised total beta
\item
  betbm0 /1.5/ : leading coefficient for NB beta fraction
\item
  bp : poloidal field (T)
\item
  bt /5.68/ : toroidal field on axis (T) (iteration variable 2)
\item
  btot : total toroidal + poloidal field (T)
\item
  burnup : fractional plasma burnup
\item
  bvert : vertical field at plasma (T)
\item
  cfe0 /0.0/ : seeded high-Z impurity fraction (n\_highZ / n\_e)
  (imprad\_model=0 only) (iteration variable 43)
\item
  csawth /1.0/ : coeff. for sawteeth effects on burn V-s requirement
\item
  cvol /1.0/ : multiplying factor times plasma volume (normally=1)
\item
  cwrmax /1.35/ : maximum ratio of conducting wall distance to plasma
  minor radius for vertical stability (constraint equation 23)
\item
  dene /9.8e19/ : electron density (/m3) (iteration variable 6)
  (calculated if ipedestal=3)
\item
  deni : fuel ion density (/m3)
\item
  dlamee : electron-electron coulomb logarithm
\item
  dlamie : ion-electron coulomb logarithm
\item
  dlimit(7) : density limit (/m3) as calculated using various models
\item
  dnalp : thermal alpha density (/m3)
\item
  dnbeam : hot beam ion density, variable (/m3)
\item
  dnbeam2 : hot beam ion density from calculation (/m3)
\item
  dnbeta /3.5/ : (Troyon-like) coefficient for beta scaling; calculated
  as (4.0*rli) if iprofile=1 (see also gtscale option)
\item
  dnelimt : density limit (/m3)
\item
  dnitot : total ion density (/m3)
\item
  dnla : line averaged electron density (/m3)
\item
  dnprot : proton ash density (/m3)
\item
  dntau : plasma average ``n-tau'' (seconds/m3)
\item
  dnz : high Z ion density (/m3)
\item
  ealphadt /3520.0/ FIX : alpha birth energy in D-T reaction (keV)
\item
  epbetmax /1.38/ : maximum (eps*beta\_poloidal) (constraint equation 6)
  revised (07/01/16) Issue \#346 ``Operation at the tokamak equilibrium
  poloidal beta-limit in TFTR'' 1992 Nucl. Fusion 32 1468
\item
  eps : inverse aspect ratio
\item
  faccd : fraction of plasma current produced by auxiliary current drive
\item
  facoh : fraction of plasma current produced inductively
\item
  falpe : fraction of alpha energy to electrons
\item
  falpha /0.95/ : fraction of alpha power deposited in plasma (Physics
  of Energetic Ions, p.2489)
\item
  falpi : fraction of alpha power to ions
\item
  fbfe /0.35/ : fraction of high-Z radiation to Bremsstrahlung
  (imprad\_model=0 only)
\item
  fdeut /0.5/ : deuterium fuel fraction
\item
  ffwal /0.92/ : factor to convert plasma surface area to first wall
  area in neutron wall load calculation (iwalld=1)
\item
  fgwped /0.85/ : fraction of Greenwald density to set as pedestal-top
  density If \textless{}0, pedestal-top density set manually using neped
  (ipedestal\textgreater{}=1) Needs to be \textgreater{}0 if ipedestal =
  3 (iteration variable 145)
\item
  fgwsep /0.50/ : fraction of Greenwald density to set as separatrix
  density If \textless{}0, separatrix density set manually using nesep
  (ipedestal\textgreater{}=1) Needs to be \textgreater{}0 if ipedestal =
  3
\item
  fhe3 /0.0/ : helium-3 fuel fraction
\item
  figmer : physics figure of merit (= plascur*aspect**sbar, where
  sbar=1)
\item
  fkzohm /1.0/ : Zohm elongation scaling adjustment factor (ishape=2, 3)
\item
  fplhsep /1.0/ : F-value for Psep \textgreater{}= Plh + Paux
  (constraint equation 73)
\item
  ftrit /0.5/ : tritium fuel fraction
\item
  fusionrate : fusion reaction rate (reactions/m3/sec)
\item
  fvsbrnni /1.0/ : fraction of the plasma current produced by
  non-inductive means (iteration variable 44)
\item
  gamma /0.4/ : Ejima coefficient for resistive startup V-s formula
\item
  gammaft : ratio of (fast alpha + neutral beam beta) to thermal beta
\item
  gtscale /0/ : switch for a/R scaling of dnbeta (iprofile=0 only):

  \begin{itemize}
  \tightlist
  \item
    = 0 do not scale dnbeta with eps;
  \item
    = 1 scale dnbeta with eps
  \end{itemize}
\item
  hfac(ipnlaws) : H factors for an ignited plasma for each energy
  confinement time scaling law
\item
  hfact /1.0/ : H factor on energy confinement times, radiation
  corrected (iteration variable 10). If ipedestal=2 or 3 and hfact = 0,
  not used in PLASMOD (see also plasmod\_i\_modeltype)
\item
  taumax /10/ : Maximum allowed energy confinement time (s)
\item
  ibss /3/ : switch for bootstrap current scaling:

  \begin{itemize}
  \tightlist
  \item
    = 1 ITER 1989 bootstrap scaling (high R/a only);
  \item
    = 2 for Nevins et al general scaling;
  \item
    = 3 for Wilson et al numerical scaling;
  \item
    = 4 for Sauter et al scaling
  \end{itemize}
\item
  iculbl /0/ : switch for beta limit scaling (constraint equation 24):

  \begin{itemize}
  \tightlist
  \item
    = 0 apply limit to total beta;
  \item
    = 1 apply limit to thermal beta;
  \item
    = 2 apply limit to thermal + neutral beam beta
  \end{itemize}
\item
  icurr /4/ : switch for plasma current scaling to use:

  \begin{itemize}
  \tightlist
  \item
    = 1 Peng analytic fit;
  \item
    = 2 Peng double null divertor scaling (ST);
  \item
    = 3 simple ITER scaling (k = 2.2, d = 0.6);
  \item
    = 4 later ITER scaling, a la Uckan;
  \item
    = 5 Todd empirical scaling I;
  \item
    = 6 Todd empirical scaling II;
  \item
    = 7 Connor-Hastie model;
  \item
    = 8 Sauter scaling allowing negative triangularity
  \end{itemize}
\item
  idensl /7/ : switch for density limit to enforce (constraint equation
  5):

  \begin{itemize}
  \tightlist
  \item
    = 1 old ASDEX;
  \item
    = 2 Borrass model for ITER (I);
  \item
    = 3 Borrass model for ITER (II);
  \item
    = 4 JET edge radiation;
  \item
    = 5 JET simplified;
  \item
    = 6 Hugill-Murakami Mq limit;
  \item
    = 7 Greenwald limit
  \end{itemize}
\item
  idivrt : number of divertors (calculated from snull)
\item
  ifalphap /1/ : switch for fast alpha pressure calculation:

  \begin{itemize}
  \tightlist
  \item
    = 0 ITER physics rules (Uckan) fit;
  \item
    = 1 Modified fit (D. Ward) - better at high temperature
  \end{itemize}
\item
  ifispact /0/ : switch for neutronics calculations:

  \begin{itemize}
  \tightlist
  \item
    = 0 neutronics calculations turned off;
  \item
    = 1 neutronics calculations turned on
  \end{itemize}
\item
  igeom /1/ : switch for plasma geometry calculation:

  \begin{itemize}
  \tightlist
  \item
    = 0 original method (possibly based on Peng ST modelling);
  \item
    = 1 improved (and traceable) method
  \end{itemize}
\item
  ignite /0/ : switch for ignition assumption:

  \begin{itemize}
  \tightlist
  \item
    = 0 do not assume plasma ignition;
  \item
    = 1 assume ignited (but include auxiliary power in costs)
  \end{itemize}

  Obviously, ignite must be zero if current drive is required. If
  ignite=1, any auxiliary power is assumed to be used only during plasma
  start-up, and is excluded from all steady-state power balance
  calculations.
\item
  iinvqd /1/ : switch for inverse quadrature in L-mode scaling laws 5
  and 9:

  \begin{itemize}
  \tightlist
  \item
    = 0 inverse quadrature not used;
  \item
    = 1 inverse quadrature with Neo-Alcator tau-E used
  \end{itemize}
\item
  impc /1.0/ : carbon impurity multiplier (imprad\_model=0 only)
\item
  impo /1.0/ : oxygen impurity multiplier (imprad\_model=0 only)
\item
  ipedestal /1/ : switch for pedestal profiles:

  \begin{itemize}
  \tightlist
  \item
    = 0 use original parabolic profiles;
  \item
    = 1 use pedestal profiles
  \item
    = 2 use pedestal profiles and run PLASMOD on final output
  \item
    = 3 use PLASMOD transport model only to calculate pedestal profiles
  \end{itemize}
\item
  iscdens /0/ : switch for pedestal profiles: OBSOLETE
\item
  ieped /0/ : switch for scaling pedestal-top temperature with plasma
  parameters:

  \begin{itemize}
  \tightlist
  \item
    = 0 set pedestal-top temperature manually using teped;
  \item
    = 1 set pedestal-top temperature using EPED scaling; (PLASMOD
    implementation of scaling within PLASMOD, ipedestal =2,3)
  \item
    https://idm.euro-fusion.org/?uid=2MSZ4T
  \end{itemize}
\item
  eped\_sf /1.0/ : Adjustment factor for EPED scaling to reduce pedestal
  temperature or pressure to mitigate o
\item
  neped /4.0e19/ : electron density of pedestal {[}m-3{]}
  (ipedestal=1,2, calculated if 3)
\item
  nesep /3.0e19/ : electron density at separatrix {[}m-3{]}
  (ipedestal=1,2, calculated if 3)
\item
  alpha\_crit : critical ballooning parameter value
\item
  nesep\_crit : critical electron density at separatrix {[}m-3{]}
\item
  plasma\_res\_factor /1.0/ : plasma resistivity pre-factor
\item
  rhopedn /1.0/ : r/a of density pedestal (ipedestal\textgreater{}=1)
\item
  rhopedt /1.0/ : r/a of temperature pedestal
  (ipedestal\textgreater{}=1)
\item
  tbeta /2.0/ : temperature profile index beta (ipedestal=1,2)
\item
  teped /1.0/ : electron temperature of pedestal (keV)
  (ipedestal\textgreater{}=1, ieped=0, calculated for ieped=1)
\item
  tesep /0.1/ : electron temperature at separatrix (keV)
  (ipedestal\textgreater{}=1) calculated if reinke criterion is used
  (icc = 78)
\item
  iprofile /1/ : switch for current profile consistency:

  \begin{itemize}
  \tightlist
  \item
    = 0 use input values for alphaj, rli, dnbeta (but see gtscale
    option);
  \item
    = 1 make these consistent with input q, q0 values (recommendation:
    use icurr=4 with this option)
  \end{itemize}
\item
  iradloss /1/ : switch for radiation loss term usage in power balance
  (see User Guide):

  \begin{itemize}
  \tightlist
  \item
    = 0 total power lost is scaling power plus radiation (needed for
    ipedestal=2,3)
  \item
    = 1 total power lost is scaling power plus core radiation only
  \item
    = 2 total power lost is scaling power only, with no additional
    allowance for radiation. This is not recommended for power plant
    models.
  \end{itemize}
\item
  isc /34 (=IPB98(y,2))/ : switch for energy confinement time scaling
  law (see description in tauscl)
\item
  tauscl(ipnlaws) : labels describing energy confinement scaling laws:

  \begin{itemize}
  \tightlist
  \item
    ( 1) Neo-Alcator (ohmic)
  \item
    ( 2) Mirnov (H-mode)
  \item
    ( 3) Merezkhin-Muhkovatov (L-mode)
  \item
    ( 4) Shimomura (H-mode)
  \item
    ( 5) Kaye-Goldston (L-mode)
  \item
    ( 6) ITER 89-P (L-mode)
  \item
    ( 7) ITER 89-O (L-mode)
  \item
    ( 8) Rebut-Lallia (L-mode)
  \item
    ( 9) Goldston (L-mode)
  \item
    (10) T10 (L-mode)
  \item
    (11) JAERI-88 (L-mode)
  \item
    (12) Kaye-Big Complex (L-mode)
  \item
    (13) ITER H90-P (H-mode)
  \item
    (14) ITER Mix (L-mode)
  \item
    (15) Riedel (L-mode)
  \item
    (16) Christiansen (L-mode)
  \item
    (17) Lackner-Gottardi (L-mode)
  \item
    (18) Neo-Kaye (L-mode)
  \item
    (19) Riedel (H-mode)
  \item
    (20) ITER H90-P amended (H-mode)
  \item
    (21) LHD (stellarator)
  \item
    (22) Gyro-reduced Bohm (stellarator)
  \item
    (23) Lackner-Gottardi (stellarator)
  \item
    (24) ITER-93H (H-mode)
  \item
    (25) OBSOLETE
  \item
    (26) ITER H-97P ELM-free (H-mode)
  \item
    (27) ITER H-97P ELMy (H-mode)
  \item
    (28) ITER-96P (=ITER-97L) (L-mode)
  \item
    (29) Valovic modified ELMy (H-mode)
  \item
    (30) Kaye PPPL April 98 (L-mode)
  \item
    (31) ITERH-PB98P(y) (H-mode)
  \item
    (32) IPB98(y) (H-mode)
  \item
    (33) IPB98(y,1) (H-mode)
  \item
    (34) IPB98(y,2) (H-mode)
  \item
    (35) IPB98(y,3) (H-mode)
  \item
    (36) IPB98(y,4) (H-mode)
  \item
    (37) ISS95 (stellarator)
  \item
    (38) ISS04 (stellarator)
  \item
    (39) DS03 (H-mode)
  \item
    (40) Murari et al non-power law (H-mode)
  \item
    (41) Petty 2008 (H-mode)
  \item
    (42) Lang et al. 2012 (H-mode)
  \item
    (43) Hubbard 2017 (I-mode) - nominal
  \item
    (44) Hubbard 2017 (I-mode) - lower bound
  \item
    (45) Hubbard 2017 (I-mode) - upper bound
  \item
    (46) NSTX (H-mode; Spherical tokamak)
  \end{itemize}
\item
  iscrp /1/ : switch for plasma-first wall clearances:

  \begin{itemize}
  \tightlist
  \item
    = 0 use 10\% of rminor;
  \item
    = 1 use input (scrapli and scraplo)
  \end{itemize}
\item
  ishape /0/ : switch for plasma cross-sectional shape calculation:

  \begin{itemize}
  \tightlist
  \item
    = 0 use input kappa, triang to calculate 95\% values;
  \item
    = 1 scale qlim, kappa, triang with aspect ratio (ST);
  \item
    = 2 set kappa to the natural elongation value (Zohm ITER scaling),
    triang input;
  \item
    = 3 set kappa to the natural elongation value (Zohm ITER scaling),
    triang95 input;
  \item
    = 4 use input kappa95, triang95 to calculate separatrix values
  \end{itemize}
\item
  itart /0/ : switch for spherical tokamak (ST) models:

  \begin{itemize}
  \tightlist
  \item
    = 0 use conventional aspect ratio models;
  \item
    = 1 use spherical tokamak models
  \end{itemize}
\item
  itartpf /0/ : switch for Spherical Tokamak PF models:

  \begin{itemize}
  \tightlist
  \item
    = 0 use Peng and Strickler (1986) model;
  \item
    = 1 use conventional aspect ratio model
  \end{itemize}
\item
  iwalld /1/ : switch for neutron wall load calculation:

  \begin{itemize}
  \tightlist
  \item
    = 1 use scaled plasma surface area;
  \item
    = 2 use first wall area directly
  \end{itemize}
\item
  kappa /1.792/ : plasma separatrix elongation (calculated if ishape
  \textgreater{} 0)
\item
  kappa95 /1.6/ : plasma elongation at 95\% surface (calculated if
  ishape \textless{} 4)
\item
  kappaa : plasma elongation calculated as xarea/(pi.a2)
\item
  ne0 : central electron density (/m3)
\item
  ni0 : central ion density (/m3)
\item
  p0 : central total plasma pressure (Pa)
\item
  palppv : alpha power per volume (MW/m3)
\item
  palpepv : alpha power per volume to electrons (MW/m3)
\item
  palpfwmw : alpha power escaping plasma and reaching first wall (MW)
\item
  palpipv : alpha power per volume to ions (MW/m3)
\item
  palpmw : alpha power (MW)
\item
  palpnb : alpha power from hot neutral beam ions (MW)
\item
  pbrempv : bremsstrahlung power per volume (MW/m3) (calculated only if
  imprad\_model=1)
\item
  pchargemw : non-alpha charged particle fusion power (MW)
\item
  pchargepv : non-alpha charged particle fusion power per volume (MW/m3)
\item
  pcoef : profile factor (= n-weighted T / average T)
\item
  pcoreradmw : total core radiation power (MW)
\item
  pcoreradpv : total core radiation power per volume (MW/m3)
\item
  pdd : deuterium-deuterium fusion power (MW)
\item
  pdhe3 : deuterium-helium3 fusion power (MW)
\item
  pdivt : power to conducted to the divertor region (MW)
\item
  pdt : deuterium-tritium fusion power (MW)
\item
  pedgeradmw : edge radiation power (MW)
\item
  pedgeradpv : edge radiation power per volume (MW/m3)
\item
  pfuscmw : charged particle fusion power (MW)
\item
  phiint : internal plasma V-s
\item
  photon\_wall : Nominal mean radiation load on inside surface of
  reactor (MW/m2)
\item
  piepv : ion/electron equilibration power per volume (MW/m3)
\item
  plascur : plasma current (A)
\item
  plinepv : line radiation power per volume (MW/m3) (calculated only if
  imprad\_model=1)
\item
  pneutmw : neutron fusion power (MW)
\item
  pneutpv : neutron fusion power per volume (MW/m3)
\item
  pohmmw : ohmic heating power (MW)
\item
  pohmpv : ohmic heating power per volume (MW/m3)
\item
  powerht : heating power (= transport loss power) (MW) used in
  confinement time calculation
\item
  powfmw : fusion power (MW)
\item
  pperim : plasma poloidal perimeter (m)
\item
  pradmw : total radiation power (MW)
\item
  pradpv : total radiation power per volume (MW/m3)
\item
  protonrate : proton production rate (particles/m3/sec)
\item
  psolradmw : SOL radiation power (MW) (stellarator only)
\item
  psyncpv : synchrotron radiation power per volume (MW/m3)
\item
  ilhthresh /6/ : switch for L-H mode power threshold scaling to use
  (see pthrmw for list)
\item
  plhthresh : L-H mode power threshold (MW) (chosen via ilhthresh, and
  enforced if constraint equation 15 is on)
\item
  pthrmw(18) : L-H power threshold for various scalings (MW):

  \begin{enumerate}
  \tightlist
  \item
    ITER 1996 scaling: nominal
  \item
    ITER 1996 scaling: upper bound
  \item
    ITER 1996 scaling: lower bound
  \item
    ITER 1997 scaling: excluding elongation
  \item
    ITER 1997 scaling: including elongation
  \item
    Martin 2008 scaling: nominal
  \item
    Martin 2008 scaling: 95\% upper bound
  \item
    Martin 2008 scaling: 95\% lower bound
  \item
    Snipes 2000 scaling: nominal
  \item
    Snipes 2000 scaling: upper bound
  \item
    Snipes 2000 scaling: lower bound
  \item
    Snipes 2000 scaling (closed divertor): nominal
  \item
    Snipes 2000 scaling (closed divertor): upper bound
  \item
    Snipes 2000 scaling (closed divertor): lower bound
  \item
    Hubbard et al. 2012 L-I threshold scaling: nominal
  \item
    Hubbard et al. 2012 L-I threshold scaling: lower bound
  \item
    Hubbard et al. 2012 L-I threshold scaling: upper bound
  \item
    Hubbard et al. 2017 L-I threshold scaling
  \end{enumerate}
\item
  ptremw : electron transport power (MW)
\item
  ptrepv : electron transport power per volume (MW/m3)
\item
  ptrimw : ion transport power (MW)
\item
  pscalingmw : Total transport power from scaling law (MW)
\item
  ptripv : ion transport power per volume (MW/m3)
\item
  q /3.0/ : safety factor `near' plasma edge (iteration variable 18):
  equal to q95 (unless icurr = 2 (ST current scaling), in which case q =
  mean edge safety factor qbar)
\item
  q0 /1.0/ : safety factor on axis
\item
  q95 : safety factor at 95\% surface
\item
  qfuel : plasma fuelling rate (nucleus-pairs/s)
\item
  tauratio /1.0/ : ratio of He and pellet particle confinement times
\item
  qlim : lower limit for edge safety factor
\item
  qstar : cylindrical safety factor
\item
  rad\_fraction : Radiation fraction = total radiation / total power
  deposited in plasma
\item
  ralpne /0.1/ : thermal alpha density / electron density (iteration
  variable 109) (calculated if ipedestal=3)
\item
  protium /0.0/ : Seeded protium density / electron density.
\item
  rli /0.9/ : plasma normalised internal inductance; calculated from
  alphaj if iprofile=1
\item
  rlp : plasma inductance (H)
\item
  rmajor /8.14/ : plasma major radius (m) (iteration variable 3)
\item
  rminor : plasma minor radius (m)
\item
  rnbeam /0.005/ : hot beam density / n\_e (iteration variable 7)
\item
  rncne : n\_carbon / n\_e
\item
  rndfuel : fuel burnup rate (reactions/second)
\item
  rnfene : n\_highZ / n\_e
\item
  rnone : n\_oxygen / n\_e
\item
  rpfac : neo-classical correction factor to rplas
\item
  rplas : plasma resistance (ohm)
\item
  res\_time : plasma current resistive diffusion time (s)
\item
  sarea : plasma surface area
\item
  sareao : outboard plasma surface area
\item
  sf : shape factor = plasma poloidal perimeter / (2.pi.rminor)
\item
  snull /1/ : switch for single null / double null plasma:

  \begin{itemize}
  \tightlist
  \item
    = 0 for double null;
  \item
    = 1 for single null (diverted side down)
  \end{itemize}
\item
  ssync /0.6/ : synchrotron wall reflectivity factor
\item
  tauee : electron energy confinement time (sec)
\item
  taueff : global thermal energy confinement time (sec)
\item
  tauei : ion energy confinement time (sec)
\item
  taup : alpha particle confinement time (sec)
\item
  te /12.9/ : volume averaged electron temperature (keV) (iteration
  variable 4) (calculated if ipedestal = 3)
\item
  te0 : central electron temperature (keV)
\item
  ten : density weighted average electron temperature (keV)
\item
  ti /12.9/ : volume averaged ion temperature (keV); N.B. calculated
  from te if tratio \textgreater{} 0.0
\item
  ti0 : central ion temperature (keV)
\item
  tin : density weighted average ion temperature (keV)
\item
  tratio /1.0/ : ion temperature / electron temperature; used to
  calculate ti if tratio \textgreater{} 0.0
\item
  triang /0.36/ : plasma separatrix triangularity (calculated if
  ishape=1, 3 or 4)
\item
  triang95 /0.24/ : plasma triangularity at 95\% surface (calculated if
  ishape \textless{} 3)
\item
  vol : plasma volume (m3)
\item
  vsbrn : V-s needed during flat-top (heat + burn times) (Wb)
\item
  vshift : plasma/device midplane vertical shift - single null
\item
  vsind : internal and external plasma inductance V-s (Wb)
\item
  vsres : resistive losses in startup V-s (Wb)
\item
  vsstt : total V-s needed (Wb)
\item
  wallmw : average neutron wall load (MW/m2)
\item
  wtgpd : mass of fuel used per day (g)
\item
  xarea : plasma cross-sectional area (m2)
\item
  zeff : plasma effective charge
\item
  zeffai : mass weighted plasma effective charge
\item
  zfear /0/ : high-Z impurity switch; 0=iron, 1=argon (if
  imprad\_model=1, only used in neutral beam stopping calc.)
\end{itemize}

\subsubsection{\texorpdfstring{\href{plasmod_variables.html}{plasmod\_variables}}{plasmod\_variables}}\label{plasmod_variables}

\begin{itemize}
\tightlist
\item
  plasmod\_tol /1.0d-10/ : tolerance to be reached at each time step
  (\%)
\item
  plasmod\_dtmin /0.05d0/ : min time step
\item
  plasmod\_dtmax /0.1d0/ : max time step
\item
  plasmod\_dt /0.01d0/ : time step
\item
  plasmod\_dtinc /2.0d0/ : decrease of dt
\item
  plasmod\_ainc /1.1d0/ : increase of dt
\item
  plasmod\_test /100000.0d0/ : max number of iterations
\item
  plasmod\_tolmin /10.1d0/ : multiplier of etolm which can not be
  exceeded
\item
  plasmod\_eopt /0.15d0/ : exponent of jipperdo
\item
  plasmod\_dtmaxmin /0.15d0/ : exponent of jipperdo2
\item
  plasmod\_dtmaxmax /0.0d0/ : stabilizing coefficient
\item
  plasmod\_capa /0.1d0/ : first radial grid point
\item
  plasmod\_maxa /0.0d0/ : diagz 0 or 1
\item
  plasmod\_dgy /1.0d-5/ : Newton differential
\item
  plasmod\_iprocess /1/ : 0 - use PLASMOD functions, 1 - use PROCESS
  functions
\item
  plasmod\_i\_modeltype /1/ : switch for the transport model

  \begin{itemize}
  \tightlist
  \item
    1 - Simple gyrobohm scaling with imposed H factor \textgreater{} 1.
    Other values give H factor as output
  \item
    111 - roughly calibrated to give H=1 for DEMO, but not fixed H
  \end{itemize}
\item
  plasmod\_i\_equiltype /1/ : 1 - EMEQ, solve with sawteeth and inputted
  q95. 2 - EMEQ, solve with sawteeth and inputted Ip (not recommended!).
\item
  plasmod\_isawt /1/ : 0 - no sawteeth, 1 - solve with sawteeth.
\item
  plasmod\_nx /41/ : number of interpolated grid points
\item
  plasmod\_nxt /7/ : number of solved grid points
\item
  plasmod\_nchannels /3/ : leave this at 3
\item
  plasmod\_i\_impmodel /1/ : impurity model: 0 - fixed concentration, 1
  - fixed concentration at pedestal top, then fixed density.
\item
  plasmod\_globtau(5) /5.0d0, 5.0d0, 7.0d0, 5.0d0, 1.0d0/ :
  tauparticle/tauE for D, T, He, Xe, Ar (NOT used for Xe!)
\item
  plasmod\_psepplh\_sup /12000.0d0/ : Psep/PLH if above this, use Xe
\item
  plasmod\_qdivt /0.0d0/ : divertor heat flux in MW/m\^{}2, if 0, dont
  use SOL model
\item
  plasmod\_imptype(3) /14, 13, 9/ : Impurities: element 1 - intrinsic
  impurity, element 2 - Psep control, elem
\item
  plasmod\_qnbi\_psepfac /50.0d0/ : dqnbi/d(1-Psep/PLH)
\item
  plasmod\_cxe\_psepfac /1.0d-4/ : dcxe/d(1-Psep/PLH)
\item
  plasmod\_car\_qdivt /1.0d-4/ : dcar/d(qdivt)
\item
  plasmod\_maxpauxor /20.0d0/ : max allowed auxiliary power / R
\item
  plasmod\_x\_heat(2) /0.0d0/ : element 1 - nbi, element 2 - ech
\item
  plasmod\_x\_cd(2) /0.0d0/ : element 1 - nbi, element 2 - ech
\item
  plasmod\_x\_fus(2) /0.0d0/ : element 1 - nbi, element 2 - ech
\item
  plasmod\_x\_control(2) /0.0d0/ : element 1 - nbi, element 2 - ech
\item
  plasmod\_dx\_heat(2) /0.2d0, 0.03d0/ : element 1 - nbi, element 2 -
  ech
\item
  plasmod\_dx\_cd(2) /0.2d0, 0.03/ : element 1 - nbi, element 2 - ech
\item
  plasmod\_dx\_fus(2) /0.2d0, 0.03d0/ : element 1 - nbi, element 2 - ech
\item
  plasmod\_dx\_control(2) /0.2d0, 0.03d0/ : element 1 - nbi, element 2 -
  ech
\item
  plasmod\_contrpovs /0.0d0/ :: control power in Paux/lateral\_area
  (MW/m2)
\item
  plasmod\_contrpovr /0.0d0/ :: control power in Paux/R (MW/m)
\item
  plasmod\_nbi\_energy /1000.0d0/ :: In keV
\item
  plasmod\_v\_loop /-1.0d-6/ :: target loop voltage. If lower than -1.e5
  do not use
\item
  plasmod\_pfus /0.0d0/ :: if 0. not used (otherwise controlled with
  Pauxheat)
\item
  plasmod\_eccdeff /0.3d0/ :: current drive multiplier: CD =
  eccdeff*PCD*TE/NE (not in use yet)
\item
  plasmod\_fcdp /-1.0d0/ :: (P\_CD - Pheat)/(Pmax-Pheat),i.e. ratio of
  CD power over available power (iteration
\item
  plasmod\_fradc /-1.0d0/ :: Pline\_Xe / (Palpha + Paux - PlineAr -
  Psync - Pbrad) (iteration variable 148)
\item
  plasmod\_pech /0.0d0/ :: ech power (not in use yet)
\item
  plasmod\_gamcdothers /1.0d0/ :: efficiency multiplier for non-CD
  heating. If 0.0 pheat treated as if it had
\item
  plasmod\_chisawpos /-1.0d0/ :: position where artificial sawtooth
  diffusivity is added, -1 - uses q=1 positi
\item
  plasmod\_chisaw /0.0d0/ :: artificial diffusivity in m\^{}2/s
\item
  plasmod\_sawpertau /1.0d-6/ :: ratio between sawtooth period and
  confinement time
\item
  plasmod\_spellet /0.0d0/ :: pellet mass in units of D in 10\^{}19
\item
  plasmod\_fpellet /0.5d0/ :: pellet frequency in Hz
\item
  plasmod\_pedscal /1.0d0/ :: multiplication factor of the pedestal
  scaling in PLASMOD can be used to scan the pedestal height.
\item
  geom :: Derived type containing all geometry information for PLASMOD
\item
  comp :: Derived type containing all composition information for
  PLASMOD
\item
  ped :: Derived type containing all pedestal information for PLASMOD
\item
  inp0 :: Derived type containing miscellaneous input information for
  PLASMOD
\item
  radp :: Derived type containing all radial profile information for
  PLASMOD
\item
  mhd :: Derived type containing all mhd information for PLASMOD
\item
  loss :: Derived type containing all power loss information for PLASMOD
\item
  num :: Derived type containing all numerics information for PLASMOD
\item
  i\_flag :: Error flag for PLASMOD
\end{itemize}

\subsubsection{\texorpdfstring{\href{current_drive_variables.html}{current\_drive\_variables}}{current\_drive\_variables}}\label{current_drive_variables}

\begin{itemize}
\tightlist
\item
  beamwd /0.58/ : width of neutral beam duct where it passes between the
  TF coils (m) (T Inoue et al, Design of neutral beam system for
  ITER-FEAT,
  \href{http://dx.doi.org/10.1016/S0920-3796(01)00339-8}{Fusion
  Engineering and Design, Volumes 56-57, October 2001, Pages 517-521})
\item
  bigq : Fusion gain; P\_fusion / (P\_injection + P\_ohmic)
\item
  bootipf : bootstrap current fraction (enforced; see ibss)
\item
  bscfmax /0.9/ : maximum fraction of plasma current from bootstrap; if
  bscfmax \textless{} 0, bootstrap fraction = abs(bscfmax)
\item
  bscf\_iter89 : bootstrap current fraction, ITER 1989 model
\item
  bscf\_nevins : bootstrap current fraction, Nevins et al model
\item
  bscf\_sauter : bootstrap current fraction, Sauter et al model
\item
  bscf\_wilson : bootstrap current fraction, Wilson et al model
\item
  cboot /1.0/ : bootstrap current fraction multiplier (ibss=1)
\item
  cnbeam : neutral beam current (A)
\item
  echpwr : ECH power (MW)
\item
  echwpow : ECH wall plug power (MW)
\item
  effcd : current drive efficiency (A/W)
\item
  enbeam /1.0e3/ : neutral beam energy (keV) (iteration variable 19)
\item
  etacd : auxiliary power wall plug to injector efficiency
\item
  etaech /0.3/ : ECH wall plug to injector efficiency
\item
  etalh /0.3/ : lower hybrid wall plug to injector efficiency
\item
  etanbi /0.3/ : neutral beam wall plug to injector efficiency
\item
  fpion : fraction of beam energy to ions
\item
  pnbitot : neutral beam power entering vacuum vessel
\item
  nbshinemw : neutral beam shine-through power
\item
  feffcd /1.0/ : current drive efficiency fudge factor (iteration
  variable 47)
\item
  forbitloss /0.0/ : fraction of neutral beam power lost after
  ionisation but before thermalisation (orbit loss fraction)
\item
  frbeam /1.05/ : R\_tangential / R\_major for neutral beam injection
\item
  ftritbm /1.0e-6/ : fraction of beam that is tritium
\item
  gamcd : normalised current drive efficiency (1.0e20 A/(W m\^{}2))
\item
  gamma\_ecrh /0.35/ : user input ECRH gamma (1.0e20 A/(W m\^{}2))
\item
  rho\_ecrh /0.1/ : normalised minor radius at which electron cyclotron
  current drive is maximum
\item
  iefrf /5/ : switch for current drive efficiency model:

  \begin{enumerate}
  \tightlist
  \item
    Fenstermacher Lower Hybrid
  \item
    Ion Cyclotron current drive
  \item
    Fenstermacher ECH
  \item
    Ehst Lower Hybrid
  \item
    ITER Neutral Beam
  \item
    new Culham Lower Hybrid model
  \item
    new Culham ECCD model
  \item
    new Culham Neutral Beam model
  \item
    Empty (Oscillating field CD removed)
  \item
    ECRH user input gamma
  \item
    ECRH ``HARE'' model (E. Poli, Physics of Plasmas 2019)
  \end{enumerate}
\item
  irfcd /1/ : switch for current drive calculation:

  \begin{itemize}
  \tightlist
  \item
    = 0 turned off;
  \item
    = 1 turned on
  \end{itemize}
\item
  nbshinef : neutral beam shine-through fraction
\item
  nbshield /0.5/ : neutral beam duct shielding thickness (m)
\item
  pheat /0.0/ : heating power not used for current drive (MW) (iteration
  variable 11)
\item
  pinjalw /150.0/ : Maximum allowable value for injected power (MW)
  (constraint equation 30)
\item
  pinjemw : auxiliary injected power to electrons (MW)
\item
  pinjimw : auxiliary injected power to ions (MW)
\item
  pinjmw : total auxiliary injected power (MW)
\item
  plhybd : lower hybrid injection power (MW)
\item
  pnbeam : neutral beam injection power (MW)
\item
  porbitlossmw : neutral beam power lost after ionisation but before
  thermalisation (orbit loss power) (MW)
\item
  pwplh : lower hybrid wall plug power (MW)
\item
  pwpnb : neutral beam wall plug power (MW)
\item
  rtanbeam : neutral beam centreline tangency radius (m)
\item
  rtanmax : maximum tangency radius for centreline of beam (m)
\item
  taubeam : neutral beam e-decay lengths to plasma centre
\item
  tbeamin /3.0/ : permitted neutral beam e-decay lengths to plasma
  centre
\end{itemize}

\subsubsection{\texorpdfstring{\href{divertor_kallenbach_variables.html}{divertor\_kallenbach\_variables}}{divertor\_kallenbach\_variables}}\label{divertor_kallenbach_variables}

\begin{itemize}
\item
  kallenbach\_switch /0/ : Switch to turn on the 1D Kallenbach divertor
  model (1=on, 0=off)
\item
  kallenbach\_tests /0/ : Switch to run tests of 1D Kallenbach divertor
  model (1=on, 0=off)
\item
  kallenbach\_test\_option /0/ : Switch to choose kallenbach test
  option:

  \begin{itemize}
  \tightlist
  \item
    = 0 Test case with user inputs;
  \item
    = 1 Test case for Kallenbach paper;
  \end{itemize}
\item
  kallenbach\_scan\_switch /0/ : Switch to run scan of 1D Kallenbach
  divertor model (1=on, 0=off)
\item
  kallenbach\_scan\_var /0/ : Switch for parameter to scan for
  kallenbach scan test:

  \begin{itemize}
  \tightlist
  \item
    = 0 ttarget
  \item
    = 1 qtargettotal
  \item
    = 2 targetangle
  \item
    = 3 lambda\_q\_omp
  \item
    = 4 netau\_sol
  \item
    kallenbach\_scan\_start /2.0/ : Start value for kallenbach scan
    parameter
  \item
    kallenbach\_scan\_end /10.0/ : End value for kallenbach scan
    parameter
  \item
    kallenbach\_scan\_num /1/ : Number of scans for kallenbach scan test
  \item
    target\_spread /0.003/ : Increase in SOL power fall-off length due
    to spreading, mapped to OMP {[}m{]}
  \item
    lambda\_q\_omp /0.002/ : SOL power fall-off length at the outer
    midplane, perpendicular to field {[}m{]}
  \item
    lcon\_factor /1.0/ : Correction factor for connection length from
    OMP to divertor = connection length/(pi*q*rmajor)
  \item
    netau\_sol /0.5/ : Parameter describing the departure from local
    ionisation equilibrium in the SOL. {[}ms.1e20
  \item
    targetangle /30.0/ : Angle between field-line and divertor target
    (degrees)
  \item
    ttarget /2.3/ : Plasma temperature adjacent to divertor sheath
    {[}eV{]} (iteration variable 120)
  \item
    qtargettotal /5.0e6/ : Power density on target including surface
    recombination {[}W/m2{]} (iteration variable 124)
  \item
    impurity\_enrichment(14) /5.0/ : Ratio of each impurity
    concentration in SOL to confined plasma+ the enrichment for Argon is
    also propagated for PLASMOD (ipedestal=3)
  \item
    psep\_kallenbach : Power conducted through the separatrix, as
    calculated by the divertor model {[}W{]} Not equal to pdivt unless
    constraint is imposed.
  \item
    teomp : separatrix temperature calculated by the Kallenbach divertor
    model {[}eV{]}
  \item
    neomp : Mean SOL density at OMP calculated by the Kallenbach
    divertor model {[}m-3{]}
  \item
    neratio /0.75/ : Ratio of mean SOL density at OMP to separatrix
    density at OMP (iteration variable 121)
  \item
    pressure0 : Total plasma pressure near target (thermal+dynamic)
    {[}Pa{]}
  \item
    fractionwidesol /0.1/ : Distance from target at which SOL gets
    broader as a fraction of connection length
  \item
    fmom : momentum factor {[}-{]}
  \item
    totalpowerlost : Total power lost due to radiation, ionisation and
    recombination {[}W{]}
  \item
    impuritypowerlost : Power lost due to impurity radiation {[}W{]}
  \item
    hydrogenicpowerlost : Power lost due to hydrogenic radiation {[}W{]}
  \item
    exchangepowerlost : Power lost due to charge exchange {[}W{]}
  \item
    ionisationpowerlost : Power lost due to electron impact ionisation
    {[}W{]}
  \item
    abserr\_sol : Absolute contribution to the error tolerance in the
    Kallenbach divertor model
  \item
    relerr\_sol : Relative contribution to the error tolerance in the
    Kallenbach divertor model
  \item
    mach0 : Mach number at target (must be just less than 1)
  \end{itemize}

  \subsubsection{\texorpdfstring{\href{divertor_variables.html}{divertor\_variables}}{divertor\_variables}}\label{divertor_variables}

  \begin{itemize}
  \tightlist
  \item
    adas : area divertor / area main plasma (along separatrix)
  \item
    anginc /0.262/ : angle of incidence of field line on plate (rad)
  \item
    betai /1.0/ : poloidal plane angle between divertor plate and leg,
    inboard (rad)
  \item
    betao /1.0/ : poloidal plane angle between divertor plate and leg,
    outboard (rad)
  \item
    bpsout /0.6/ : reference B\_p at outboard divertor strike point (T)
  \item
    c1div /0.45/ : fitting coefficient to adjust ptpdiv, ppdiv
  \item
    c2div /-7.0/ : fitting coefficient to adjust ptpdiv, ppdiv
  \item
    c3div /0.54/ : fitting coefficient to adjust ptpdiv, ppdiv
  \item
    c4div /-3.6/ : fitting coefficient to adjust ptpdiv, ppdiv
  \item
    c5div /0.7/ : fitting coefficient to adjust ptpdiv, ppdiv
  \item
    c6div /0.0/ : fitting coefficient to adjust ptpdiv, ppdiv
  \item
    delld /1.0/ : coeff for power distribution along main plasma
  \item
    dendiv : plasma density at divertor (10**20 /m3)
  \item
    densin : density at plate (on separatrix) (10**20 /m3)
  \item
    divclfr /0.3/ : divertor coolant fraction
  \item
    divdens /1.0e4/ : divertor structure density (kg/m3)
  \item
    divdum /0/ : switch for divertor Zeff model: 0=calc, 1=input
  \item
    divfix /0.2/ : divertor structure vertical thickness (m)
  \item
    divmas : divertor plate mass (kg)
  \item
    divplt /0.035/ : divertor plate thickness (m) (from Spears, Sept
    1990)
  \item
    divsur : divertor surface area (m2)
  \item
    fdfs /10.0/ : radial gradient ratio
  \item
    fdiva /1.11/ : divertor area fudge factor (for ITER, Sept 1990)
  \item
    fgamp /1.0/ : sheath potential factor (not used)
  \item
    fhout : fraction of power to outboard divertor (for single null)
  \item
    fififi /0.004/ : coefficient for gamdiv
  \item
    frrp /0.4/ : fraction of radiated power to plate
  \item
    hldiv : divertor heat load (MW/m2)
  \item
    hldivlim /5.0/ : heat load limit (MW/m2)
  \item
    ksic /0.8/ : power fraction for outboard double-null scrape-off
    plasma
  \item
    lamp : power flow width (m)
  \item
    minstang : minimum strike angle for heat flux calculation
  \item
    omegan /1.0/ : pressure ratio (nT)\_plasma / (nT)\_scrape-off
  \item
    omlarg : power spillage to private flux factor
  \item
    ppdivr : peak heat load at plate (with radiation) (MW/m2)
  \item
    prn1 /0.285/ : n-scrape-off / n-average plasma; (input for
    ipedestal=0, = nesep/dene if ipedestal\textgreater{}=1)
  \item
    ptpdiv : peak temperature at the plate (eV)
  \item
    rconl : connection length ratio, outboard side
  \item
    rlclolcn : ratio of collision length / connection length
  \item
    rlenmax /0.5/ : maximum value for length ratio (rlclolcn) (eqn.22)
  \item
    rsrd : effective separatrix/divertor radius ratio
  \item
    tconl : main plasma connection length (m)
  \item
    tdiv /2.0/ : temperature at divertor (eV) (input for stellarator
    only, calculated for tokamaks)
  \item
    tsep : temperature at the separatrix (eV)
  \item
    xparain /2.1e3/ : parallel heat transport coefficient (m2/s)
  \item
    xpertin /2.0/ : perpendicular heat transport coefficient (m2/s)
  \item
    zeffdiv /1.0/ : Zeff in the divertor region (if divdum /= 0)
  \end{itemize}

  \subsubsection{\texorpdfstring{\href{fwbs_variables.html}{fwbs\_variables}}{fwbs\_variables}}\label{fwbs_variables}

  \begin{itemize}
  \item
    bktlife : blanket lifetime (years)
  \item
    coolmass : mass of water coolant (in shield, blanket, first wall,
    divertor) (kg)
  \item
    vvmass : vacuum vessel mass (kg)
  \item
    denstl /7800.0/ : density of steel (kg/m3)
  \item
    denw /19250.0/ : density of tungsten (kg/m3)
  \item
    dewmkg : total mass of vacuum vessel + cryostat (kg)
  \item
    emult /1.269/ : energy multiplication in blanket and shield
  \item
    emultmw : power due to energy multiplication in blanket and shield
    {[}MW{]}
  \item
    fblss /0.09705/ : KIT blanket model: steel fraction of breeding zone
  \item
    fdiv /0.115/ : area fraction taken up by divertor
  \item
    fhcd /0.0/ : area fraction covered by heating/current drive
    apparatus plus diagnostics
  \item
    fhole /0.0/ : area fraction taken up by other holes (not used)
  \item
    fwbsshape /2/ : first wall, blanket, shield and vacuum vessel shape:

    \begin{itemize}
    \tightlist
    \item
      = 1 D-shaped (cylinder inboard + ellipse outboard);
    \item
      = 2 defined by two ellipses
    \end{itemize}
  \item
    fwlife : first wall full-power lifetime (y)
  \item
    fwmass : first wall mass (kg)
  \item
    fw\_armour\_mass : first wall armour mass (kg)
  \item
    fw\_armour\_thickness /0.005/ : first wall armour thickness (m)
  \item
    fw\_armour\_vol : first wall armour volume (m3)
  \item
    iblanket /1/ : switch for blanket model:

    \begin{itemize}
    \tightlist
    \item
      = 1 CCFE HCPB model;
    \item
      = 2 KIT HCPB model;
    \item
      = 3 CCFE HCPB model with Tritium Breeding Ratio calculation;
    \item
      = 4 KIT HCLL model
    \end{itemize}
  \item
    iblnkith /1/ : switch for inboard blanket:

    \begin{itemize}
    \tightlist
    \item
      = 0 No inboard blanket (blnkith=0.0);
    \item
      = 1 Inboard blanket present
    \end{itemize}
  \item
    inuclear /0/ : switch for nuclear heating in the coils:

    \begin{itemize}
    \tightlist
    \item
      = 0 Frances Fox model (default);
    \item
      = 1 Fixed by user (qnuc)
    \end{itemize}
  \item
    qnuc /0.0/ : nuclear heating in the coils (W) (inuclear=1)
  \item
    li6enrich /30.0/ : lithium-6 enrichment of breeding material (\%)
  \item
    pnucblkt : nuclear heating in the blanket (MW)
  \item
    pnuccp : nuclear heating in the ST centrepost (MW)
  \item
    pnucdiv : nuclear heating in the divertor (MW)
  \item
    pnucfw : nuclear heating in the first wall (MW)
  \item
    pnuchcd : nuclear heating in the HCD apparatus and diagnostics (MW)
  \item
    pnucloss : nuclear heating lost via holes (MW)
  \item
    pnucloss : nuclear heating to vacuum vessel and beyond(MW)
  \item
    pnucshld : nuclear heating in the shield (MW)
  \item
    whtblkt : mass of blanket (kg)
  \item
    whtblss : mass of blanket - steel part (kg)
  \item
    armour\_fw\_bl\_mass : Total mass of armour, first wall and blanket
    (kg)
  \item
    \textbf{The following are used only in the CCFE HCPB blanket model
    (iblanket=1):}
  \item
    breeder\_f /0.5/ : Volume ratio: Li4SiO4/(Be12Ti+Li4SiO4) (iteration
    variable 108)
  \item
    breeder\_multiplier /0.75/ : combined breeder/multipler fraction of
    blanket by volume
  \item
    vfcblkt /0.05295/ : He coolant fraction of blanket by volume
    (iblanket = 1 or 3 (CCFE HCPB))
  \item
    vfpblkt /0.1/ : He purge gas fraction of blanket by volume (iblanket
    = 1 or 3 (CCFE HCPB))
  \item
    whtblli4sio4 : mass of lithium orthosilicate in blanket (kg)
    (iblanket = 1 or 3 (CCFE HCPB))
  \item
    whtbltibe12 : mass of titanium beryllide in blanket (kg) (iblanket =
    1 or 3 (CCFE HCPB))
  \item
    \textbf{The following are used in the KIT HCPB blanket model
    (iblanket=2):}
  \item
    breedmat /1/ : breeder material switch (iblanket=2 (KIT HCPB)):

    \begin{itemize}
    \tightlist
    \item
      = 1 Lithium orthosilicate;
    \item
      = 2 Lithium methatitanate;
    \item
      = 3 Lithium zirconate
    \end{itemize}
  \item
    densbreed : density of breeder material (kg/m3) (iblanket=2 (KIT
    HCPB))
  \item
    fblbe /0.6/ : beryllium fraction of blanket by volume (if
    (iblanket=2 (KIT HCPB)), Be fraction of breeding zone)
  \item
    fblbreed /0.154/ : breeder fraction of blanket breeding zone by
    volume (iblanket=2 (KIT HCPB))
  \item
    fblhebmi /0.40/ : helium fraction of inboard blanket box manifold by
    volume (iblanket=2 (KIT HCPB))
  \item
    fblhebmo /0.40/ : helium fraction of outboard blanket box manifold
    by volume (iblanket=2 (KIT HCPB))
  \item
    fblhebpi /0.6595/ : helium fraction of inboard blanket back plate by
    volume (iblanket=2 (KIT HCPB))
  \item
    fblhebpo /0.6713/ : helium fraction of outboard blanket back plate
    by volume (iblanket=2 (KIT HCPB))
  \item
    hcdportsize /1/ : size of heating/current drive ports (iblanket=2
    (KIT HCPB)):

    \begin{itemize}
    \tightlist
    \item
      = 1 `small'
    \item
      = 2 `large'
    \end{itemize}
  \item
    nflutf : peak fast neutron fluence on TF coil superconductor (n/m2)
    (iblanket=2 (KIT HCPB))
  \item
    npdiv /2/ : number of divertor ports (iblanket=2 (KIT HCPB))
  \item
    nphcdin /2/ : number of inboard ports for heating/current drive
    (iblanket=2 (KIT HCPB))
  \item
    nphcdout /2/ : number of outboard ports for heating/current drive
    (iblanket=2 (KIT HCPB))
  \item
    tbr : tritium breeding ratio (iblanket=2,3 (KIT HCPB/HCLL))
  \item
    tritprate : tritium production rate (g/day) (iblanket=2 (KIT HCPB))
  \item
    vvhemax : maximum helium concentration in vacuum vessel at end of
    plant life (appm) (iblanket=2 (KIT HCPB))
  \item
    wallpf /1.21/ : neutron wall load peaking factor (iblanket=2 (KIT
    HCPB))
  \item
    whtblbreed : mass of blanket - breeder part (kg) (iblanket=2 (KIT
    HCPB))
  \item
    whtblbe : mass of blanket - beryllium part (kg)
  \item
    \textbf{CCFE HCPB model with Tritium Breeding Ratio calculation
    (iblanket=3):}
  \item
    tbrmin /1.1/ : minimum tritium breeding ratio (constraint equation
    52) (If iblanket=1, tbrmin=minimum 5-year time-averaged tritium
    breeding ratio)
  \item
    iblanket\_thickness /2/ : Blanket thickness switch:

    \begin{itemize}
    \tightlist
    \item
      = 1 thin 0.53 m inboard, 0.91 m outboard
    \item
      = 2 medium 0.64 m inboard, 1.11 m outboard
    \item
      = 3 thick 0.75 m inboard, 1.30 m outboard
    \end{itemize}
  \item
    Do not set blnkith, blnkoth, fwith or fwoth when iblanket=3.
  \item
    primary\_pumping /2/ : Switch for pumping power for primary coolant
    (06/01/2016): (mechanical power only)

    \begin{itemize}
    \tightlist
    \item
      = 0 User sets pump power directly (htpmw\_blkt, htpmw\_fw,
      htpmw\_div, htpmw\_shld)
    \item
      = 1 User sets pump power as a fraction of thermal power
      (fpumpblkt, fpumpfw, fpumpdiv, fpumpshld)
    \item
      = 2 Mechanical pumping power is calculated
    \item
      = 3 Mechanical pumping power is calculated using specified
      pressure drop
    \end{itemize}
  \item
    (peak first wall temperature is only calculated if primary\_pumping
    = 2)
  \item
    secondary\_cycle /0/ : Switch for power conversion cycle:

    \begin{itemize}
    \tightlist
    \item
      = 0 Set efficiency for chosen blanket, from detailed models
      (divertor heat not used)
    \item
      = 1 Set efficiency for chosen blanket, from detailed models
      (divertor heat used)
    \item
      = 2 user input thermal-electric efficiency (etath)
    \item
      = 3 steam Rankine cycle
    \item
      = 4 supercritical CO2 cycle
    \end{itemize}
  \item
    coolwh : Blanket coolant (set via blkttype):

    \begin{itemize}
    \tightlist
    \item
      = 1 helium;
    \item
      = 2 pressurized water
    \end{itemize}
  \item
    afwi /0.008/ : inner radius of inboard first wall/blanket coolant
    channels OBSOLETE (m)
  \item
    afwo /0.008/ : inner radius of outboard first wall/blanket coolant
    channels OBSOLETE (m)
  \item
    fwcoolant /helium/ : first wall coolant (can be different from
    blanket coolant) `helium' or `water' (27/11/2015)
  \item
    fw\_wall /0.003/ : wall thickness of first wall coolant channels (m)
    (27/11/2015)
  \item
    afw /0.006/ : radius of first wall cooling channels (m) (27/11/15)
  \item
    pitch /0.020/ : pitch of first wall cooling channels (m) (27/11/15)
  \item
    fwinlet /573/ : inlet temperature of first wall coolant (K)
    (27/11/2015)
  \item
    fwoutlet /823/ : outlet temperature of first wall coolant (K)
    (27/11/2015)
  \item
    fwpressure /15.5e6/ : first wall coolant pressure (Pa)
    (secondary\_cycle\textgreater{}1)
  \item
    tpeak : peak first wall temperature (K) (27/11/2015)
  \item
    roughness /1e-6/ : first wall channel roughness epsilon (m)
    (27/11/2015)
  \item
    fw\_channel\_length /4.0/ : Length of a single first wall channel
    (all in parallel) (m) (27/11/2015) (iteration variable 114, useful
    for constraint equation 39)
  \item
    peaking\_factor /1.0/ : peaking factor for first wall heat loads
    (27/11/2015) (Applied separately to inboard and outboard loads.
    Applies to both neutron and surface loads. Only used to calculate
    peak temperature - not the coolant flow rate.)
  \item
    blpressure /15.5e6/ : blanket coolant pressure (Pa)
    (secondary\_cycle\textgreater{}1) (27/11/2015)
  \item
    inlet\_temp /573.0/ : inlet temperature of blanket coolant (K)
    (secondary\_cycle\textgreater{}1) (27/11/2015)
  \item
    outlet\_temp /823.0/ : outlet temperature of blanket coolant (K)
    (27/11/2015)

    \begin{itemize}
    \tightlist
    \item
      (secondary\_cycle\textgreater{}1);
    \item
      input if coolwh=1 (helium), calculated if coolwh=2 (water)
    \end{itemize}
  \item
    coolp /15.5e6/ : blanket coolant pressure (Pa) stellarator ONLY
    (27/11/2015)
  \item
    nblktmodpo /8/ : number of outboard blanket modules in poloidal
    direction (secondary\_cycle\textgreater{}1)
  \item
    nblktmodpi /7/ : number of inboard blanket modules in poloidal
    direction (secondary\_cycle\textgreater{}1)
  \item
    nblktmodto /48/ : number of outboard blanket modules in toroidal
    direction (secondary\_cycle\textgreater{}1)
  \item
    nblktmodti /32/ : number of inboard blanket modules in toroidal
    direction (secondary\_cycle\textgreater{}1)
  \item
    tfwmatmax /823.0/ : maximum temperature of first wall material (K)
    (secondary\_cycle\textgreater{}1)
  \item
    fw\_th\_conductivity /28.34/ : thermal conductivity of first wall
    material at 293 K (W/m/K) (Temperature dependence is as for
    unirradiated Eurofer)
  \item
    fvoldw /1.74/ : area coverage factor for vacuum vessel volume
  \item
    fvolsi /1.0/ : area coverage factor for inboard shield volume
  \item
    fvolso /0.64/ : area coverage factor for outboard shield volume
  \item
    fwclfr /0.15/ : first wall coolant fraction (calculated if lpulse=1
    or ipowerflow=1)
  \item
    praddiv : radiation power incident on the divertor (MW)
  \item
    pradfw : radiation power incident on the divertor (MW)
  \item
    pradhcd : radiation power incident on the divertor (MW)
  \item
    pradloss : radiation power incident on the divertor (MW)
  \item
    ptfnuc : nuclear heating in the TF coil (MW)
  \item
    ptfnucpm3 : nuclear heating in the TF coil (MW/m3)
    (blktmodel\textgreater{}0)
  \item
    rdewex : cryostat radius (m)
  \item
    zdewex : cryostat height (m)
  \item
    rpf2dewar /0.5/ : radial distance between outer edge of largest
    ipfloc=3 PF coil (or stellarator modular coil) and cryostat (m)
  \item
    vdewex : cryostat volume (m3)
  \item
    vdewin : vacuum vessel volume (m3)
  \item
    vfshld /0.25/ : coolant void fraction in shield
  \item
    volblkt : volume of blanket (m3)
  \item
    volblkti : volume of inboard blanket (m3)
  \item
    volblkto : volume of outboard blanket (m3)
  \item
    volshld : volume of shield (m3)
  \item
    whtshld : mass of shield (kg)
  \item
    wpenshld : mass of the penetration shield (kg)
  \item
    wtshldi : mass of inboard shield (kg)
  \item
    wtshldo : mass of outboard shield (kg)
  \item
    irefprop /1/ : obsolete
  \item
    fblli2o /0.08/ : lithium oxide fraction of blanket by volume
    (blktmodel=0)
  \item
    fbllipb /0.68/ : lithium lead fraction of blanket by volume
    (blktmodel=0)
  \item
    fblvd /0.0/ : vanadium fraction of blanket by volume (blktmodel=0)
  \item
    wtblli2o : mass of blanket - Li\_2O part (kg)
  \item
    wtbllipb : mass of blanket - Li-Pb part (kg)
  \item
    whtblvd : mass of blanket - vanadium part (kg)
  \item
    whtblli : mass of blanket - lithium part (kg)
  \item
    vfblkt /0.25/ : coolant void fraction in blanket (blktmodel=0),
    (calculated if blktmodel \textgreater{} 0)
  \item
    blktmodel /0/ : switch for blanket/tritium breeding model (but see
    \texttt{iblanket}):

    \begin{itemize}
    \tightlist
    \item
      = 0 original simple model;
    \item
      = 1 KIT model based on a helium-cooled pebble-bed blanket (HCPB)
      reference design
    \end{itemize}
  \item
    declblkt /0.075/ : neutron power deposition decay length of blanket
    structural material (m) (Stellarators only)
  \item
    declfw /0.075/ : neutron power deposition decay length of first wall
    structural material (m) (Stellarators only)
  \item
    declshld /0.075/ : neutron power deposition decay length of shield
    structural material (m) (Stellarators only)
  \item
    blkttype /3/ : Switch for blanket type:

    \begin{itemize}
    \tightlist
    \item
      = 1 WCLL; efficiency taken from WP13-DAS08-T02, EFDA\_D\_2M97B7
    \item
      = 2 HCLL; efficiency taken from WP12-DAS08-T01, EFDA\_D\_2LLNBX
    \item
      = 3 HCPB; efficiency taken from WP12-DAS08-T01, EFDA\_D\_2LLNBX
    \end{itemize}
  \item
    etaiso /0.85/ : isentropic efficiency of FW and blanket coolant
    pumps
  \item
    etahtp /0.95/ : electrical efficiency of primary coolant pumps
  \end{itemize}

  \subsubsection{\texorpdfstring{\href{primary_pumping_variables.html}{primary\_pumping\_variables}}{primary\_pumping\_variables}}\label{primary_pumping_variables}

  \begin{itemize}
  \tightlist
  \item
    gamma\_he /1.667/ FIX : ratio of specific heats for helium
    (primary\_pumping=3)
  \item
    cp\_he /5195/ FIX: specific heat capacity at constant pressure:
    helium (primary\_pumping=3) {[}J/(kg.K){]}
  \item
    t\_in\_bb /573.13/ FIX: temperature in FW and blanket coolant at
    blanket entrance (primary\_pumping=3) {[}K{]}
  \item
    t\_out\_bb /773.13/ FIX: temperature in FW and blanket coolant at
    blanket exit (primary\_pumping=3) {[}K{]}
  \item
    p\_he /8.0e6/ FIX: pressure in FW and blanket coolant at pump exit
    (primary\_pumping=3) {[}Pa{]}
  \item
    dp\_he /5.5e5/ FIX: pressure drop in FW and blanket coolant
    including heat exchanger and pipes (primary\_pump
  \item
    htpmw\_fw\_blkt : mechanical pumping power for FW and blanket
    including heat exchanger and pipes (primary\_pum
  \end{itemize}

  \subsubsection{\texorpdfstring{\href{pfcoil_variables.html}{pfcoil\_variables}}{pfcoil\_variables}}\label{pfcoil_variables}

  \begin{itemize}
  \tightlist
  \item
    ngrpmx /8/ FIX : maximum number of groups of PF coils
  \item
    nclsmx /2/ FIX : maximum number of PF coils in a given group
  \item
    nptsmx /32/ FIX : maximum number of points across the midplane of
    the plasma at which the field from the PF coils is fixed
  \item
    nfixmx /64/ FIX : maximum number of fixed current PF coils
  \item
    alfapf /5.0e-10/ : smoothing parameter used in PF coil current
    calculation at the beginning of pulse (BoP)
  \item
    alstroh /4.0D8/ : allowable hoop stress in Central Solenoid
    structural material (Pa)
  \item
    i\_cs\_stress /0/ : Switch for CS stress calculation:

    \begin{itemize}
    \tightlist
    \item
      = 0 Hoop stress only;
    \item
      = 1 Hoop + Axial stress
    \end{itemize}
  \item
    areaoh : central solenoid cross-sectional area (m2)
  \item
    awpoh : central solenoid conductor+void area (m2)
  \item
    bmaxoh : maximum field in central solenoid at end of flat-top (EoF)
    (T)
  \item
    bmaxoh0 : maximum field in central solenoid at beginning of pulse
    (T)
  \item
    bpf(ngc2) : peak field at coil i (T)
  \item
    cohbop : central solenoid overall current density at beginning of
    pulse (A/m2)
  \item
    coheof /1.85e7/ : central solenoid overall current density at end of
    flat-top (A/m2) (iteration variable 37)
  \item
    cpt(ngc2,6) : current per turn in coil i at time j (A)
  \item
    cptdin(ngc2) /4.0e4/: peak current per turn input for PF coil i (A)
  \item
    curpfb(ngc2) : work array
  \item
    curpff(ngc2) : work array
  \item
    curpfs(ngc2) : work array
  \item
    etapsu /0.9/ : Efficiency of ohmic heating
  \item
    fcohbof : ratio of central solenoid overall current density at
    beginning of flat-top / end of flat-top
  \item
    fcohbop /0.9/ : ratio of central solenoid overall current density at
    beginning of pulse / end of flat-top (iteration variable 41)
  \item
    fcuohsu /0.7/ : copper fraction of strand in central solenoid
  \item
    fcupfsu /0.69/ : copper fraction of cable conductor (PF coils)
  \item
    ipfloc(ngc) /2,2,3/ : switch for locating scheme of PF coil group i:

    \begin{itemize}
    \tightlist
    \item
      = 1 PF coil on top of central solenoid;
    \item
      = 2 PF coil on top of TF coil;
    \item
      = 3 PF coil outside of TF coil
    \end{itemize}
  \item
    ipfres /0/ : switch for PF coil type:

    \begin{itemize}
    \tightlist
    \item
      = 0 superconducting PF coils;
    \item
      = 1 resistive PF coils
    \end{itemize}
  \item
    itr\_sum : total sum of I x turns x radius for all PF coils and CS
    (Am)
  \item
    isumatoh /1/ : switch for superconductor material in central
    solenoid:

    \begin{itemize}
    \tightlist
    \item
      = 1 ITER Nb3Sn critical surface model with standard ITER
      parameters;
    \item
      = 2 Bi-2212 high temperature superconductor (range of validity T
      \textless{} 20K, adjusted field b \textless{} 104 T, B
      \textgreater{} 6 T);
    \item
      = 3 NbTi;
    \item
      = 4 ITER Nb3Sn model with user-specified parameters
    \item
      = 5 WST Nb3Sn parameterisation
    \item
      = 6 REBCO HTS parameterisation
    \end{itemize}
  \item
    isumatpf /1/ : switch for superconductor material in PF coils:

    \begin{itemize}
    \tightlist
    \item
      = 1 ITER Nb3Sn critical surface model with standard ITER
      parameters;
    \item
      = 2 Bi-2212 high temperature superconductor (range of validity T
      \textless{} 20K, adjusted field b \textless{} 104 T, B
      \textgreater{} 6 T);
    \item
      = 3 NbTi;
    \item
      = 4 ITER Nb3Sn model with user-specified parameters
    \item
      = 5 WST Nb3Sn parameterisation
    \end{itemize}
  \item
    jscoh\_bop : central solenoid superconductor critical current
    density (A/m2) at beginning-of-pulse
  \item
    jscoh\_eof : central solenoid superconductor critical current
    density (A/m2) at end-of-flattop
  \item
    jstrandoh\_bop : central solenoid strand critical current density
    (A/m2) at beginning-of-pulse
  \item
    jstrandoh\_eof : central solenoid strand critical current density
    (A/m2) at end-of-flattop
  \item
    ncirt : number of PF circuits (including central solenoid and
    plasma)
  \item
    ncls(ngrpmx+2) /1,1,2/ : number of PF coils in group j
  \item
    nfxfh /7/ : number of filaments the top and bottom of the central
    solenoid should be broken into during scaling (5 - 10 is good)
  \item
    ngrp /3/ : number of groups of PF coils. Symmetric coil pairs should
    all be in the same group
  \item
    nohc : number of PF coils (excluding the central solenoid) + 1
  \item
    ohhghf /0.71/ : central solenoid height / TF coil internal height
  \item
    oh\_steel\_frac /0.5/ : central solenoid steel fraction (iteration
    variable 122)
  \item
    pfcaseth(ngc2) : steel case thickness for PF coil i (m)
  \item
    pfclres /2.5e-8/ : PF coil resistivity (if ipfres=1) (Ohm-m)
  \item
    pfmmax : mass of heaviest PF coil (tonnes)
  \item
    pfrmax : radius of largest PF coil (m)
  \item
    pfsec : PF Coil waste heat (MW)
  \item
    pfwp : PF Coil wall-plug power requirements (MW)
  \item
    powohres : central solenoid resistive power during flattop (W)
  \item
    powpfres : total PF coil resistive losses during flattop (W)
  \item
    ra(ngc2) : inner radius of coil i (m)
  \item
    rb(ngc2) : outer radius of coil i (m)
  \item
    ric(ngc2) : peak current in coil i (MA-turns)
  \item
    rjconpf(ngc2) /3.0e7/ : average winding pack current density of PF
    coil i (A/m2) at time of peak current in that coil (calculated for
    ipfloc=1 coils)
  \item
    rjohc : allowable central solenoid current density at end of
    flat-top (A/m2)
  \item
    rjohc0 : allowable central solenoid current density at beginning of
    pulse (A/m2)
  \item
    rjpfalw(ngc2) : allowable winding pack current density of PF coil i
    (A/m2)
  \item
    rohc : radius to the centre of the central solenoid (m)
  \item
    routr /1.5/ : radial distance (m) from outboard TF coil leg to
    centre of ipfloc=3 PF coils
  \item
    rpf(ngc2) : radius of PF coil i (m)
  \item
    rpf1 /0.0/ : offset (m) of radial position of ipfloc=1 PF coils from
    being directly above the central solenoid
  \item
    rpf2 /-1.63/ : offset (m) of radial position of ipfloc=2 PF coils
    from being at rmajor (offset = rpf2*triang*rminor)
  \item
    s\_tresca\_oh : Tresca stress coils/central solenoid {[}MPa{]}
  \item
    sigpfcalw /500.0/ : maximum permissible tensile stress (MPa) in
    steel coil cases for superconducting PF coils (ipfres=0)
  \item
    sigpfcf /0.666/ : fraction of JxB hoop force supported by steel case
    for superconducting PF coils (ipfres=0)
  \item
    sxlg(ngc2,ngc2) : mutual inductance matrix (H)
  \item
    tmargoh : Central solenoid temperature margin (K)
  \item
    turns(ngc2) : number of turns in PF coil i
  \item
    vf(ngc2) /0.3/ : winding pack void fraction of PF coil i for coolant
  \item
    vfohc /0.3/ : void fraction of central solenoid conductor for
    coolant
  \item
    vsbn : total flux swing available for burn (Wb)
  \item
    vsefbn : flux swing from PF coils for burn (Wb)
  \item
    vsefsu : flux swing from PF coils for startup (Wb)
  \item
    vseft : total flux swing from PF coils (Wb)
  \item
    vsoh : total flux swing from the central solenoid (Wb)
  \item
    vsohbn : central solenoid flux swing for burn (Wb)
  \item
    vsohsu : central solenoid flux swing for startup (Wb)
  \item
    vssu : total flux swing for startup (eqn 51 to enforce
    vssu=vsres+vsind) (Wb)
  \item
    vstot : total flux swing for pulse (Wb)
  \item
    waves(ngc2, 6) : used in current waveform of PF coils/central
    solenoid
  \item
    whtpf : total mass of the PF coil conductor (kg)
  \item
    whtpfs : total mass of the PF coil structure (kg)
  \item
    wtc(ngc2) : conductor mass for PF coil i (kg)
  \item
    wts(ngc2) : structure mass for PF coil i (kg)
  \item
    zh(ngc2) : upper point of PF coil i (m)
  \item
    zl(ngc2) : lower point of PF coil i (m)
  \item
    zpf(ngc2) : z (height) location of PF coil i (m)
  \item
    zref(ngrpmx) /../ : PF coil vertical positioning adjuster:

    \begin{itemize}
    \tightlist
    \item
      - for groups j with ipfloc(j) = 1; zref(j) is ignored
    \item
      - for groups j with ipfloc(j) = 2 AND itart=1 (only); zref(j) is
      distance of centre of PF coil from inside edge of TF coil
      (remember that PF coils for STs lie within the TF coil)
    \item
      - for groups j with ipfloc(j) = 3; zref(j) = ratio of height of
      coil group j to plasma minor radius
    \end{itemize}
  \item
    bmaxcs\_lim : Central solenoid max field limit {[}T{]}
  \item
    fbmaxcs : F-value for CS mmax field (cons. 79, itvar 149)
  \end{itemize}

  \subsubsection{\texorpdfstring{\href{tfcoil_variables.html}{tfcoil\_variables}}{tfcoil\_variables}}\label{tfcoil_variables}

  \begin{itemize}
  \item
    acasetf : external case area per coil (inboard leg) (m2)
  \item
    acasetfo : external case area per coil (outboard leg) (m2)
  \item
    acndttf : area of the cable conduit (m2)
  \item
    acond : conductor area (winding pack) (m2)
  \item
    acstf : internal area of the cable space (m2)
  \item
    insulation\_area : single turn insulation area (m2)
  \item
    aiwp : winding pack insulation area (m2)
  \item
    alstrtf /6.0D8/ : allowable Tresca stress in TF coil structural
    material (Pa)
  \item
    arealeg : outboard TF leg area (m2)
  \item
    aswp : winding pack structure area (m2)
  \item
    avwp : winding pack void (He coolant) area (m2)
  \item
    awphec : winding pack He coil area (m2)
  \item
    bcritsc /24.0/ : upper critical field (T) for Nb3Sn superconductor
    at zero temperature and strain (isumattf=4, =bc20m)
  \item
    bmaxtf : mean peak field at TF coil (T)
  \item
    bmaxtfrp : peak field at TF conductor with ripple (T)
  \item
    casestr : case strain
  \item
    casthi /0.0/ : EITHER: inboard TF coil case plasma side thickness
    (m) (calculated for stellarators)
  \item
    casthi\_fraction /0.05/ : OR: inboard TF coil case plasma side
    thickness as a fraction of tfcth
  \item
    casths /0.0/ : EITHER: inboard TF coil sidewall case thickness (m)
    (calculated for stellarators)
  \item
    casths\_fraction /0.03/ : OR: inboard TF coil sidewall case
    thickness as a fraction of tftort
  \item
    conductor\_width : Width of square conductor (m)
  \item
    leno : Dimension of each turn including inter-turn insulation (m)
  \item
    leni : Dimension of space inside conductor (m)
  \item
    acs : Area of space inside conductor (m2)
  \item
    cdtfleg : TF outboard leg current density (A/m2) (resistive coils
    only)
  \item
    cforce : centering force on inboard leg (per coil) (N/m)
  \item
    cph2o /4180.0/ FIX : specific heat capacity of water (J/kg/K)
  \item
    cpttf /7.0e4/ : TF coil current per turn (A). (calculated for
    stellarators) (calculated for integer-turn TF coils
    i\_tf\_turns\_integer=1) (iteration variable 60)
  \item
    cpttf\_max /9.0e4/ : Max TF coil current per turn {[}A{]}. (For
    stellarators and i\_tf\_turns\_integer=1) (constraint equation 77)
  \item
    dcase /8000.0/ : density of coil case (kg/m3)
  \item
    dcond(6) /9000.0/ : density of superconductor type given by
    isumattf/isumatoh/isumatpf (kg/m3)
  \item
    dcondins /1800.0/ : density of conduit + ground-wall insulation
    (kg/m3)
  \item
    dcopper /8900.0/ : density of copper (kg/m3)
  \item
    deflect : TF coil deflection at full field (m)
  \item
    denh2o /985.0/ FIX : density of water (kg/m3)
  \item
    dhecoil /0.005/ : diameter of He coil in TF winding (m)
  \item
    estotf : stored energy per TF coil (GJ) OBSOLETE
  \item
    estotftgj : total stored energy in the toroidal field (GJ)
  \item
    eyins /2.0e10/ : insulator Young's modulus (Pa) (default value from
    DDD11-2 v2 2 (2009))
  \item
    eyoung(2) : work array used in stress calculation (Pa)
  \item
    eystl /2.05e11/ : steel case Young's modulus (Pa) (default value
    from DDD11-2 v2 2 (2009))
  \item
    eywp /6.6e8/ : winding pack Young's modulus (Pa)
  \item
    eyzwp : winding pack vertical Young's modulus (Pa) (tfc\_model=1)
  \item
    farc4tf /0.7/ : factor to size height of point 4 on TF coil
  \item
    fcutfsu /0.69/ : copper fraction of cable conductor (TF coils)
    (iteration variable 59)
  \item
    fhts /0.5/ : technology adjustment factor for critical current
    density fit for isumat..=2 Bi-2212 superconductor, to describe the
    level of technology assumed (i.e. to account for stress, fatigue,
    radiation, AC losses, joints or manufacturing variations; 1.0 would
    be very optimistic)
  \item
    insstrain : radial strain in insulator (tfc\_model=1)
  \item
    i\_tf\_tresca /0/ : switch for TF coil conduit Tresca stress
    criterion:

    \begin{itemize}
    \tightlist
    \item
      = 0 Tresca (no adjustment);
    \item
      = 1 Tresca with CEA adjustment factors (radial+2\%, vertical+60\%)
    \end{itemize}
  \item
    i\_tf\_turns\_integer /0/ : switch for TF coil integer/non-integer
    turns

    \begin{itemize}
    \tightlist
    \item
      = 0 non-integer turns;
    \item
      = 1 integer turns
    \end{itemize}
  \item
    isumattf /1/ : switch for superconductor material in TF coils:

    \begin{itemize}
    \tightlist
    \item
      = 1 ITER Nb3Sn critical surface model with standard ITER
      parameters;
    \item
      = 2 Bi-2212 high temperature superconductor (range of validity T
      \textless{} 20K, adjusted field b \textless{} 104 T, B
      \textgreater{} 6 T);
    \item
      = 3 NbTi;
    \item
      = 4 ITER Nb3Sn model with user-specified parameters
    \item
      = 5 WST Nb3Sn parameterisation
    \item
      = 6 REBCO HTS tape in CroCo strand
    \end{itemize}
  \item
    itfsup /1/ : switch for TF coil conductor model:

    \begin{itemize}
    \tightlist
    \item
      = 0 copper;
    \item
      = 1 superconductor
    \end{itemize}
  \item
    jbus /1.25e6/ : bussing current density (A/m2)
  \item
    jeff(2) : work array used in stress calculation (A/m2)
  \item
    jwdgcrt : critical current density for winding pack (A/m2)
  \item
    jwdgpro : allowable TF coil winding pack current density, for dump
    temperature rise protection (A/m2)
  \item
    jwptf : winding pack current density (A/m2)
  \item
    n\_pancake /10/ : Number of pancakes in TF coil
    (i\_tf\_turns\_integer=1)
  \item
    n\_layer /20/ : Number of layers in TF coil
    (i\_tf\_turns\_integer=1)
  \item
    oacdcp /1.4e7/ : overall current density in TF coil inboard legs
    (A/m2) (iteration variable 12)
  \item
    poisson /0.3/ : Poisson's ratio for TF stress calculation (assumed
    constant over entire coil)
  \item
    radtf(3) : work array used in stress calculation (m)
  \item
    rbmax : radius of maximum TF B-field (m)
  \item
    rhotfleg : TF coil leg resistance (ohm)
  \item
    ripmax /1.0/ : maximum allowable toroidal field ripple amplitude at
    plasma edge (\%)
  \item
    ripple : peak/average toroidal field ripple at plasma edge (\%)
  \item
    ritfc : total (summed) current in TF coils (A)
  \item
    sigrad : radial TF coil stress (MPa)
  \item
    sigrcon : radial stress in the conductor conduit (Pa)
  \item
    sigrtf(2) : radial stress in TF coil regions (Pa)
  \item
    sigtan : transverse TF coil stress (MPa)
  \item
    sigtcon : tangential stress in the conductor conduit (Pa)
  \item
    sigttf(2) : tangential stress in TF coil regions (Pa)
  \item
    s\_tresca\_case : TF coil case Tresca stress (MPa)
  \item
    s\_tresca\_cond : TF coil conduit Tresca stress (MPa)
  \item
    s\_vmises\_case : TF coil case von Mises stress (MPa)
  \item
    s\_vmises\_cond : TF coil conduit von Mises stress (MPa)
  \item
    sigver : vertical TF coil stress (MPa)
  \item
    sigvert : vertical tensile stress in TF coil (Pa)
  \item
    sigvvall /9.3e7/ : allowable stress from TF quench in vacuum vessel
    (Pa)
  \item
    strncon\_cs /-0.005/ : strain in CS superconductor material (used in
    Nb3Sn critical surface model, isumatoh=1, 4 or 5)
  \item
    strncon\_pf /-0.005/ : strain in PF superconductor material (used in
    Nb3Sn critical surface model, isumatph=1, 4 or 5)
  \item
    strncon\_tf /-0.005/ : strain in TF superconductor material (used in
    Nb3Sn critical surface model, isumattf=1, 4 or 5)
  \item
    strtf1 : Constrained stress in TF conductor conduit (Pa)
  \item
    strtf2 : Constrained stress in TF coil case (Pa)
  \item
    quench\_model /exponential/ : switch for TF coil quench model:

    \begin{itemize}
    \tightlist
    \item
      = `exponential' exponential quench with constant discharge
      resistor
    \item
      = `linear' quench with constant voltage
    \end{itemize}

    Only applies to REBCO magnet at present
  \item
    quench\_detection\_ef /0.0/ : Electric field at which TF quench is
    detected and discharge begins (V/m)
  \item
    time1 : Time at which TF quench is detected (s)
  \item
    taucq : allowable TF quench time (s)
  \item
    tcritsc /16.0/ : critical temperature (K) for superconductor at zero
    field and strain (isumattf=4, =tc0m)
  \item
    tdmptf /10.0/ : fast discharge time for TF coil in event of quench
    (s) (iteration variable 56) For REBCO model, meaning depends on
    quench\_model:
  \item
    exponential quench : e-folding time (s)
  \item
    linear quench : discharge time (s)
  \item
    tfareain : area of inboard midplane TF legs (m2)
  \item
    tfboreh : TF coil horizontal inner bore (m)
  \item
    tfborev : TF coil vertical inner bore (m)
  \item
    tfbusl : TF coil bus length (m)
  \item
    tfbusmas : TF coil bus mass (kg)
  \item
    tfckw : available DC power for charging the TF coils (kW)
  \item
    tfc\_model /1/ : switch for TF coil magnet stress model:

    \begin{itemize}
    \tightlist
    \item
      = 0 simple model (solid copper coil)
    \item
      = 1 CCFE two-layer stress model; superconductor
    \end{itemize}
  \item
    tfcmw : peak power per TF power supply (MW)
  \item
    tfcpmw : peak resistive TF coil inboard leg power (MW)
  \item
    tfcryoarea : surface area of toroidal shells covering TF coils (m2)
  \item
    tficrn : TF coil half-width - inner bore (m)
  \item
    tfind : TF coil inductance (H)
  \item
    tfinsgap /0.010/ : TF coil WP insertion gap (m)
  \item
    tflegmw : TF coil outboard leg resistive power (MW)
  \item
    tflegres /2.5e-8/ : resistivity of a TF coil leg and bus(Ohm-m)
  \item
    tfleng : TF coil circumference (m)
  \item
    tfno /16.0/ : number of TF coils (default = 50 for stellarators)
  \item
    tfocrn : TF coil half-width - outer bore (m)
  \item
    tfsai : area of the inboard TF coil legs (m2)
  \item
    tfsao : area of the outboard TF coil legs (m2)
  \item
    tftmp /4.5/ : peak helium coolant temperature in TF coils and PF
    coils (K)
  \item
    tftort : TF coil toroidal thickness (m)
  \item
    thicndut /8.0e-4/ : conduit insulation thickness (m)
  \item
    layer\_ins /0/ : Additional insulation thickness between layers (m)
  \item
    thkcas /0.3/ : inboard TF coil case outer (non-plasma side)
    thickness (m) (iteration variable 57) (calculated for stellarators)
  \item
    thkwp /0.0/ : radial thickness of winding pack (m) (iteration
    variable 140)
  \item
    thwcndut /8.0e-3/ : TF coil conduit case thickness (m) (iteration
    variable 58)
  \item
    tinstf /0.018/ : ground insulation thickness surrounding winding
    pack (m)
  \item
    Includes allowance for 10 mm insertion gap. (calculated for
    stellarators)
  \item
    tmargmin\_tf /0/ : minimum allowable temperature margin : TF coils
    (K)
  \item
    tmargmin\_cs /0/ : minimum allowable temperature margin : CS (K)
  \item
    tmargmin /0/ : minimum allowable temperature margin : TFC AND CS (K)
  \item
    temp\_margin : temperature margin (K)
  \item
    tmargtf : TF coil temperature margin (K)
  \item
    tmaxpro /150.0/ : maximum temp rise during a quench for protection
    (K)
  \item
    tmax\_croco /200.0/ : CroCo strand: maximum permitted temp during a
    quench (K)
  \item
    tmax\_jacket /150.0/ : Jacket: maximum temp during a quench (K)
  \item
    croco\_quench\_temperature : CroCo strand: Actual temp reached
    during a quench (K)
  \item
    tmpcry /4.5/ : coil temperature for cryogenic plant power
    calculation (K)
  \item
    turnstf : number of turns per TF coil
  \item
    vdalw /20.0/ : max voltage across TF coil during quench (kV)
    (iteration variable 52)
  \item
    vforce : vertical separating force on inboard leg/coil (N)
  \item
    vftf /0.4/ : coolant fraction of TFC `cable' (itfsup=1), or of TFC
    leg (itfsup=0)
  \item
    voltfleg : volume of each TF coil outboard leg (m3)
  \item
    vtfkv : TF coil voltage for resistive coil including bus (kV)
  \item
    vtfskv : voltage across a TF coil during quench (kV)
  \item
    whtcas : mass per coil of external case (kg)
  \item
    whtcon : TF coil conductor mass per coil (kg)
  \item
    whtconcu : copper mass in TF coil conductor (kg/coil)
  \item
    whtconin : conduit insulation mass in TF coil conductor (kg/coil)
  \item
    whtconsc : superconductor mass in TF coil cable (kg/coil)
  \item
    whtconsh : steel conduit mass in TF coil conductor (kg/coil)
  \item
    whtgw : mass of ground-wall insulation layer per coil (kg/coil)
  \item
    whttf : total mass of the TF coils (kg)
  \item
    windstrain : longitudinal strain in winding pack (tfc\_model=1)
  \item
    wwp1 : width of first step of winding pack (m)
  \item
    wwp2 : width of second step of winding pack (m)
  \item
    \textbf{Superconducting TF coil shape parameters} (see also
    farc4tf);\\
    the TF inner surface top half is approximated by four circular arcs.
    Arc 1 goes through points 1 and 2 on the inner surface. Arc 2 goes
    through points 2 and 3, etc.
  \item
    dthet(4) : angle of arc i (rad)
  \item
    radctf(4) : radius of arc i (m)
  \item
    xarc(5) : x location of arc point i on surface (m)
  \item
    xctfc(4) : x location of arc centre i (m)
  \item
    yarc(5) : y location of arc point i on surface (m)
  \item
    yctfc(4) : y location of arc centre i (m)
  \item
    tfa(4) : Horizontal radius of inside edge of TF coil (m)
  \item
    tfb(4) : Vertical radius of inside edge of TF coil (m)
  \item
    \textbf{Quantities relating to the spherical tokamak model
    (itart=1)} (and in some cases, also to resistive TF coils,
    itfsup=0):
  \item
    drtop /0.0/ : centrepost taper maximum radius adjustment (m)
  \item
    dztop /0.0/ : centrepost taper height adjustment (m)
  \item
    etapump /0.8/ : centrepost coolant pump efficiency
  \item
    fcoolcp /0.3/ : coolant fraction of TF coil inboard legs (iteration
    variable 23)
  \item
    frhocp /1.0/ : centrepost resistivity enhancement factor
  \item
    kcp /330.0/ FIX : thermal conductivity of centrepost (W/m/K)
  \item
    kh2o /0.651/ FIX : thermal conductivity of water (W/m/K)
  \item
    muh2o /4.71e-4/ FIX : water dynamic viscosity (kg/m/s)
  \item
    ncool : number of centrepost coolant tubes
  \item
    ppump : centrepost coolant pump power (W)
  \item
    prescp : resistive power in the centrepost (W)
  \item
    ptempalw /200.0/ : maximum peak centrepost temperature (C)
    (constraint equation 44)
  \item
    rcool /0.005/ : average radius of coolant channel (m) (iteration
    variable 69)
  \item
    rhocp : TF coil inboard leg resistivity (Ohm-m)
  \item
    tcoolin /40.0/ : centrepost coolant inlet temperature (C)
  \item
    tcpav /100.0/ : average temp of TF coil inboard leg conductor (C)
    (resistive coils) (iteration variable 20)
  \item
    tcpav2 : centrepost average temperature (C) (for consistency)
  \item
    tcpmax : peak centrepost temperature (C)
  \item
    vcool /20.0/ : max centrepost coolant flow speed at midplane (m/s)
    (iteration variable 70)
  \item
    volcp : total volume of TF coil inboard legs (m3)
  \item
    whtcp : mass of TF coil inboard legs (kg)
  \item
    whttflgs : mass of the TF coil legs (kg)
  \end{itemize}

  \subsubsection{\texorpdfstring{\href{structure_variables.html}{structure\_variables}}{structure\_variables}}\label{structure_variables}

  \begin{itemize}
  \tightlist
  \item
    aintmass : intercoil structure mass (kg)
  \item
    clgsmass : gravity support structure for TF coil, PF coil and
    intercoil support systems (kg)
  \item
    coldmass : total mass of components at cryogenic temperatures (kg)
  \item
    fncmass : PF coil outer support fence mass (kg)
  \item
    gsmass : reactor core gravity support mass (kg)
  \end{itemize}

  \subsubsection{\texorpdfstring{\href{vacuum_variables.html}{vacuum\_variables}}{vacuum\_variables}}\label{vacuum_variables}

  \begin{itemize}
  \item
    vacuum\_model /old/ : switch for vacuum pumping model:

    \begin{itemize}
    \tightlist
    \item
      = `old' for old detailed ETR model;
    \item
      = `simple' for simple steady-state model with comparison to ITER
      cryopumps
    \end{itemize}
  \item
    niterpump : number of high vacuum pumps (real number), each with the
    throughput of one ITER cryopump (50 Pa m3 s-1), all operating at the
    same time (vacuum\_model = `simple')
  \item
    ntype /1/ : switch for vacuum pump type:

    \begin{itemize}
    \tightlist
    \item
      = 0 for turbomolecular pump (magnetic bearing) with speed of 2.0
      m3/s (1.95 for N2, 1.8 for He, 1.8 for DT);
    \item
      = 1 for compound cryopump with nominal speed of 10.0 m3/s (9.0 for
      N2, 5.0 for He and 25.0 for DT)
    \end{itemize}
  \item
    nvduct : number of ducts (torus to pumps)
  \item
    dlscal : vacuum system duct length scaling
  \item
    pbase /5.0e-4/ : base pressure during dwell before gas pre-fill(Pa)
  \item
    prdiv /0.36/ : divertor chamber pressure during burn (Pa)
  \item
    pumptp /1.2155D22/ : Pump throughput (molecules/s) (default is ITER
    value)
  \item
    rat /1.3e-8/ : plasma chamber wall outgassing rate (Pa-m/s)
  \item
    tn /300.0/ : neutral gas temperature in chamber (K)
  \item
    vacdshm : mass of vacuum duct shield (kg)
  \item
    vcdimax : diameter of duct passage (m)
  \item
    vpumpn : number of high vacuum pumps
  \item
    dwell\_pump /0/ : switch for dwell pumping options:

    \begin{itemize}
    \tightlist
    \item
      = 0 pumping only during tdwell;
    \item
      = 1 pumping only during tramp
    \item
      = 2 pumping during tdwell + tramp
    \end{itemize}
  \item
    \textbf{The following are used in the Battes, Day and Rohde
    pump-down model See ``Basic considerations on the pump-down time in
    the dwell phase of a pulsed fusion DEMO''
    http://dx.doi.org/10.1016/j.fusengdes.2015.07.011)
    (vacuum\_model=simple'):}
  \item
    pumpareafraction /0.0203/ : area of one pumping port as a fraction
    of plasma surface area
  \item
    pumpspeedmax /27.3/ : maximum pumping speed per unit area for
    deuterium \& tritium, molecular flow
  \item
    pumpspeedfactor /0.167/ : effective pumping speed reduction factor
    due to duct impedance
  \item
    initialpressure /1.0/ : initial neutral pressure at the beginning of
    the dwell phase (Pa)
  \item
    pbase /5.0e-4/ : base pressure during dwell before gas pre-fill (Pa)
  \item
    outgasindex /1.0/ : outgassing decay index
  \item
    outgasfactor /0.0235/ : outgassing prefactor kw: outgassing rate at
    1 s per unit area (Pa m s-1)
  \end{itemize}

  \subsubsection{\texorpdfstring{\href{pf_power_variables.html}{pf\_power\_variables}}{pf\_power\_variables}}\label{pf_power_variables}

  \begin{itemize}
  \tightlist
  \item
    acptmax : average of currents in PF circuits (A)
  \item
    ensxpfm : maximum stored energy in the PF circuits (MJ)
  \item
    iscenr /2/ : Switch for PF coil energy storage option:

    \begin{itemize}
    \tightlist
    \item
      = 1 all power from MGF (motor-generator flywheel) units;
    \item
      = 2 all pulsed power from line;
    \item
      = 3 PF power from MGF, heating from line
    \end{itemize}

    (In fact, options 1 and 3 are not treated differently)
  \item
    pfckts : number of PF coil circuits
  \item
    spfbusl : total PF coil circuit bus length (m)
  \item
    spsmva : sum of PF power supply ratings (MVA)
  \item
    srcktpm : sum of resistive PF coil power (kW)
  \item
    vpfskv : PF coil voltage (kV)
  \item
    peakpoloidalpower : Peak absolute rate of change of stored energy in
    poloidal field (MW) (11/01/16)
  \item
    maxpoloidalpower /1000/ : Maximum permitted absolute rate of change
    of stored energy in poloidal field (MW)
  \item
    poloidalpower : Poloidal power usage at time t (MW)
  \end{itemize}

  \subsubsection{\texorpdfstring{\href{heat_transport_variables.html}{heat\_transport\_variables}}{heat\_transport\_variables}}\label{heat_transport_variables}

  \begin{itemize}
  \tightlist
  \item
    baseel /5.0e6/ : base plant electric load (W)
  \item
    crypmw : cryogenic plant power (MW)
  \item
    etatf /0.9/ : AC to resistive power conversion for TF coils
  \item
    etath /0.35/ : thermal to electric conversion efficiency if
    secondary\_cycle=2; otherwise calculated
  \item
    fachtmw : facility heat removal (MW)
  \item
    fcsht : total baseline power required at all times (MW)
  \item
    fgrosbop : scaled fraction of gross power to balance-of-plant
  \item
    fmgdmw /0.0/ : power to mgf (motor-generator flywheel) units (MW)
    (ignored if iscenr=2)
  \item
    fpumpblkt /0.005/ : fraction of total blanket thermal power required
    to drive the blanket coolant pumps (default assumes water coolant)
    (secondary\_cycle=0)
  \item
    fpumpdiv /0.005/ : fraction of total divertor thermal power required
    to drive the divertor coolant pumps (default assumes water coolant)
  \item
    fpumpfw /0.005/ : fraction of total first wall thermal power
    required to drive the FW coolant pumps (default assumes water
    coolant) (secondary\_cycle=0)
  \item
    fpumpshld /0.005/ : fraction of total shield thermal power required
    to drive the shield coolant pumps (default assumes water coolant)
  \item
    htpmw\_min /0.0/ : Minimum total electrical power for primary
    coolant pumps (MW) NOT RECOMMENDED
  \item
    helpow : heat removal at cryogenic temperatures (W)
  \item
    htpmw :: heat transport system electrical pump power (MW)
  \item
    htpmw\_blkt /0.0/ : blanket coolant mechanical pumping power (MW)
  \item
    htpmw\_div /0.0/ : divertor coolant mechanical pumping power (MW)
  \item
    htpmw\_fw /0.0/ : first wall coolant mechanical pumping power (MW)
  \item
    htpmw\_shld /.0/ : shield and vacuum vessel coolant mechanical
    pumping power (MW)
  \item
    htpsecmw : Waste power lost from primary coolant pumps (MW)
  \item
    ipowerflow /1/ : switch for power flow model:

    \begin{itemize}
    \tightlist
    \item
      = 0 pre-2014 version;
    \item
      = 1 comprehensive 2014 model
    \end{itemize}
  \item
    iprimnloss /0/ : switch for lost neutron power through holes
    destiny:

    \begin{itemize}
    \tightlist
    \item
      = 0 does not contribute to energy generation cycle;
    \item
      = 1 contributes to energy generation cycle
    \end{itemize}

    (ipowerflow=0)
  \item
    iprimshld /1/ : switch for shield thermal power destiny:

    \begin{itemize}
    \tightlist
    \item
      = 0 does not contribute to energy generation cycle;
    \item
      = 1 contributes to energy generation cycle
    \end{itemize}
  \item
    nphx : number of primary heat exchangers
  \item
    pacpmw : total pulsed power system load (MW)
  \item
    peakmva : peak MVA requirement
  \item
    pfwdiv : heat removal from first wall/divertor (MW)
  \item
    pgrossmw : gross electric power (MW)
  \item
    pinjht : power dissipated in heating and current drive system (MW)
  \item
    pinjmax : maximum injector power during pulse (heating and
    ramp-up/down phase) (MW)
  \item
    pinjwp : injector wall plug power (MW)
  \item
    pnetelmw : net electric power (MW)
  \item
    precircmw : recirculating electric power (MW)
  \item
    priheat : total thermal power removed from fusion core (MW)
  \item
    psecdiv : Low-grade heat lost in divertor (MW)
  \item
    psechcd : Low-grade heat lost into HCD apparatus (MW)
  \item
    psechtmw : Low-grade heat (MW)
  \item
    pseclossmw : Low-grade heat (VV + lost)(MW)
  \item
    psecshld : Low-grade heat deposited in shield (MW)
  \item
    pthermmw : High-grade heat useful for electric production (MW)
  \item
    pwpm2 /150.0/ : base AC power requirement per unit floor area (W/m2)
  \item
    tfacpd : total steady state TF coil AC power demand (MW)
  \item
    tlvpmw : estimate of total low voltage power (MW)
  \item
    trithtmw /15.0/ : power required for tritium processing (MW)
  \item
    tturb : coolant temperature at turbine inlet (K) (secondary\_cycle =
    3,4)
  \item
    vachtmw /0.5/ : vacuum pump power (MW)
  \end{itemize}

  \subsubsection{\texorpdfstring{\href{times_variables.html}{times\_variables}}{times\_variables}}\label{times_variables}

  \begin{itemize}
  \tightlist
  \item
    pulsetimings /0.0/ : switch for pulse timings (if lpulse=1):

    \begin{itemize}
    \tightlist
    \item
      = 0, tohs = Ip(MA)/0.1 tramp, tqnch = input;
    \item
      = 1, tohs = iteration var or input. tramp/tqnch max of input or
      tohs
    \end{itemize}
  \item
    tburn /1000.0/ : burn time (s) (calculated if lpulse=1)
  \item
    tburn0 : burn time (s) - used for internal consistency
  \item
    tcycle : full cycle time (s)
  \item
    tdown : down time (s)
  \item
    tdwell /1800.0/ : time between pulses in a pulsed reactor (s)
    (iteration variable 17)
  \item
    theat /10.0/ : heating time, after current ramp up (s)
  \item
    tim(6) : array of time points during plasma pulse (s)
  \item
    timelabel(6) : array of time labels during plasma pulse (s)
  \item
    intervallabel(6) : time intervals - as strings (s)
  \item
    tohs /30.0/ : plasma current ramp-up time for current initiation (s)
    (but calculated if lpulse=0) (iteration variable 65)
  \item
    tohsin /0.0/ : switch for plasma current ramp-up time (if lpulse=0):

    \begin{itemize}
    \tightlist
    \item
      = 0, tohs = tramp = tqnch = Ip(MA)/0.5;
    \item
      \textless{}\textgreater{}0, tohs = tohsin; tramp, tqnch are input
    \end{itemize}
  \item
    tpulse : pulse length = tohs + theat + tburn + tqnch
  \item
    tqnch /15.0/ : shut down time for PF coils (s); if pulsed, = tohs
  \item
    tramp /15.0/ : initial PF coil charge time (s); if pulsed, = tohs
  \end{itemize}

  \subsubsection{\texorpdfstring{\href{buildings_variables.html}{buildings\_variables}}{buildings\_variables}}\label{buildings_variables}

  \begin{itemize}
  \tightlist
  \item
    admv /1.0e5/ : administration building volume (m3)
  \item
    admvol : volume of administration buildings (m3)
  \item
    clh1 /2.5/ : vertical clearance from TF coil to cryostat (m)
    (calculated for tokamaks)
  \item
    clh2 /15.0/ : clearance beneath TF coil to foundation (including
    basement) (m)
  \item
    conv /6.0e4/ : control building volume (m3)
  \item
    convol : volume of control, protection and i\&c building (m3)
  \item
    cryvol : volume of cryoplant building (m3)
  \item
    efloor : effective total floor space (m2)
  \item
    elevol : volume of electrical equipment building (m3)
  \item
    esbldgm3 /1.0e3/ : volume of energy storage equipment building (m3)
    (not used if lpulse=0)
  \item
    fndt /2.0/ : foundation thickness (m)
  \item
    hccl /5.0/ : clearance around components in hot cell (m)
  \item
    hcwt /1.5/ : hot cell wall thickness (m)
  \item
    mbvfac /2.8/ : maintenance building volume multiplication factor
  \item
    pfbldgm3 /2.0e4/ : volume of PF coil power supply building (m3)
  \item
    pibv /2.0e4/ : power injection building volume (m3)
  \item
    rbrt /1.0/ : reactor building roof thickness (m)
  \item
    rbvfac /1.6/ : reactor building volume multiplication factor
  \item
    rbvol : reactor building volume (m3)
  \item
    rbwt /2.0/ : reactor building wall thickness (m)
  \item
    rmbvol : volume of maintenance and assembly building (m3)
  \item
    row /4.0/ : clearance to building wall for crane operation (m)
  \item
    rxcl /4.0/ : clearance around reactor (m)
  \item
    shmf /0.5/ : fraction of shield mass per TF coil to be moved in the
    maximum shield lift
  \item
    shov /1.0e5/ : shops and warehouse volume (m3)
  \item
    shovol :volume of shops and buildings for plant auxiliaries (m3)
  \item
    stcl /3.0/ : clearance above crane to roof (m)
  \item
    tfcbv /2.0e4/ : volume of TF coil power supply building (m3)
    (calculated if TF coils are superconducting)
  \item
    trcl /1.0/ : transportation clearance between components (m)
  \item
    triv /4.0e4/ : volume of tritium, fuel handling and health physics
    buildings (m3)
  \item
    volnucb : sum of nuclear buildings volumes (m3)
  \item
    volrci : internal volume of reactor building (m3)
  \item
    wgt /5.0e5/ : reactor building crane capacity (kg) (calculated if 0
    is input)
  \item
    wgt2 /1.0e5/ : hot cell crane capacity (kg) (calculated if 0 is
    input)
  \item
    wrbi : distance from centre of machine to building wall (m), i.e.
    reactor building half-width
  \item
    wsvfac /1.9/ : warm shop building volume multiplication factor
  \item
    wsvol : volume of warm shop building (m3)
  \end{itemize}

  \subsubsection{\texorpdfstring{\href{build_variables.html}{build\_variables}}{build\_variables}}\label{build_variables}

  \begin{itemize}
  \tightlist
  \item
    aplasmin /0.25/ : minimum minor radius (m)
  \item
    blarea : blanket total surface area (m2)
  \item
    blareaib : inboard blanket surface area (m2)
  \item
    blareaob : outboard blanket surface area (m2)
  \item
    blbmith /0.17/ : inboard blanket box manifold thickness (m)
    (blktmodel\textgreater{}0)
  \item
    blbmoth /0.27/ : outboard blanket box manifold thickness (m)
    (blktmodel\textgreater{}0)
  \item
    blbpith /0.30/ : inboard blanket base plate thickness (m)
    (blktmodel\textgreater{}0)
  \item
    blbpoth /0.35/ : outboard blanket base plate thickness (m)
    (blktmodel\textgreater{}0)
  \item
    blbuith /0.365/ : inboard blanket breeding zone thickness (m)
    (blktmodel\textgreater{}0) (iteration variable 90)
  \item
    blbuoth /0.465/ : outboard blanket breeding zone thickness (m)
    (blktmodel\textgreater{}0) (iteration variable 91)
  \item
    blnkith /0.115/ : inboard blanket thickness (m); (calculated if
    blktmodel \textgreater{} 0) (=0.0 if iblnkith=0)
  \item
    blnkoth /0.235/ : outboard blanket thickness (m); calculated if
    blktmodel \textgreater{} 0
  \item
    blnktth : top blanket thickness (m), = mean of inboard and outboard
    blanket thicknesses
  \item
    bore /1.42/ : central solenoid inboard radius (m) (iteration
    variable 29)
  \item
    clhsf /4.268/ : cryostat lid height scaling factor (tokamaks)
  \item
    ddwex /0.07/ : cryostat thickness (m)
  \item
    ddwi /0.07/ : vacuum vessel thickness (TF coil / shield) (m)
  \item
    fcspc /0.6/ : Fraction of space occupied by CS pre-compression
    structure
  \item
    fmsbc /0.0/ : Martensitic fraction of steel in (non-existent!)
    bucking cylinder
  \item
    fmsbl /0.0/ : Martensitic fraction of steel in blanket
  \item
    fmsdwe /0.0/ : Martensitic fraction of steel in cryostat
  \item
    fmsdwi /0.0/ : Martensitic fraction of steel in vacuum vessel
  \item
    fmsfw /0.0/ : Martensitic fraction of steel in first wall
  \item
    fmsoh /0.0/ : Martensitic fraction of steel in central solenoid
  \item
    fmssh /0.0/ : Martensitic fraction of steel in shield
  \item
    fmstf /0.0/ : Martensitic fraction of steel in TF coil
  \item
    fseppc /3.5d8/ : Separation force in CS coil pre-compression
    structure
  \item
    fwarea : first wall total surface area (m2)
  \item
    fwareaib : inboard first wall surface area (m2)
  \item
    fwareaob : outboard first wall surface area (m2)
  \item
    fwith : inboard first wall thickness, initial estimate (m)
  \item
    fwoth : outboard first wall thickness, initial estimate (m)
  \item
    gapds /0.155/ : gap between inboard vacuum vessel and thermal shield
    (m) (iteration variable 61)
  \item
    gapoh /0.08/ : gap between central solenoid and TF coil (m)
    (iteration variable 42)
  \item
    gapomin /0.234/ : minimum gap between outboard vacuum vessel and TF
    coil (m) (iteration variable 31)
  \item
    gapsto : gap between outboard vacuum vessel and TF coil (m)
  \item
    hmax : maximum (half-)height of TF coil (inside edge) (m)
  \item
    hpfdif : difference in distance from midplane of upper and lower
    portions of TF legs (non-zero for single-null devices) (m)
  \item
    hpfu : height to top of (upper) TF coil leg (m)
  \item
    hr1 : half-height of TF coil inboard leg straight section (m)
  \item
    iohcl /1/ : switch for existence of central solenoid:

    \begin{itemize}
    \tightlist
    \item
      = 0 central solenoid not present;
    \item
      = 1 central solenoid exists
    \end{itemize}
  \item
    iprecomp /1/ : switch for existence of central solenoid
    pre-compression structure:

    \begin{itemize}
    \tightlist
    \item
      = 0 no pre-compression structure;
    \item
      = 1 calculated pre-compression structure
    \end{itemize}
  \item
    ohcth /0.811/ : central solenoid thickness (m) (iteration variable
    16)
  \item
    precomp : CS coil precompression structure thickness (m)
  \item
    rbld : sum of thicknesses to the major radius (m)
  \item
    rinboard /0.651/ : plasma inboard radius (m) (consistency equation
    29)
  \item
    rsldi : radius to inboard shield (inside point) (m)
  \item
    rsldo : radius to outboard shield (outside point) (m)
  \item
    rtfcin : radius of centre of inboard TF leg (m)
  \item
    rtot : radius to the centre of the outboard TF coil leg (m)
  \item
    scrapli /0.14/ : gap between plasma and first wall, inboard side (m)
    (used if iscrp=1) (iteration variable 73)
  \item
    scraplo /0.15/ : gap between plasma and first wall, outboard side
    (m) (used if iscrp=1) (iteration variable 74)
  \item
    sharea : shield total surface area (m2)
  \item
    shareaib : inboard shield surface area (m2)
  \item
    shareaob : outboard shield surface area (m2)
  \item
    shldith /0.69/ : inboard shield thickness (m) (iteration variable
    93)
  \item
    shldlth /0.7/ : lower (under divertor) shield thickness (m)
  \item
    shldoth /1.05/ : outboard shield thickness (m) (iteration variable
    94)
  \item
    shldtth /0.60/ : upper/lower shield thickness (m); calculated if
    blktmodel \textgreater{} 0
  \item
    sigallpc /3.0d8/ : allowable stress in CSpre-compression structure
    (Pa);
  \item
    tfcth /1.173/ : inboard TF coil thickness, (centrepost for ST) (m)
    (calculated for stellarators) (iteration variable 13)
  \item
    tfcth : inboard TF coil thickness (m)
    (calculated, NOT an iteration variable)
  \item
    tfoffset : vertical distance between centre of TF coils and centre
    of plasma (m)
  \item
    tfootfi /1.19/ : TF coil outboard leg / inboard leg radial thickness
    ratio (itfsup=0 only) (iteration variable 75)
  \item
    tfthko : outboard TF coil thickness (m)
  \item
    tftsgap /0.05/ : Minimum metal-to-metal gap between TF coil and
    thermal shield (m)
  \item
    thshield /0.05/ : TF-VV thermal shield thickness (m)
  \item
    vgap2 /0.163/ : vertical gap between vacuum vessel and thermal shield (m)
  \item
    vgap /0.0/ : vertical gap between x-point and divertor (m) (if = 0,
    it is calculated)
  \item
    vgaptop /0.60/ : vertical gap between top of plasma and first wall
    (m)
  \item
    vvblgap /0.05/ : gap between vacuum vessel and blanket (m)
  \item
    plleni /1.0/ : length of inboard divertor plate (m)
  \item
    plleno /1.0/ : length of outboard divertor plate (m)
  \item
    plsepi /1.0/ : poloidal length, x-point to inboard strike point (m)
  \item
    plsepo /1.5/ : poloidal length, x-point to outboard strike point (m)
  \item
    rspo : outboard strike point radius (m)
  \end{itemize}

  \subsubsection{\texorpdfstring{\href{cost_variables.html}{cost\_variables}}{cost\_variables}}\label{cost_variables}

  \begin{itemize}
  \tightlist
  \item
    abktflnc /5.0/ : allowable first wall/blanket neutron fluence
    (MW-yr/m2) (blktmodel=0)
  \item
    adivflnc /7.0/ : allowable divertor heat fluence (MW-yr/m2)
  \item
    blkcst : blanket direct cost (M\$)
  \item
    c221 : total account 221 cost (M\$) (first wall, blanket, shield,
    support structure and divertor plates)
  \item
    c222 : total account 222 cost (M\$) (TF coils + PF coils)
  \item
    capcost : total capital cost including interest (M\$)
  \item
    cconfix /80.0/ : fixed cost of superconducting cable (\$/m)
  \item
    cconshpf /70.0/ : cost of PF coil steel conduit/sheath (\$/m)
  \item
    cconshtf /75.0/ : cost of TF coil steel conduit/sheath (\$/m)
  \item
    cdcost : current drive direct costs (M\$)
  \item
    cdirt : total plant direct cost (M\$)
  \item
    cdrlife : lifetime of heating/current drive system (y)
  \item
    cfactr /0.75/ : Total plant availability fraction; input if iavail =
    0
  \item
    cpfact : Total plant capacity factor
  \item
    cfind(4) /0.244,0.244,0.244,0.29/ : indirect cost factor (func of
    lsa)
  \item
    cland /19.2/ : cost of land (M\$)
  \item
    coe : cost of electricity (\$/MW-hr)
  \item
    coecap : capital cost of electricity (m\$/kW-hr)
  \item
    coefuelt : `fuel' (including replaceable components) contribution to
    cost of electricity (m\$/kW-hr)
  \item
    coeoam : operation and maintenance contribution to cost of
    electricity (m\$/kW-hr)
  \item
    concost : plant construction cost (M\$)
  \item
    costexp /0.8/ : cost exponent for scaling in 2015 costs model
  \item
    costexp\_pebbles /0.6/ : cost exponent for pebbles in 2015 costs
    model
  \item
    cost\_factor\_buildings /1.0/ : cost scaling factor for buildings
  \item
    cost\_factor\_land /1.0/ : cost scaling factor for land
  \item
    cost\_factor\_tf\_coils /1.0/ : cost scaling factor for TF coils
  \item
    cost\_factor\_fwbs /1.0/ : cost scaling factor for fwbs
  \item
    cost\_factor\_rh /1.0/ : cost scaling factor for remote handling
  \item
    cost\_factor\_vv /1.0/ : cost scaling factor for vacuum vessel
  \item
    cost\_factor\_bop /1.0/ : cost scaling factor for energy conversion
    system
  \item
    cost\_factor\_misc /1.0/ : cost scaling factor for remaining
    subsystems
  \item
    maintenance\_fwbs /0.2/ : Maintenance cost factor: first wall,
    blanket, shield, divertor
  \item
    maintenance\_gen /0.05/ : Maintenance cost factor: All other
    components except coils, vacuum vessel, thermal shield, cryostat,
    land
  \item
    amortization /13.6/ : amortization factor (fixed charge factor)
    ``A'' (years)
  \item
    cost\_model /1/ : switch for cost model:

    \begin{itemize}
    \tightlist
    \item
      = 0 use \$ 1990 PROCESS model
    \item
      = 1 use \$ 2015 Kovari model
    \end{itemize}
  \item
    cowner /0.15/ : owner cost factor
  \item
    cplife : lifetime of centrepost (y)
  \item
    cpstcst : ST centrepost direct cost (M\$)
  \item
    cpstflnc /10.0/ : allowable ST centrepost neutron fluence (MW-yr/m2)
  \item
    crctcore : reactor core costs (categories 221, 222 and 223)
  \item
    csi /16.0/ : allowance for site costs (M\$)
  \item
    cturbb /380.0/ : cost of turbine building (M\$)
  \item
    decomf /0.1/ : proportion of constructed cost required for
    decommissioning fund
  \item
    dintrt /0.0/ : diff between borrowing and saving interest rates
  \item
    divcst : divertor direct cost (M\$)
  \item
    divlife : lifetime of divertor (y)
  \item
    dtlife /0.0/ : period prior to the end of the plant life that the
    decommissioning fund is used (years)
  \item
    fcap0 /1.165/ : average cost of money for construction of plant
    assuming design/construction time of six years
  \item
    fcap0cp /1.08/ : average cost of money for replaceable components
    assuming lead time for these of two years
  \item
    fcdfuel /0.1/ : fraction of current drive cost treated as fuel (if
    ifueltyp = 1)
  \item
    fcontng /0.195/ : project contingency factor
  \item
    fcr0 /0.0966/ : fixed charge rate during construction
  \item
    fkind /1.0/ : multiplier for Nth of a kind costs
  \item
    fwallcst : first wall cost (M\$)
  \item
    iavail /2/ : switch for plant availability model:

    \begin{itemize}
    \tightlist
    \item
      = 0 use input value for cfactr;
    \item
      = 1 calculate cfactr using Taylor and Ward 1999 model;
    \item
      = 2 calculate cfactr using new (2015) model
    \end{itemize}
  \item
    avail\_min /0.75/ : Minimum availability (constraint equation 61)
  \item
    tok\_build\_cost\_per\_vol /1283.0/ : Unit cost for tokamak complex
    buildings, including building and site services (\$/m3)
  \item
    light\_build\_cost\_per\_vol /270.0/ : Unit cost for unshielded
    non-active buildings (\$/m3)
  \item
    favail /1.0/ : F-value for minimum availability (constraint equation
    61)
  \item
    num\_rh\_systems /4/ : Number of remote handling systems (1-10)
  \item
    conf\_mag /0.99/ : c parameter, which determines the temperature
    margin at which magnet lifetime starts to d
  \item
    div\_prob\_fail /0.0002/ : Divertor probability of failure (per op
    day)
  \item
    div\_umain\_time /0.25/ : Divertor unplanned maintenance time
    (years)
  \item
    div\_nref /7000/ : Reference value for cycle cycle life of divertor
  \item
    div\_nu /14000/ : The cycle when the divertor fails with 100\%
    probability
  \item
    fwbs\_nref /20000/ : Reference value for cycle life of blanket
  \item
    fwbs\_nu /40000/ : The cycle when the blanket fails with 100\%
    probability
  \item
    fwbs\_prob\_fail /0.0002/ : Fwbs probability of failure (per op day)
  \item
    fwbs\_umain\_time /0.25/ : Fwbs unplanned maintenance time (years)
  \item
    redun\_vacp /25/ : Vacuum system pump redundancy level (\%)
  \item
    redun\_vac : Number of redundant vacuum pumps
  \item
    t\_operation : Operational time (yrs)
  \item
    tbktrepl /0.5/ : time taken to replace blanket (y) (iavail=1)
  \item
    tcomrepl /0.5/ : time taken to replace both blanket and divertor (y)
    (iavail=1)
  \item
    tdivrepl /0.25/ : time taken to replace divertor (y) (iavail=1)
  \item
    uubop /0.02/ : unplanned unavailability factor for balance of plant
    (iavail=1)
  \item
    uucd /0.02/ : unplanned unavailability factor for current drive
    (iavail=1)
  \item
    uudiv /0.04/ : unplanned unavailability factor for divertor
    (iavail=1)
  \item
    uufuel /0.02/ : unplanned unavailability factor for fuel system
    (iavail=1)
  \item
    uufw /0.04/ : unplanned unavailability factor for first wall
    (iavail=1)
  \item
    uumag /0.02/ : unplanned unavailability factor for magnets
    (iavail=1)
  \item
    uuves /0.04/ : unplanned unavailability factor for vessel (iavail=1)
  \item
    ifueltyp /0/ : switch:

    \begin{itemize}
    \tightlist
    \item
      = 1 treat blanket divertor, first wall and fraction fcdfuel of CD
      equipment as fuel cost;
    \item
      = 0 treat these as capital cost
    \end{itemize}
  \item
    ipnet /0/ : switch for net electric power calculation:

    \begin{itemize}
    \tightlist
    \item
      = 0 scale so that always \textgreater{} 0;
    \item
      = 1 let go \textless{} 0 (no c-o-e)
    \end{itemize}
  \item
    ireactor /1/ : switch for net electric power and cost of electricity
    calculations:

    \begin{itemize}
    \tightlist
    \item
      = 0 do not calculate MW(electric) or c-o-e;
    \item
      = 1 calculate MW(electric) and c-o-e
    \end{itemize}
  \item
    lsa /4/ : level of safety assurance switch (generally, use 3 or 4):

    \begin{itemize}
    \tightlist
    \item
      = 1 truly passively safe plant;
    \item
      = 2,3 in-between;
    \item
      = 4 like current fission plant
    \end{itemize}
  \item
    moneyint : interest portion of capital cost (M\$)
  \item
    output\_costs /1/ : switch for costs output:

    \begin{itemize}
    \tightlist
    \item
      = 0 do not write cost-related outputs to file;
    \item
      = 1 write cost-related outputs to file
    \end{itemize}
  \item
    ratecdol /0.0435/ : effective cost of money in constant dollars
  \item
    tlife /30.0/ : plant life (years)
  \item
    ucad /180.0/ FIX : unit cost for administration buildings (M\$/m3)
  \item
    ucaf /1.5e6/ FIX : unit cost for aux facility power equipment (\$)
  \item
    ucahts /31.0/ FIX : unit cost for aux heat transport equipment
    (\$/W**exphts)
  \item
    ucap /17.0/ FIX : unit cost of auxiliary transformer (\$/kVA)
  \item
    ucblbe /260.0/ : unit cost for blanket beryllium (\$/kg)
  \item
    ucblbreed /875.0/ : unit cost for breeder material (\$/kg)
    (blktmodel\textgreater{}0)
  \item
    ucblli /875.0/ : unit cost for blanket lithium (\$/kg) (30\% Li6)
  \item
    ucblli2o /600.0/ : unit cost for blanket Li\_2O (\$/kg)
  \item
    ucbllipb /10.3/ : unit cost for blanket Li-Pb (\$/kg) (30\% Li6)
  \item
    ucblss /90.0/ : unit cost for blanket stainless steel (\$/kg)
  \item
    ucblvd /200.0/ : unit cost for blanket vanadium (\$/kg)
  \item
    ucbpmp /2.925e5/ FIX : vacuum system backing pump cost (\$)
  \item
    ucbus /0.123/ : cost of aluminium bus for TF coil (\$/A-m)
  \item
    uccase /50.0/ : cost of superconductor case (\$/kg)
  \item
    ucco /350.0/ FIX : unit cost for control buildings (M\$/m3)
  \item
    uccpcl1 /250.0/ : cost of high strength tapered copper (\$/kg)
  \item
    uccpclb /150.0/ : cost of TF outboard leg plate coils (\$/kg)
  \item
    uccpmp /3.9e5/ FIX : vacuum system cryopump cost (\$)
  \item
    uccr /460.0/ FIX : unit cost for cryogenic building (M\$/vol)
  \item
    uccry /9.3e4/ : heat transport system cryoplant costs (\$/W**expcry)
  \item
    uccryo /32.0/ : unit cost for vacuum vessel (\$/kg)
  \item
    uccu /75.0/ : unit cost for copper in superconducting cable (\$/kg)
  \item
    ucdgen /1.7e6/ FIX : cost per 8 MW diesel generator (\$)
  \item
    ucdiv /2.8e5/ : cost of divertor blade (\$)
  \item
    ucdtc /13.0/ FIX : detritiation, air cleanup cost (\$/10000m3/hr)
  \item
    ucduct /4.225e4/ FIX : vacuum system duct cost (\$/m)
  \item
    ucech /3.0/ : ECH system cost (\$/W)
  \item
    ucel /380.0/ FIX : unit cost for electrical equipment building
    (M\$/m3)
  \item
    uces1 /3.2e4/ FIX : MGF (motor-generator flywheel) cost factor
    (\$/MVA**0.8)
  \item
    uces2 /8.8e3/ FIX : MGF (motor-generator flywheel) cost factor
    (\$/MJ**0.8)
  \item
    ucf1 /2.23e7/ : cost of fuelling system (\$)
  \item
    ucfnc /35.0/ : outer PF coil fence support cost (\$/kg)
  \item
    ucfpr /4.4e7/ FIX : cost of 60g/day tritium processing unit (\$)
  \item
    ucfuel /3.45/ : unit cost of D-T fuel (M\$/year/1200MW)
  \item
    ucfwa /6.0e4/ FIX : first wall armour cost (\$/m2)
  \item
    ucfwps /1.0e7/ FIX : first wall passive stabiliser cost (\$)
  \item
    ucfws /5.3e4/ FIX : first wall structure cost (\$/m2)
  \item
    ucgss /35.0/ FIX : cost of reactor structure (\$/kg)
  \item
    uche3 /1.0e6/ : cost of helium-3 (\$/kg)
  \item
    uchrs /87.9e6/ : cost of heat rejection system (\$)
  \item
    uchts(2) /15.3,19.1/ : cost of heat transport system equipment per
    loop (\$/W); dependent on coolant type (coolwh)
  \item
    uciac /1.5e8/ : cost of instrumentation, control \& diagnostics (\$)
  \item
    ucich /3.0/ : ICH system cost (\$/W)
  \item
    ucint /35.0/ FIX : superconductor intercoil structure cost (\$/kg)
  \item
    uclh /3.3/ : lower hybrid system cost (\$/W)
  \item
    uclv /16.0/ FIX : low voltage system cost (\$/kVA)
  \item
    ucmb /260.0/ FIX: unit cost for reactor maintenance building
    (M\$/m3)
  \item
    ucme /1.25e8/ : cost of maintenance equipment (\$/)
  \item
    ucmisc /2.5e7/ : miscellaneous plant allowance (\$)
  \item
    ucnbi /3.3/ : NBI system cost (\$/W)
  \item
    ucnbv /1000.0/ FIX : cost of nuclear building ventilation (\$/m3)
  \item
    ucoam(4) /68.8,68.8,68.8,74.4/ : annual cost of operation and
    maintenance (M\$/year/1200MW**0.5)
  \item
    ucpens /32.0/ : penetration shield cost (\$/kg)
  \item
    ucpfb /210.0/ : cost of PF coil buses (\$/kA/m)
  \item
    ucpfbk /1.66e4/ : cost of PF coil DC breakers (\$/MVA)
  \item
    ucpfbs /4.9e3/ : cost of PF burn power supplies (\$/kW**0.7)
  \item
    ucpfcb /7.5e4/ : cost of PF coil AC breakers (\$/circuit)
  \item
    ucpfdr1 /150.0/ : cost factor for dump resistors (\$/MJ)
  \item
    ucpfic /1.0e4/ : cost of PF instrumentation and control (\$/channel)
  \item
    ucpfps /3.5e4/ : cost of PF coil pulsed power supplies (\$/MVA)
  \item
    ucphx /15.0/ FIX : primary heat transport cost (\$/W**exphts)
  \item
    ucpp /48.0/ FIX : cost of primary power transformers (\$/kVA**0.9)
  \item
    ucrb /400.0/ : cost of reactor building (M\$/m3)
  \item
    ucsc(6) /600.0,600.0,300.0,600.0/ : cost of superconductor (\$/kg)
  \item
    ucsh /115.0/ FIX : cost of shops and warehouses (M\$/m3)
  \item
    ucshld /32.0/ : cost of shield structural steel (\$/kg)
  \item
    ucswyd /1.84e7/ FIX : switchyard equipment costs (\$)
  \item
    uctfbr /1.22/ : cost of TF coil breakers (\$/W**0.7)
  \item
    uctfbus /100.0/ : cost of TF coil bus (\$/kg)
  \item
    uctfdr /1.75e-4/ FIX : cost of TF coil dump resistors (\$/J)
  \item
    uctfgr /5000.0/ FIX : additional cost of TF coil dump resistors
    (\$/coil)
  \item
    uctfic /1.0e4/ FIX : cost of TF coil instrumentation and control
    (\$/coil/30)
  \item
    uctfps /24.0/ : cost of TF coil power supplies (\$/W**0.7)
  \item
    uctfsw /1.0/ : cost of TF coil slow dump switches (\$/A)
  \item
    uctpmp /1.105e5/ FIX : cost of turbomolecular pump (\$)
  \item
    uctr /370.0/ FIX : cost of tritium building (\$/m3)
  \item
    ucturb(2) /230.0e6, 245.0e6/: cost of turbine plant equipment (\$)
    (dependent on coolant type coolwh)
  \item
    ucvalv /3.9e5/ FIX : vacuum system valve cost (\$)
  \item
    ucvdsh /26.0/ FIX : vacuum duct shield cost (\$/kg)
  \item
    ucviac /1.3e6/ FIX : vacuum system instrumentation and control cost
    (\$)
  \item
    ucwindpf /465.0/ : cost of PF coil superconductor windings (\$/m)
  \item
    ucwindtf /480.0/ : cost of TF coil superconductor windings (\$/m)
  \item
    ucws /460.0/ FIX : cost of active assembly shop (\$/m3)
  \item
    ucwst(4) /0.0,3.94,5.91,7.88/ : cost of waste disposal
    (M\$/y/1200MW)
  \end{itemize}

  \subsubsection{\texorpdfstring{\href{constraint_variables.html}{constraint\_variables}}{constraint\_variables}}\label{constraint_variables}

  \begin{itemize}
  \tightlist
  \item
    auxmin /0.1/ : minimum auxiliary power (MW) (constraint equation 40)
  \item
    betpmx /0.19/ : maximum poloidal beta (constraint equation 48)
  \item
    bigqmin /10.0/ : minimum fusion gain Q (constraint equation 28)
  \item
    bmxlim /12.0/ : maximum peak toroidal field (T) (constraint equation
    25)
  \item
    fauxmn /1.0/ : f-value for minimum auxiliary power (constraint
    equation 40, iteration variable 64)
  \item
    fbeta /1.0/ : f-value for epsilon beta-poloidal (constraint equation
    6, iteration variable 8)
  \item
    fbetap /1.0/ : f-value for poloidal beta (constraint equation 48,
    iteration variable 79)
  \item
    fbetatry /1.0/ : f-value for beta limit (constraint equation 24,
    iteration variable 36)
  \item
    fcpttf /1.0/ : f-value for TF coil current per turn upper limit
    (constraint equation 77, iteration variable 146)
  \item
    fcwr /1.0/ : f-value for conducting wall radius / rminor limit
    (constraint equation 23, iteration variable 104)
  \item
    fdene /1.0/ : f-value for density limit (constraint equation 5,
    iteration variable 9) (invalid if ipedestal = 3)
  \item
    fdivcol /1.0/ : f-value for divertor collisionality (constraint
    equation 22, iteration variable 34)
  \item
    fdtmp /1.0/ : f-value for first wall coolant temperature rise
    (constraint equation 38, iteration variable 62)
  \item
    fflutf /1.0/ : f-value for neutron fluence on TF coil (constraint
    equation 53, iteration variable 92)
  \item
    ffuspow /1.0/ : f-value for maximum fusion power (constraint
    equation 9, iteration variable 26)
  \item
    fgamcd /1.0/ : f-value for current drive gamma (constraint equation
    37, iteration variable 40)
  \item
    fhldiv /1.0/ : f-value for divertor heat load (constraint equation
    18, iteration variable 27)
  \item
    fiooic /0.5/ : f-value for TF coil operating current / critical
    current ratio (constraint equation 33, iteration variable 50)
  \item
    fipir /1.0/ : f-value for Ip/Irod limit (constraint equation 46,
    iteration variable 72)
  \item
    fjohc /1.0/ : f-value for central solenoid current at end-of-flattop
    (constraint equation 26, iteration variable 38)
  \item
    fjohc0 /1.0/ : f-value for central solenoid current at beginning of
    pulse (constraint equation 27, iteration variable 39)
  \item
    fjprot /1.0/ : f-value for TF coil winding pack current density
    (constraint equation 35, iteration variable 53)
  \item
    flhthresh /1.0/ : f-value for L-H power threshold (constraint
    equation 15, iteration variable 103)
  \item
    fmva /1.0/ : f-value for maximum MVA (constraint equation 19,
    iteration variable 30)
  \item
    fnbshinef /1.0/ : f-value for maximum neutral beam shine-through
    fraction (constraint equation 59, iteration variable 105)
  \item
    fnesep /1.0/ : f-value for Eich critical separatrix density
    (constraint equation 76, iteration variable 144)
  \item
    foh\_stress /1.0/ : f-value for Tresca stress in Central Solenoid
    (constraint equation 72, iteration variable 123)
  \item
    fpeakb /1.0/ : f-value for maximum toroidal field (constraint
    equation 25, iteration variable 35)
  \item
    fpinj /1.0/ : f-value for injection power (constraint equation 30,
    iteration variable 46)
  \item
    fpnetel /1.0/ : f-value for net electric power (constraint equation
    16, iteration variable 25)
  \item
    fportsz /1.0/ : f-value for neutral beam tangency radius limit
    (constraint equation 20, iteration variable 33)
  \item
    fpsepbqar /1.0/ : f-value for maximum Psep*Bt/qAR limit (constraint
    equation 68, iteration variable 117)
  \item
    fpsepr /1.0/ : f-value for maximum Psep/R limit (constraint equation
    56, iteration variable 97)
  \item
    fptemp /1.0/ : f-value for peak centrepost temperature (constraint
    equation 44, iteration variable 68)
  \item
    fptfnuc /1.0/ : f-value for maximum TF coil nuclear heating
    (constraint equation 54, iteration variable 95)
  \item
    fq /1.0/ : f-value for edge safety factor (constraint equation 45,
    iteration variable 71)
  \item
    fqval /1.0/ : f-value for Q (constraint equation 28, iteration
    variable 45)
  \item
    fradpwr /1.0/ : f-value for core radiation power limit (constraint
    equation 17, iteration variable 28)
  \item
    fradwall /1.0/ : f-value for upper limit on radiation wall load
    (constr. equ. 67, iteration variable 116 )
  \item
    freinke /1.0/ : f-value for Reinke detachment criterion (constr.
    equ. 78, iteration variable 147)
  \item
    frminor /1.0/ : f-value for minor radius limit (constraint equation
    21, iteration variable 32)
  \item
    fstrcase /1.0/ : f-value for TF coil case stress (constraint
    equation 31, iteration variable 48)
  \item
    fstrcond /1.0/ : f-value for TF coil conduit stress (constraint
    equation 32, iteration variable 49)
  \item
    ftaucq /1.0/ : f-value for calculated minimum TF quench time
    (constraint equation 65, iteration variable 113)
  \item
    ftbr /1.0/ : f-value for minimum tritium breeding ratio (constraint
    equation 52, iteration variable 89)
  \item
    ftburn /1.0/ : f-value for minimum burn time (constraint equation
    13, iteration variable 21)
  \item
    ftcycl /1.0/ : f-value for cycle time (constraint equation 42,
    iteration variable 67)
  \item
    ftmargoh /1.0/ : f-value for central solenoid temperature margin
    (constraint equation 60, iteration variable 106)
  \item
    ftmargtf /1.0/ : f-value for TF coil temperature margin (constraint
    equation 36, iteration variable 54)
  \item
    ftohs /1.0/ : f-value for plasma current ramp-up time (constraint
    equation 41, iteration variable 66)
  \item
    ftpeak /1.0/ : f-value for first wall peak temperature (constraint
    equation 39, iteration variable 63)
  \item
    fvdump /1.0/ : f-value for dump voltage (constraint equation 34,
    iteration variable 51)
  \item
    fvs /1.0/ : f-value for flux-swing (V-s) requirement (STEADY STATE)
    (constraint equation 12, iteration variable 15)
  \item
    fvvhe /1.0/ : f-value for vacuum vessel He concentration limit
    (iblanket = 2) (constraint equation 55, iteration variable 96)
  \item
    fwalld /1.0/ : f-value for maximum wall load (constraint equation 8,
    iteration variable 14)
  \item
    fzeffmax /1.0/ : f-value for maximum zeff (constraint equation 64,
    iteration variable 112)
  \item
    gammax /2.0/ : maximum current drive gamma (constraint equation 37)
  \item
    maxradwallload /1.0/ : Maximum permitted radiation wall load
    (MW/m\^{}2) (constraint equation 67)
  \item
    mvalim /40.0/ : maximum MVA limit (constraint equation 19)
  \item
    nbshinefmax /1.0e-3/ : maximum neutral beam shine-through fraction
    (constraint equation 59)
  \item
    nflutfmax /1.0e23/ : max fast neutron fluence on TF coil (n/m2)
    (blktmodel\textgreater{}0) (constraint equation 53)
  \item
    peakfactrad /3.33/ : peaking factor for radiation wall load
    (constraint equation 67)
  \item
    peakradwallload : Peak radiation wall load (MW/m\^{}2) (constraint
    equation 67)
  \item
    pnetelin /1000.0/ : required net electric power (MW) (constraint
    equation 16)
  \item
    powfmax /1500.0/ : maximum fusion power (MW) (constraint equation 9)
  \item
    psepbqarmax /9.5/ : maximum ratio of Psep*Bt/qAR (MWT/m) (constraint
    equation 68)
  \item
    pseprmax /25.0/ : maximum ratio of power crossing the separatrix to
    plasma major radius (Psep/R) (MW/m) (constraint equation 56)
  \item
    ptfnucmax /1.0e-3/ : maximum nuclear heating in TF coil (MW/m3)
    (constraint equation 54)
  \item
    tbrmin /1.1/ : minimum tritium breeding ratio (constraint equation
    52)
  \item
    tbrnmn /1.0/ : minimum burn time (s) (KE - no longer itv., see issue
    706)
  \item
    tcycmn : minimum cycle time (s) (constraint equation 42)
  \item
    tohsmn : minimum plasma current ramp-up time (s) (constraint
    equation 41)
  \item
    vvhealw /1.0/ : allowed maximum helium concentration in vacuum
    vessel at end of plant life (appm) (iblanket =2) (constraint
    equation 55)
  \item
    walalw /1.0/ : allowable wall-load (MW/m2) (constraint equation 8)
  \item
    taulimit /5.0/ : Lower limit on taup/taueff the ratio of alpha
    particle to energy confinement times (constraint equation 62)
  \item
    ftaulimit /1.0/ : f-value for lower limit on taup/taueff the ratio
    of alpha particle to energy confinement times (constraint equation
    62, iteration variable 110)
  \item
    fniterpump /1.0/ : f-value for constraint that number of pumps
    \textless{} tfno (constraint equation 63, iteration variable 111)
  \item
    zeffmax /3.6/ : maximum value for Zeff (constraint equation 64)
  \item
    fpoloidalpower /1.0/ : f-value for constraint on rate of change of
    energy in poloidal field (constraint equation 66, iteration variable
    115)
  \item
    fpsep /1.0/ : f-value to ensure separatrix power is less than value
    from Kallenbach divertor (Not required as constraint 69 is an
    equality)
  \item
    fcqt /1.0/ : f-value: TF coil quench temparature remains below
    tmax\_croco (constraint equation 74, iteration variable 141)
  \end{itemize}

  \subsubsection{\texorpdfstring{\href{stellarator_variables.html}{stellarator\_variables}}{stellarator\_variables}}\label{stellarator_variables}

  \begin{itemize}
  \tightlist
  \item
    istell /0/ : switch for stellarator option (set via
    \texttt{device.dat}):

    \begin{itemize}
    \tightlist
    \item
      = 0 use tokamak model;
    \item
      = 1 use stellarator model
    \end{itemize}
  \item
    bmn /0.001/ : relative radial field perturbation
  \item
    f\_asym /1.0/ : divertor heat load peaking factor
  \item
    f\_rad /0.85/ : radiated power fraction in SOL
  \item
    f\_w /0.5/ : island size fraction factor
  \item
    fdivwet /0.3333/ : wetted fraction of the divertor area
  \item
    flpitch /0.001/ : field line pitch (rad)
  \item
    hportamax : maximum available area for horizontal ports (m2)
  \item
    hportpmax : maximum available poloidal extent for horizontal ports
    (m)
  \item
    hporttmax : maximum available toroidal extent for horizontal ports
    (m)
  \item
    iotabar /1.0/ : rotational transform (reciprocal of tokamak q) for
    stellarator confinement time scaling laws
  \item
    isthtr /3/ : switch for stellarator auxiliary heating method:

    \begin{itemize}
    \tightlist
    \item
      = 1 electron cyclotron resonance heating;
    \item
      = 2 lower hybrid heating;
    \item
      = 3 neutral beam injection
    \end{itemize}
  \item
    m\_res /5/ : poloidal resonance number
  \item
    n\_res /5/ : toroidal resonance number
  \item
    shear /0.5/ : magnetic shear, derivative of iotabar
  \item
    vmec\_info\_file /vmec\_info.dat/ : file containing general VMEC
    settings
  \item
    vmec\_rmn\_file /vmec\_Rmn.dat/ : file containing plasma boundary
    R(m,n) Fourier components
  \item
    vmec\_zmn\_file /vmec\_Zmn.dat/ : file containing plasma boundary
    Z(m,n) Fourier components
  \item
    vportamax : maximum available area for vertical ports (m2)
  \item
    vportpmax : maximum available poloidal extent for vertical ports (m)
  \item
    vporttmax : maximum available toroidal extent for vertical ports (m)
  \end{itemize}

  \subsubsection{\texorpdfstring{\href{pulse_variables.html}{pulse\_variables}}{pulse\_variables}}\label{pulse_variables}

  \begin{itemize}
  \tightlist
  \item
    bctmp /320.0/ : first wall bulk coolant temperature (C)
  \item
    bfw : outer radius of each first wall structural tube (m) (0.5 *
    average of fwith and fwoth)
  \item
    dtstor /300.0/ : maximum allowable temperature change in stainless
    steel thermal storage block (K) (istore=3)
  \item
    istore /1/ : switch for thermal storage method:

    \begin{itemize}
    \tightlist
    \item
      = 1 option 1 of Electrowatt report, AEA FUS 205;
    \item
      = 2 option 2 of Electrowatt report, AEA FUS 205;
    \item
      = 3 stainless steel block
    \end{itemize}
  \item
    itcycl /1/ : switch for first wall axial stress model:

    \begin{itemize}
    \tightlist
    \item
      = 1 total axial constraint, no bending;
    \item
      = 2 no axial constraint, no bending;
    \item
      = 3 no axial constraint, bending
    \end{itemize}
  \item
    lpulse /0/ : switch for reactor model:

    \begin{itemize}
    \tightlist
    \item
      = 0 continuous operation;
    \item
      = 1 pulsed operation
    \end{itemize}
  \end{itemize}

  \subsubsection{\texorpdfstring{\href{startup_variables.html}{startup\_variables}}{startup\_variables}}\label{startup_variables}

  \begin{itemize}
  \tightlist
  \item
    ftaue : factor in energy confinement time formula
  \item
    gtaue : offset term in energy confinement time scaling
  \item
    nign : electron density at ignition (start-up) (/m3)
  \item
    ptaue : exponent for density term in energy confinement time formula
  \item
    qtaue : exponent for temperature term in energy confinement time
    formula
  \item
    rtaue : exponent for power term in energy confinement time formula
  \item
    tign : electron temperature at ignition (start-up) (keV)
  \end{itemize}

  \subsubsection{\texorpdfstring{\href{fispact_variables.html}{fispact\_variables}}{fispact\_variables}}\label{fispact_variables}

  \begin{itemize}
  \tightlist
  \item
    Fispact arrays with 3 elements contain the results at the following
    times: (1) - at end of component life (2) - after 3 months cooling
    time (3) - 100 years after end of plant life
  \item
    bliact(3) : inboard blanket total activity (Bq)
  \item
    bligdr(3) : inboard blanket total gamma dose rate (Sv/hr)
  \item
    blihkw(3) : inboard blanket total heat output (kW)
  \item
    bliizp : inboard blanket integrated zone power / neutron
  \item
    blimzp : inboard blanket mean zone power density / neutron
  \item
    bloact(3) : outboard blanket total activity (Bq)
  \item
    blogdr(3) : outboard blanket total gamma dose rate (Sv/hr)
  \item
    blohkw(3) : outboard blanket total heat output (kW)
  \item
    bloizp : outboard blanket integrated zone power / neutron
  \item
    blomzp : outboard blanket mean zone power density / neutron
  \item
    fwiact(3) : inboard first wall total activity (Bq)
  \item
    fwigdr(3) : inboard first wall total gamma dose rate (Sv/hr)
  \item
    fwihkw(3) : inboard first wall total heat output (kW)
  \item
    fwiizp : inboard first wall integrated zone power / neutron
  \item
    fwimzp : inboard first wall mean zone power density/neutron
  \item
    fwoact(3) : outboard first wall total activity (Bq)
  \item
    fwogdr(3) : outboard first wall total gamma dose rate (Sv/hr)
  \item
    fwohkw(3) : outboard first wall total heat output (kW)
  \item
    fwoizp : outboard first wall integrated zone power / neutron
  \item
    fwomzp : outboard first wall mean zone power density/neutron
  \item
    fwtemp : outboard first wall temperature after a LOCA (K)
  \end{itemize}

  \subsubsection{\texorpdfstring{\href{rebco_variables.html}{rebco\_variables}}{rebco\_variables}}\label{rebco_variables}

  \begin{itemize}
  \tightlist
  \item
    rebco\_thickness /1.0e-6/ : thickness of REBCO layer in tape (m)
    (iteration variable 138)
  \item
    copper\_thick /100e-6/ : thickness of copper layer in tape (m)
    (iteration variable 139)
  \item
    hastelloy\_thickness /50/e-6 : thickness of Hastelloy layer in tape
    (m)
  \item
    tape\_width /3.75e-3/ : Mean width of tape (m)
  \item
    croco\_od /10.4e-3/ : Outer diameter of CroCo strand (m)
  \item
    croco\_id /5.4e-3/ : Inner diameter of CroCo copper tube (m)
  \item
    copper\_bar /1.0/ : area of central copper bar, as a fraction of the
    cable space
  \item
    copper\_rrr /100.0/ : residual resistivity ratio copper in TF
    superconducting cable
  \item
    cable\_helium\_fraction /0.284/ : Helium area as a fraction of the
    cable space.
  \item
    copperA\_m2\_max /1e8/ : Maximum TF coil current / copper area
    (A/m2)
  \item
    f\_copperA\_m2 /1/ : f-value for constraint 75: TF coil current /
    copper area \textless{} copperA\_m2\_max
  \end{itemize}

  \subsubsection{\texorpdfstring{\href{resistive_material.html}{resistive\_material}}{resistive\_material}}\label{resistive_material}

  \subsubsection{\texorpdfstring{\href{reinke_variables.html}{reinke\_variables}}{reinke\_variables}}\label{reinke_variables}

  \begin{itemize}
  \tightlist
  \item
    impvardiv /9/ : index of impurity to be iterated for Reinke divertor
    detachment criterion
  \item
    lhat /4.33/ : connection length factor L\textbar{}\textbar{} = lhat
    qstar R for Reinke criterion, default value from Post et al. 1995 J.
    Nucl. Mat. 220-2 1014
  \item
    fzmin : Minimum impurity fraction necessary for detachment This is
    the impurity at the SOL/Div
  \item
    fzactual : Actual impurity fraction of divertor impurity (impvardiv)
    in the SoL (taking impurity\_enrichment into account) (iteration
    variable 148)
  \item
    reinke\_mode /0/ : Switch for Reinke criterion H/I mode
  \item
    = 0 H-mode;
  \item
    = 1 I-mode;
  \end{itemize}
\end{itemize}

\subsubsection{\texorpdfstring{\href{numerics.html}{numerics}}{numerics}}\label{numerics}

\begin{itemize}
\tightlist
\item
  ipnvars FIX : total number of variables available for iteration
\item
  ipeqns FIX : number of constraint equations available
\item
  ipnfoms FIX : number of available figures of merit
\item
  ioptimz /1/ : code operation switch:

  \begin{itemize}
  \tightlist
  \item
    = -1 for no optimisation, HYBRD only;
  \item
    = 0 for HYBRD and VMCON (not recommended);
  \item
    = 1 for optimisation, VMCON only
  \end{itemize}
\item
  minmax /7/ : switch for figure-of-merit (see lablmm for descriptions)
  negative =\textgreater{} maximise, positive =\textgreater{} minimise
\item
  lablmm(ipnfoms) : labels describing figures of merit:

  \begin{itemize}
  \tightlist
  \item
    ( 1) major radius
  \item
    ( 2) not used
  \item
    ( 3) neutron wall load
  \item
    ( 4) P\_tf + P\_pf
  \item
    ( 5) fusion gain Q
  \item
    ( 6) cost of electricity
  \item
    ( 7) capital cost (direct cost if ireactor=0, constructed cost
    otherwise)
  \item
    ( 8) aspect ratio
  \item
    ( 9) divertor heat load
  \item
    (10) toroidal field
  \item
    (11) total injected power
  \item
    (12) hydrogen plant capital cost OBSOLETE
  \item
    (13) hydrogen production rate OBSOLETE
  \item
    (14) pulse length
  \item
    (15) plant availability factor (N.B. requires iavail=1 to be set)
  \item
    (16) linear combination of major radius (minimised) and pulse length
    (maximised) note: FoM should be minimised only!
  \item
    (17) net electrical output
  \item
    (18) Null Figure of Merit
  \item
    (19) linear combination of big Q and pulse length (maximised) note:
    FoM should be minimised only!
  \end{itemize}
\item
  ncalls : number of function calls during solution
\item
  neqns /0/ : number of equality constraints to be satisfied
\item
  nfev1 : number of calls to FCNHYB (HYBRD function caller) made
\item
  nfev2 : number of calls to FCNVMC1 (VMCON function caller) made
\item
  nineqns /0/ : number of inequality constraints VMCON must satisfy
  (leave at zero for now)
\item
  nvar /16/ : number of iteration variables to use
\item
  nviter : number of VMCON iterations performed
\item
  icc(ipeqns) /0/ : array defining which constraint equations to
  activate (see lablcc for descriptions)
\item
  active\_constraints(ipeqns) : Logical array showing which constraints
  are active
\item
  lablcc(ipeqns) : labels describing constraint equations (corresponding
  itvs)

  \begin{itemize}
  \tightlist
  \item
    ( 1) Beta (consistency equation) (itv 5)
  \item
    ( 2) Global power balance (consistency equation) (itv
    10,1,2,3,4,6,11)
  \item
    ( 3) Ion power balance DEPRECATED (itv 10,1,2,3,4,6,11)
  \item
    ( 4) Electron power balance DEPRECATED (itv 10,1,2,3,4,6,11)
  \item
    ( 5) Density upper limit (itv 9,1,2,3,4,5,6)
  \item
    ( 6) (Epsilon x beta poloidal) upper limit (itv 8,1,2,3,4,6)
  \item
    ( 7) Beam ion density (NBI) (consistency equation) (itv 7)
  \item
    ( 8) Neutron wall load upper limit (itv 14,1,2,3,4,6)
  \item
    ( 9) Fusion power upper limit (itv 26,1,2,3,4,6)
  \item
    (10) Toroidal field 1/R (consistency equation) (itv 12,1,2,3,13 )
  \item
    (11) Radial build (consistency equation) (itv 3,1,13,16,29,42,61)
  \item
    (12) Volt second lower limit (STEADY STATE) (itv 15,1,2,3)
  \item
    (13) Burn time lower limit (PULSE) (itv 21,1,16,17,29,42,44,61)
  \item
    (14) Neutral beam decay lengths to plasma centre (NBI) (consistency
    equation)
  \item
    (15) LH power threshold limit (itv 103)
  \item
    (16) Net electric power lower limit (itv 25,1,2,3)
  \item
    (17) Radiation fraction upper limit (itv 28)
  \item
    (18) Divertor heat load upper limit (itv 27)
  \item
    (19) MVA upper limit (itv 30)
  \item
    (20) Neutral beam tangency radius upper limit (NBI) (itv 33,31,3,13)
  \item
    (21) Plasma minor radius lower limit (itv 32)
  \item
    (22) Divertor collisionality upper limit (itv 34,43)
  \item
    (23) Conducting shell to plasma minor radius ratio upper limit (itv
    104,1,74)
  \item
    (24) Beta upper limit (itv 36,1,2,3,4,6,18)
  \item
    (25) Peak toroidal field upper limit (itv 35,3,13,29)
  \item
    (26) Central solenoid EOF current density upper limit (ipfres=0)
    (itv 38,37,41,12)
  \item
    (27) Central solenoid BOP current density upper limit (ipfres=0)
    (itv 39,37,41,12)
  \item
    (28) Fusion gain Q lower limit (itv 45,47,40)
  \item
    (29) Inboard radial build consistency (itv 3,1,13,16,29,42,61)
  \item
    (30) Injection power upper limit (itv 46,47,11)
  \item
    (31) TF coil case stress upper limit (SCTF) (itv
    48,56,57,58,59,60,24)
  \item
    (32) TF coil conduit stress upper limit (SCTF) (itv
    49,56,57,58,59,60,24)
  \item
    (33) I\_op / I\_critical (TF coil) (SCTF) (itv 50,56,57,58,59,60,24)
  \item
    (34) Dump voltage upper limit (SCTF) (itv 51,52,56,57,58,59,60,24)
  \item
    (35) J\_winding pack/J\_protection upper limit (SCTF) (itv
    53,56,57,58,59,60,24)
  \item
    (36) TF coil temperature margin lower limit (SCTF) (itv
    54,55,56,57,58,59,60,24)
  \item
    (37) Current drive gamma upper limit (itv 40,47)
  \item
    (38) First wall coolant temperature rise upper limit (itv 62)
  \item
    (39) First wall peak temperature upper limit (itv 63)
  \item
    (40) Start-up injection power lower limit (PULSE) (itv 64)
  \item
    (41) Plasma current ramp-up time lower limit (PULSE) (itv 66,65)
  \item
    (42) Cycle time lower limit (PULSE) (itv 67,65,17)
  \item
    (43) Average centrepost temperature (TART) (consistency equation)
    (itv 69,70,13)
  \item
    (44) Peak centrepost temperature upper limit (TART) (itv 68,69,70)
  \item
    (45) Edge safety factor lower limit (TART) (itv 71,1,2,3)
  \item
    (46) Ip/Irod upper limit (TART) (itv 72,2,60)
  \item
    (47) NOT USED
  \item
    (48) Poloidal beta upper limit (itv 79,2,3,18)
  \item
    (49) NOT USED
  \item
    (50) NOT USED
  \item
    (51) Startup volt-seconds consistency (PULSE) (itv 16,29,3,1)
  \item
    (52) Tritium breeding ratio lower limit (itv 89,90,91)
  \item
    (53) Neutron fluence on TF coil upper limit (itv 92,93,94)
  \item
    (54) Peak TF coil nuclear heating upper limit (itv 95,93,94)
  \item
    (55) Vacuum vessel helium concentration upper limit iblanket =2 (itv
    96,93,94)
  \item
    (56) Pseparatrix/Rmajor upper limit (itv 97,1,3,102)
  \item
    (57) NOT USED
  \item
    (58) NOT USED
  \item
    (59) Neutral beam shine-through fraction upper limit (NBI) (itv
    105,6,19,4 )
  \item
    (60) Central solenoid temperature margin lower limit (SCTF) (itv
    106)
  \item
    (61) Minimum availability value (itv 107)
  \item
    (62) taup/taueff the ratio of particle to energy confinement times
    (itv 110)
  \item
    (63) The number of ITER-like vacuum pumps niterpump \textless{} tfno
    (itv 111)
  \item
    (64) Zeff less than or equal to zeffmax (itv 112)
  \item
    (65) Dump time set by VV loads (itv 56, 113)
  \item
    (66) Limit on rate of change of energy in poloidal field (Use
    iteration variable 65(tohs), 115)
  \item
    (67) Simple Radiation Wall load limit (itv 116, 102, 4,6)
  \item
    (68) Psep * Bt / qAR upper limit (itv 117)
  \item
    (69) ensure separatrix power = the value from Kallenbach divertor
    (itv 118)
  \item
    (70) ensure that teomp = separatrix temperature in the pedestal
    profile, (itv 119 (tesep))
  \item
    (71) ensure that neomp = separatrix density (nesep) x neratio
  \item
    (72) central solenoid Tresca stress limit (itv 123 foh\_stress)
  \item
    (73) Psep \textgreater{}= Plh + Paux (itv 137 (fplhsep))
  \item
    (74) TFC quench \textless{} tmax\_croco (itv 141 (fcqt))
  \item
    (75) TFC current/copper area \textless{} Maximum (itv 143
    f\_copperA\_m2)
  \item
    (76) Eich critical separatrix density
  \item
    (77) TF coil current per turn upper limit
  \item
    (78) Reinke criterion impurity fraction lower limit (itv 147
    freinke)
  \item
    (79) F-value for max peak CS field (itv 149 fbmaxcs)
  \end{itemize}
\item
  ixc(ipnvars) /0/ : array defining which iteration variables to
  activate (see lablxc for descriptions)
\item
  lablxc(ipnvars) : labels describing iteration variables (NEW:THERE ARE
  NO DEFAULTS):

  \begin{itemize}
  \tightlist
  \item
    ( 1) aspect
  \item
    ( 2) bt
  \item
    ( 3) rmajor
  \item
    ( 4) te
  \item
    ( 5) beta
  \item
    ( 6) dene
  \item
    ( 7) rnbeam
  \item
    ( 8) fbeta (f-value for equation 6)
  \item
    ( 9) fdene (f-value for equation 5)
  \item
    (10) hfact
  \item
    (11) pheat
  \item
    (12) oacdcp
  \item
    (13) tfcth (NOT RECOMMENDED)
  \item
    (14) fwalld (f-value for equation 8)
  \item
    (15) fvs (f-value for equation 12)
  \item
    (16) ohcth
  \item
    (17) tdwell
  \item
    (18) q
  \item
    (19) enbeam
  \item
    (20) tcpav
  \item
    (21) ftburn (f-value for equation 13)
  \item
    (22) NOT USED
  \item
    (23) fcoolcp
  \item
    (24) NOT USED
  \item
    (25) fpnetel (f-value for equation 16)
  \item
    (26) ffuspow (f-value for equation 9)
  \item
    (27) fhldiv (f-value for equation 18)
  \item
    (28) fradpwr (f-value for equation 17), total radiation fraction
  \item
    (29) bore
  \item
    (30) fmva (f-value for equation 19)
  \item
    (31) gapomin
  \item
    (32) frminor (f-value for equation 21)
  \item
    (33) fportsz (f-value for equation 20)
  \item
    (34) fdivcol (f-value for equation 22)
  \item
    (35) fpeakb (f-value for equation 25)
  \item
    (36) fbetatry (f-value for equation 24)
  \item
    (37) coheof
  \item
    (38) fjohc (f-value for equation 26)
  \item
    (39) fjohc0 (f-value for equation 27)
  \item
    (40) fgamcd (f-value for equation 37)
  \item
    (41) fcohbop
  \item
    (42) gapoh
  \item
    (43) cfe0
  \item
    (44) fvsbrnni
  \item
    (45) fqval (f-value for equation 28)
  \item
    (46) fpinj (f-value for equation 30)
  \item
    (47) feffcd
  \item
    (48) fstrcase (f-value for equation 31)
  \item
    (49) fstrcond (f-value for equation 32)
  \item
    (50) fiooic (f-value for equation 33)
  \item
    (51) fvdump (f-value for equation 34)
  \item
    (52) vdalw
  \item
    (53) fjprot (f-value for equation 35)
  \item
    (54) ftmargtf (f-value for equation 36)
  \item
    (55) obsolete
  \item
    (56) tdmptf
  \item
    (57) thkcas
  \item
    (58) thwcndut
  \item
    (59) fcutfsu
  \item
    (60) cpttf
  \item
    (61) gapds
  \item
    (62) fdtmp (f-value for equation 38)
  \item
    (63) ftpeak (f-value for equation 39)
  \item
    (64) fauxmn (f-value for equation 40)
  \item
    (65) tohs
  \item
    (66) ftohs (f-value for equation 41)
  \item
    (67) ftcycl (f-value for equation 42)
  \item
    (68) fptemp (f-value for equation 44)
  \item
    (69) rcool
  \item
    (70) vcool
  \item
    (71) fq (f-value for equation 45)
  \item
    (72) fipir (f-value for equation 46)
  \item
    (73) scrapli
  \item
    (74) scraplo
  \item
    (75) tfootfi
  \item
    (76) NOT USED
  \item
    (77) NOT USED
  \item
    (78) NOT USED
  \item
    (79) fbetap (f-value for equation 48)
  \item
    (80) NOT USED
  \item
    (81) NOT USED
  \item
    (82) NOT USED
  \item
    (83) NOT USED
  \item
    (84) NOT USED
  \item
    (85) NOT USED
  \item
    (86) NOT USED
  \item
    (87) NOT USED
  \item
    (88) NOT USED
  \item
    (89) ftbr (f-value for equation 52)
  \item
    (90) blbuith
  \item
    (91) blbuoth
  \item
    (92) fflutf (f-value for equation 53)
  \item
    (93) shldith
  \item
    (94) shldoth
  \item
    (95) fptfnuc (f-value for equation 54)
  \item
    (96) fvvhe (f-value for equation 55)
  \item
    (97) fpsepr (f-value for equation 56)
  \item
    (98) li6enrich
  \item
    (99) NOT USED
  \item
    (100) NOT USED
  \item
    (101) NOT USED
  \item
    (102) fimpvar
  \item
    (103) flhthresh (f-value for equation 15)
  \item
    (104) fcwr (f-value for equation 23)
  \item
    (105) fnbshinef (f-value for equation 59)
  \item
    (106) ftmargoh (f-value for equation 60)
  \item
    (107) favail (f-value for equation 61)
  \item
    (108) breeder\_f: Volume of Li4SiO4 / (Volume of Be12Ti + Li4SiO4)
  \item
    (109) ralpne: thermal alpha density / electron density
  \item
    (110) ftaulimit: Lower limit on taup/taueff the ratio of alpha
    particle to energy confinement times (f-value for equation 62)
  \item
    (111) fniterpump: f-value for constraint that number of vacuum pumps
    \textless{} TF coils (f-value for equation 63)
  \item
    (112) fzeffmax: f-value for max Zeff (f-value for equation 64)
  \item
    (113) ftaucq: f-value for minimum quench time (f-value for equation
    65)
  \item
    (114) fw\_channel\_length: Length of a single first wall channel
  \item
    (115) fpoloidalpower: f-value for max rate of change of energy in
    poloidal field (f-value for equation 66)
  \item
    (116) fradwall: f-value for radiation wall load limit (eq. 67)
  \item
    (117) fpsepbqar: f-value for Psep*Bt/qar upper limit (eq. 68)
  \item
    (118) fpsep: f-value to ensure separatrix power is less than value
    from Kallenbach divertor (f-value for equation 69)
  \item
    (119) tesep: separatrix temperature calculated by the Kallenbach
    divertor model
  \item
    (120) ttarget: Plasma temperature adjacent to divertor sheath
    {[}eV{]}
  \item
    (121) neratio: ratio of mean SOL density at OMP to separatrix
    density at OMP
  \item
    (122) oh\_steel\_frac : streel fraction of Central Solenoid
  \item
    (123) foh\_stress : f-value for CS coil Tresca stress limit (f-value
    for eq. 72)
  \item
    (124) qtargettotal : Power density on target including surface
    recombination {[}W/m2{]}
  \item
    (125) fimp(3) : Beryllium density fraction relative to electron
    density
  \item
    (126) fimp(4) : Carbon density fraction relative to electron density
  \item
    (127) fimp(5) : Nitrogen fraction relative to electron density
  \item
    (128) fimp(6) : Oxygen density fraction relative to electron density
  \item
    (129) fimp(7) : Neon density fraction relative to electron density
  \item
    (130) fimp(8) : Silicon density fraction relative to electron
    density
  \item
    (131) fimp(9) : Argon density fraction relative to electron density
  \item
    (132) fimp(10) : Iron density fraction relative to electron density
  \item
    (133) fimp(11) : Nickel density fraction relative to electron
    density
  \item
    (134) fimp(12) : Krypton density fraction relative to electron
    density
  \item
    (135) fimp(13) : Xenon density fraction relative to electron density
  \item
    (136) fimp(14) : Tungsten density fraction relative to electron
    density
  \item
    (137) fplhsep (f-value for equation 73)
  \item
    (138) rebco\_thickness : thickness of REBCO layer in tape (m)
  \item
    (139) copper\_thick : thickness of copper layer in tape (m)
  \item
    (140) thkwp : radial thickness of TFC winding pack (m)
  \item
    (141) fcqt : TF coil quench temperature \textless{} tmax\_croco
    (f-value for equation 74)
  \item
    (142) nesep : electron density at separatrix {[}m-3{]}
  \item
    (143) f\_copperA\_m2 : TF coil current / copper area \textless{}
    Maximum value (f-value for equation 75)
  \item
    (144) fnesep : Eich critical electron density at separatrix (f-value
    for constraint equation 76)
  \item
    (145) fgwped : fraction of Greenwald density to set as pedestal-top
    density
  \item
    (146) fcpttf : F-value for TF coil current per turn limit
    (constraint equation 77)
  \item
    (147) freinke : F-value for Reinke detachment criterion (constraint
    equation 78)
  \item
    (148) fzactual : fraction of impurity at SOL with Reinke detachment
    criterion
  \item
    (149) fbmaxcs : F-value for max peak CS field (con. 79, itvar 149)
  \item
    (150) plasmod\_fcdp : (P\_CD - Pheat)/(Pmax-Pheat),i.e. ratio of CD
    power over available power
  \item
    (151) plasmod\_fradc : Pline\_Xe / (Palpha + Paux - PlineAr - Psync
    - Pbrad)
  \item
    (152) fbmaxcs : Ratio of separatrix density to Greenwald density
  \end{itemize}
\item
  sqsumsq : sqrt of the sum of the square of the constraint residuals
\item
  epsfcn /1.0e-3/ : finite difference step length for HYBRD/VMCON
  derivatives
\item
  epsvmc /1.0e-6/ : error tolerance for VMCON
\item
  factor /0.1/ : used in HYBRD for first step size
\item
  ftol /1.0e-4/ : error tolerance for HYBRD
\item
  boundl(ipnvars) /../ : lower bounds used on ixc variables during VMCON
  optimisation runs
\item
  boundu(ipnvars) /../ : upper bounds used on ixc variables during VMCON
  optimisation runs
\end{itemize}

\subsubsection{\texorpdfstring{\href{eqsolv.html}{eqsolv}}{eqsolv}}\label{eqsolv}

\subsubsection{\texorpdfstring{\href{optimiz.html}{optimiz}}{optimiz}}\label{optimiz}

\end{document}
