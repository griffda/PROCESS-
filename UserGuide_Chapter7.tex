\mychapter{Code Management Tools}
\label{chap:codetools}

This chapter will be of interest to people involved in the continuing
maintenance of the \process\ source code. As stated elsewhere, the source code
is maintained by CCFE, and resides in a \textit{Git}\
repository on the CCFE servers.

\section{The Makefile}
\label{sec:makefile}

In addition to its normal role for compilation, the makefile (named
\texttt{Makefile}) includes a number of utility functions that perform tasks
such as automatic generation of the code documentation, and the creation of a
tar file containing the entire source code, its documentation files, and the
input and output files. This has proved to be of great benefit in keeping all
of the data from a given run together for archival purposes.

\subsection{Compilation}

Compilation is trivially performed using the makefile. Currently, the
available options are (type the following on the command line):
\begin{tabbing}
123\=12345678901234567890\= \kill
\> \texttt{make ARCH=FUN} \> CCFE Fusion Unix Network (Intel Fortran ifort compiler) \\
\> \texttt{make ARCH=GFORT} \> GNU Fortran compiler (gfortran; N.B.\ only versions
4.6.3 or above) \\
\> \texttt{make ARCH=JAC} \> JET Analysis Cluster (pgf95 compiler)
\end{tabbing}
The Intel Fortran compiler is the default option, so typing simply
\texttt{make} will have the same effect as typing \texttt{make ARCH=FUN}.

The code is almost trivial to port to new architectures, by adding extra
stanzas to the makefile as required.

To force a full recompilation, type \texttt{make clean}. N.B.\ This will
remove (without prompting) all ``backup'' files (\texttt{*\~}) as well as
certain other files from the working directory.

Extra run-time diagnostics (array-bound checking, etc.) is turned on by
default at present; the code is sufficiently quick to run that the performance
penalty is barely noticeable. However, this can be turned off (beware\ldots
results can change!), by adding \texttt{DEBUG=NO} to the relevant command
above, i.e.\ type \verb+make ARCH=... DEBUG=NO+.  N.B.\ there must be no
spaces either side of the \texttt{=} characters in any of the above commands.

\subsection{Archiving utilities}

The makefile can be used in a number of ways to pack the various file sets
together into a single compressed tar file, for ease of portability and
archiving.

\begin{itemize}

\item \texttt{make tar}: Typing this command produces a file
  \texttt{process.tar.gz} that contains all of the source files, utility
  programs and documentation files necessary to run the program from scratch
  on a new machine or in a new directory. (Note that the input files \indat\
  and \texttt{device.dat} are not included.) This is useful for transferring
  new copies of source files, etc.\ into an existing directory already
  containing a previous \process\ run, as the pre-existing customised input
  files will not be overwritten. To extract the individual files again, copy
  the file to the destination working directory and type
  \verb+tar zxvf process.tar.gz+

\item \texttt{make archive}: Typing this command produces a file
  \texttt{process\_run.tar.gz} containing the working directory's input and
  output files (\indat, \outdat, \plotdat, \mfile\ and \texttt{device.dat}
  (which must exist to avoid an error message; see Section~\ref{sec:device}),
  together with all of the source files, utility programs and documentation
  files. These files together comprise the full set that define a given run,
  and so the file produced by this command is suitable for long-term storage
  for archival purposes. To extract the contents, type
  \verb+tar zxvf process_run.tar.gz+
\end{itemize}

Note that each of the above commands will overwrite the given named tar file
if one already exists in the working directory.

\subsection{Code documentation}

The makefile can also be used to produce source code documentation, as
described in the next section.

\section{Automatic Documentation}
\label{sec:autodoc}

The \process\ source code is self-documenting to a degree, using an included
parser program (\texttt{autodoc}) to generate html files for each subprogram
from specially-formatted comment lines within the code. It is the
responsibility of the programmer to keep the \texttt{autodoc} comments within
the source code relevant, comprehensive and up-to-date! Use the examples in
the code as a starting basis for new routines; the output section
corresponding to the various \texttt{autodoc} tags should be
self-explanatory. The following files are used:
\begin{verbatim}
   autodoc.f90
   adheader.src
   adfooter.src
\end{verbatim}
To create the ($\sim 380$) html files from the source code, type
\verb+make html+.  Then, point your favourite web browser to the file
\texttt{process.html} or \texttt{calling\_tree.html} in the working
directory. The variable descriptor file \texttt{vardes.html} (see
Section~\ref{sec:vardes}) is also produced at the same time (though it is best
to modify the file manually afterwards to remove empty sections from it).

In addition, a full \LaTeX\/ User Guide \texttt{process.tex} (this document!) is
rigorously maintained to ensure its continuing strict agreement with the
evolving source code. To create an Adobe PDF file \texttt{process.pdf} from
\texttt{process.tex} (and the associated PostScript figures),
type \verb+make userguide+.

Finally, to produce both the html files and the User Guide in one go, simply
type \verb+make doc+.

