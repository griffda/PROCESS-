\mychapter{Introduction}
\label{chap:intro}

\section{Rationale}

During the course of studies into a proposed fusion power plant, there may be
times when questions of the following type arise:
\begin{quote}
Are the machine's physics and engineering parameters consistent with one
another?

Which machine of a given size and shape produces the cheapest electricity?

What is the effect of a more optimistic limit on the maximum plasma density on
the amount of auxiliary power required?
\end{quote}

Questions such as these are extremely difficult to answer, since the large
number of parameters involved are highly dependent on one another.
Fortunately, computer programs have been written to address these issues, and
\process\ is one of them.

Suppose that an outline power plant design calls for a machine with a given
size and shape, which will produce a certain net electric power.  There may be
a vast number of different conceptual machines that satisfy the problem as
stated so far, and \process\ can be used in ``non-optimisation'' mode to find
one of these whose physics and engineering parameters are
self-consistent. However, the machine found by \process\ in this manner may
not be possible to build in practice --- the coils may be overstressed, for
instance, or the plasma pressure may exceed the maximum possible
value. \process\ contains a large number of constraints to prevent the code
from finding a machine with such problems, and running the code in so-called
``optimisation'' mode forces these constraints to be met. The number of
possible conceptual machines is thus considerably reduced, and optimisation of
the parameters with respect to (say) the cost of electricity will reduce this
number to a minimum (possibly one).

Formally then, \process\ is a systems code that calculates in a
self-consistent manner the parameters of a fusion power plant with a specified
performance, ensuring that its operating limits are not violated, and with the
option to optimise a given function of these parameters.

\section{History}

\process\ is derived from several earlier systems codes, but is largely based
on the TETRA (Tokamak Engineering Test Reactor Analysis) code~\cite{tetra} and
its descendant STORAC (Spherical TOrus Reactor Analysis Code)~\cite{storac},
which includes routines relevant to the spherical tokamak class of
machines. These codes, and much of the original version of \process\ itself,
were written by personnel at Oak Ridge National Laboratory in Tennessee, USA,
with contributions from a number of other laboratories in the USA\@. In
addition, many of the mathematical routines have been taken from a number of
different well-established source libraries.

A great deal of effort was expended at Culham on the code's arrival from ORNL in the early
1990s to upgrade and extend the code, including the addition of machines based on the stellerator, reversed field pinch and inertial confinement concepts.

\process\ is being developed actively. This User Guide is updated in parallel with the 
code itself to ensure that the documentation remains consistent with
the latest version of the code. It is to be hoped that it will be of
assistance not only to users of \process, but to anyone using  \process\ outputs or models based on them.

%\section{Layout of the User Guide}

%Chapter~\ref{chap:overview} provides an overview of the program, and outlines
%the numerical and programming concepts involved. Chapter~\ref{chap:models}
%describes the physics, engineering and economic models that are used within
%the code, and lists the switches available allowing the user to customise the
%models' details to achieve the desired simulation. Chapter~\ref{chap:run}
%describes how to run the program from scratch, and provides a number of hints
%and suggestions for the user to bear in mind to help the code find a feasible
%machine. Chapter~\ref{chap:modify} shows how to modify the code in specific
%ways, for example how to add extra constraints and variables to the code. A
%useful set of utility programs is introduced in Chapter~\ref{chap:utilities},
%and some code management tools are described in
%Chapter~\ref{chap:codetools}. Finally, the Appendices give detailed
%information about the optimisation method, and example input files for \process\
%in non-optimisation and optimisation modes.

\section{Sources of information}

The output file OUT.DAT contains details of all the constraints and iteration variables selected, the input and output parameters, and which models have been chosen. It is essential to study the full output before using the results in any way.

Details of the physics, engineering and cost models are being published:

~\cite{kovari_physics}: M. Kovari et al., PROCESS: a systems code for fusion power plants - Part 1: Physics

~\cite{kovari_eng}: M. Kovari et al., PROCESS: a systems code for fusion power plants - Part 2: Engineering, in preparation

~\cite{kovari_cost}: M. Kovari et al., The cost of a fusion power plant: extrapolation from ITER, in preparation

To view sample input and output files and a description of the variables, or to report bugs, request improvements, propose new models, or contact the \process\ developers, please visit

\textcolor{blue}{\href{http://www.ccfe.ac.uk/powerplants.aspx}{ccfe.ac.uk/powerplants.aspx}}.

