\mychapter{Introduction}
\label{chap:intro}

\section{Rationale}

During the course of studies into a proposed fusion power plant, there may be
times when questions of the following type arise:
\begin{quote}
Are the machine's physics and engineering parameters consistent with one
another?

Which machine of a given size and shape produces the cheapest electricity?

What is the effect of a more optimistic limit on the maximum plasma density on
the amount of auxiliary power required?
\end{quote}

Questions such as these are extremely difficult to answer, since the large
number of parameters involved are highly dependent on one another.
Fortunately, computer programs have been written to address these issues, and
\process\ is one of them.

Suppose that an outline power plant design calls for a machine with a given
size and shape, which will produce a certain net electric power.  There may be
a vast number of different conceptual machines that satisfy the problem as
stated so far, and \process\ can be used in ``non-optimisation'' mode to find
one of these whose physics and engineering parameters are
self-consistent. However, the machine found by \process\ in this manner may
not be possible to build in practice --- the coils may be overstressed, for
instance, or the plasma pressure may exceed the maximum possible
value. \process\ contains a large number of constraints to prevent the code
from finding a machine with such problems, and running the code in so-called
``optimisation'' mode forces these constraints to be met. The number of
possible conceptual machines is thus considerably reduced, and optimisation of
the parameters with respect to (say) the cost of electricity will reduce this
number to a minimum (possibly one).

Formally then, \process\ is a systems code that calculates in a
self-consistent manner the parameters of a fusion power plant with a specified
performance, ensuring that its operating limits are not violated, and with the
option to optimise a given function of these parameters.

It would not be fair to call \process\ a fusion power plant \textit{design}\/
code, as this implies that a great deal of complexity would need to be present
in each and every model describing one of the component systems. Such
complexity is, however, incompatible with the code's iterative approach to
solving the optimisation problem, since this requires repeated evaluation of
the same (large number of) expressions. This is not to say that the models
employed by the code are oversimplified --- in general they represent good
numerical estimates of present theoretical understanding, or are fits to
experimental data. \process\ provides a useful overall description of how a
conceptual and feasible power plant may look.

\section{History}

\process\ is derived from several earlier systems codes, but is largely based
on the TETRA (Tokamak Engineering Test Reactor Analysis) code~\cite{tetra} and
its descendant STORAC (Spherical TOrus Reactor Analysis Code)~\cite{storac},
which includes routines relevant to the spherical tokamak class of
machines. These codes, and much of the original version of \process\ itself,
were written by personnel at Oak Ridge National Laboratory in Tennessee, USA,
with contributions from a number of other laboratories in the USA\@. In
addition, many of the mathematical routines have been taken from a number of
different well-established source libraries.

Since the code is descended from such a wide range of sources, its structure
was initially not ideal from the programmer's viewpoint.  Non-standard
practices and inconsistent layout within the code led to difficulties in
modifying, interpreting and indeed running the code. A great deal of effort
was therefore expended at Culham on the code's arrival from ORNL in the early
1990s to improve this situation, with the code being given a complete but
careful upgrade, routine by routine. For many years this Fortran~77 code was
used for systems code studies of various power plant scenarios, and was
modified from time-to-time by the addition of new and/or improved models,
including machines based on the stellerator, reversed field pinch and inertial
confinement concepts.

In 2012, the code structure was revised again to allow it to benefit from
modern software practices, and the whole program was upgraded to
Fortran~90/95. At the same time a number of useful code management utilities
were added.

As with all active research codes, \process\ will continue to be developed
into the future. This User Guide is updated in parallel with the Fortran
source code itself to ensure that the documentation remains consistent with
the latest version of the code. It is to be hoped that it will be of
assistance to all users of \process, whether they are planning to modify or
run the code, or are simply trying to understand what the code aims to
achieve.

\section{Layout of the User Guide}

The User Guide is divided into a small number of logically separate chapters,
each one of which provides specific information on a given topic. It depends
on the user's motive for referring to the guide as to which chapter will be
the most useful, although hopefully the style and structure adopted will allow
one to browse through without difficulty.

Chapter~\ref{chap:overview} provides an overview of the program, and outlines
the numerical and programming concepts involved. Chapter~\ref{chap:models}
describes the physics, engineering and economic models that are used within
the code, and lists the switches available allowing the user to customise the
models' details to achieve the desired simulation. Chapter~\ref{chap:run}
describes how to run the program from scratch, and provides a number of hints
and suggestions for the user to bear in mind to help the code find a feasible
machine. Chapter~\ref{chap:modify} shows how to modify the code in specific
ways, for example how to add extra constraints and variables to the code. A
useful set of utility programs is introduced in Chapter~\ref{chap:utilities},
and some code management tools are described in
Chapter~\ref{chap:codetools}. Finally, the Appendices give detailed
information about the optimisation method, example input files for \process\
in non-optimisation and optimisation modes, and lists of references that
provide information about the code status, its location, and other details
relating to the implementation of \process\ to date.

