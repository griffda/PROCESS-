%%-*-Latex-*-
\mychapter{Introduction}
\label{chap:intro}

\section{Rationale}

The \PS systems code is being developed to provide an integrated and
self-consistent treatment of the physics, engineering, economic, safety and
environmental characteristics of fusion power plants. This will enable issues
of the feasibility and safety advantages of different power plant designs to
be systematically explored.

During the course of studies of a proposed fusion power plant, there may be
times when questions of the following type arise:
\begin{quote}
Are the machine's physics and engineering parameters consistent with one
another?

Which machine of a given size and shape produces the cheapest electricity?

What is the effect of a more optimistic limit on the plasma beta on the amount
of auxiliary power required?
\end{quote}

Questions such as these are extremely difficult to answer, since the large
number of parameters involved are highly dependent on one another.
Fortunately, computer programs have been written to address these issues, and
\PS is one of them.

Suppose that an outline power plant design calls for a machine with a given
size and shape, which will produce a certain net electric power.  There may be
a vast number of different conceptual machines that satisfy the problem as
stated so far, and \PS can be used in non-optimisation mode to find one of
these whose physics and engineering parameters are self-consistent. However,
the machine found by \PS in this manner may not be possible to build in
practice --- the coils may be overstressed, for instance, or the plasma
pressure may exceed the maximum possible value. \PS contains a large number of
constraints to prevent the code from finding a machine with such problems, and
running the code in optimisation mode forces these constraints to be met. The
number of possible conceptual machines is thus considerably reduced, and
optimisation of the parameters with respect to (say) the cost of electricity
will reduce this number to a minimum (possibly one).

Formally then, \PS is a systems code that calculates in a self-consistent
manner the parameters of a fusion power plant with a specified performance,
ensuring that its operating limits are not violated, and with the option to
optimise a given function of these parameters.

It would not be fair to call \PS a fusion power plant design code, as this
implies that a great deal of complexity would need to be present in each and
every model describing one of the component systems. Such complexity is,
however, incompatible with the code's iterative approach to solving the
optimisation problem, since this requires repeated evaluation of the same
(large number of) expressions. This is not to say that the models employed by
the code are oversimplified --- in general they represent good numerical
estimates of present theoretical understanding, or are fits to experimental
data. \PS provides a useful overall description of how a conceptual and
feasible power plant may look.

\section{History}

\PS is derived from several earlier systems codes, but is largely based on the
TETRA (Tokamak Engineering Test Reactor Analysis) code~\cite{tetra} and its
descendant STORAC (Spherical TOrus Reactor Analysis Code)~\cite{storac}, which
includes routines relevant to the tight aspect ratio class of tokamaks. These
codes, and much of the original version of \PS itself, were written by
personnel at Oak Ridge National Laboratory in Tennessee, USA, with
contributions from a number of other laboratories in the USA\@. In addition,
many of the mathematical routines have been taken from a number of different
well-established source libraries.

Since the code is descended from such a wide range of sources, its structure
was initially not ideal from the programmer's viewpoint.  Non-standard
practices and inconsistent layout within the code could have led to
difficulties in modifying, interpreting and indeed running the code. A great
deal of effort has therefore been expended at Culham since the code's arrival
from ORNL to improve this situation, with the code being given a complete but
careful upgrade, routine by routine. A {\em single}\/ master copy of \PS now
exists, the details of which are described here. The culmination of the work
to improve the usability of the code is this User Guide, which hopefully will
be of assistance to all users of \PSC whether they are planning to modify or
run the code, or are simply trying to understand what the code aims to
achieve.

As with all active research codes, \PS will continue to be developed for some
time. As explained earlier in this introduction, the code will be used as the
basis of power plant environmental and safety studies by the inclusion of
further models and constraints. This User Guide is updated regularly to ensure
that the documentation is consistent with the latest version of the code.

\section{Layout of the User Guide}

The User Guide is divided into a small number of logically separate units,
each one of which provides specific information on a given topic. It depends
on the user's motive for referring to the manual as to which chapter will be
the most useful, although hopefully the style and structure adopted will allow
one to browse through without difficulty.

Chapter~\ref{chap:overview} provides an overview of the program and the
machine that is modelled by it. Chapter~\ref{chap:models} goes into slightly
more detail, and discusses the various physics and engineering models that are
used within the code to describe the power plant systems.
Chapter~\ref{chap:run} describes how to run the program from scratch, and
provides a number of hints and suggestions to bear in mind when the code does
not find a feasible machine. Chapter~\ref{chap:modify} shows how to modify the
code in specific ways, for example how extra constraints and variables should
be added to the code. Appendices~\ref{app:infile1} and~\ref{app:infile2}
contain example input files for \PS in non-optimisation and optimisation
modes, respectively. Finally, Appendix~\ref{app:doc} contains references for
useful Work File Notes~\cite{PWF} that provide information about the code
status, its location, and other details relating to the implementation of \PS
to date.

This manual has been written in such a way as to (a) lead a new user of \PS
into a clear understanding of the code's concepts, structure and models, and
(b) help a more experienced user to set up and run the code efficiently and
quickly. Potential users of \PS need only a basic knowledge of potential
fusion power plants, and access to the code itself. No specialised knowledge
in computers or computing is required.

