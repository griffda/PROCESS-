\myappendix{Non-optimisation Input File}
\label{app:infile1}

The following is a typical input file used to run \PS in
non-optimisation mode. Comments have been added to the right of each
line.

\footnotesize
\begin{verbatim}
$inpt1                                | Start of input block INPT1 (numerics)
NEQNS = 14,                           | Number of active constraint equations
NVAR = 14,                            | Number of active iteration variables
ICC =   1,  2, 10, 11, 7, 16, 24, 5, 31, 32, 33, 34, 35, 36, | Constraint eqns
ixc =   5, 10, 12, 3,  7,  6, 36, 9, 48, 49, 50, 51, 53, 54, | Iteration variables
IOPTIMZ = -1,                         | Turn off optimisation
$end                                  | End of input block INPT1
                                      | 
$INEQDAT                              | Start of input block INEQDAT (f-values etc)
FBETATRY = 1.0                        | N.B. active iteration variable 36
$end                                  | End of input block INEQDAT
                                      | 
$PHYDAT                               | Start of input block PHYDAT (physics)
ASPECT = 3.5,                         | Machine aspect ratio
BETA = 0.042,                         | N.B. active iteration variable 5
BT = 6.,                              | Toroidal field on axis
DENE = 1.5e20,                        | N.B. active iteration variable 6
FVSBRNNI = 1.0,                       | Non-inductive volt seconds fraction
DNBETA = 3.5,                         | Troyon g coefficient
HFACT = 2.,                           | N.B. active iteration variable 10
ICURR = 4,                            | Use ITER current scaling
ISC = 6,                              | Use ITER 89-P confinement time scaling law
IINVQD = 1,                           | Use inverse quadrature
IITER = 1,                            | Use ITER fusion power calculations
ISHAPE = 0,                           | Use input values for KAPPA and TRIANG
KAPPA = 2.218,                        | Plasma elongation
Q = 3.0,                              | Edge safety factor
RMAJOR = 7.0,                         | N.B. active iteration variable 3
RNBEAM = 0.0002,                      | N.B. active iteration variable 7
TE = 15.,                             | Electron temperature
TRIANG = 0.6                          | Plasma triangularity
$END                                  | End of input block PHYDAT
                                      | 
$CDDAT                                | Start of input block CDDAT (current drive)
IRFCD = 1,                            | Use current drive
IEFRF = 5                             | Use ITER neutral beam current drive
FEFFCD = 3.,                          | Artificially enhance efficiency
$END                                  | End of input block CDDAT
                                      | 
$TIME                                 | Start of input block TIME (times)
TBURN = 227.9                         | Burn time
$END                                  | End of input block TIME
                                      | 
$DIVT                                 | Start of input block DIVT (divertor)
ANGINC=0.262,                         | Angle of incidence of field lines on plate
PRN1=0.285                            | Density ratio
$END                                  | End of input block DIVT
                                      | 
$BLD                                  | Start of input block BLD (machine build)
BORE = 0.12,                          | Machine bore
OHCTH = 0.1,                          | OH coil thickness
GAPOH = 0.08,                         | Inboard gap
TFCTH = 0.9,                          | Inboard TF coil leg thickness
DDWI = 0.07,                          | Internal dewar thickness
SHLDITH = 0.69,                       | Inboard shield thickness
BLNKITH = 0.115,                      | Inboard blanket thickness
FWITH = 0.035,                        | Inboard first wall thickness
SCRAPLI = 0.14,                       | Inboard scrape-off layer thickness
SCRAPLO = 0.15,                       | Outboard scrape-off layer thickness
FWOTH = 0.035,                        | Outboard first wall thickness
BLNKOTH = 0.235,                      | Outboard blanket thickness
SHLDOTH = 1.05,                       | Outboard shield thickness
GAPOMIN = 0.21,                       | Outboard gap
VGAPTF = 0,                           | Vertical gap
$END                                  | End of input block BLD
                                      | 
$TFC                                  | Start of input block TFC (TF coils)
OACDCP = 1.4e7,                       | N.B. active iteration variable 12
ITFSUP = 1,                           | Use superconducting TF coils
RIPMAX = 5.,                          | Maximum TF ripple
$END                                  | End of input block TFC
                                      | 
$PFC                                  | Start of input block PFC (PF coils)
NGRP = 3,                             | Three groups of PF coils
IPFLOC = 1,2,3,                       | Locations for each group
NCLS = 2,2,2,1,                       | Number of coils in each group
COHEOF = 1.85e7,                      | OH coil current at End Of Flat-top
FCOHBOP = 0.9,                        | OH coil current at Begin. Of Pulse / COHEOF
ROUTR = 1.5,                          | Radial position for group 3
ZREF(3) = 2.5,                        | Z position for group 3
OHHGHF = .71                          | Height ratio OH coil / TF coil
$END                                  | End of input block PFC
                                      | 
$FWBLSH                               | Start of input block FWBLSH (1st wall etc.)
LBLNKT=0                              | Use old blanket model
DENSTL=7800.                          | Steel density
$END                                  | End of input block FWBLSH
                                      | 
$COSTINP                              | Start of input block COSTINP (costs)
IREACTOR = 1,                         | Calculate cost of electricity
IFUELTYP = 0                          | Treat blanket, first wall etc as capital cost
$END                                  | End of input block COSTINP
                                      | 
$UCSTINP                              | Start of input block UCSTINP (unit costs)
UCHRS = 87.9,                         | }
UCCPCL1 = 250,                        | } Unit costs
UCCPCLB = 150                         | }
$END                                  | End of input block UCSTINP
                                      | 
$SWEP                                 | Start of input block SWEP (scans)
ISWEEP=0                              | No scans (non-optimisation mode)
$END                                  | End of input block SWEP
                                      | 
$BCOM                                 | Start of input block BCOM (buildings)
FNDT = 2.                             | Foundation thickness
$END                                  | End of input block BCOM
                                      | 
$BLDINP                               | Start of input block BLDINP (buildings)
EFLOOR=1.d5                           | Effective total floor space
$END                                  | End of input block BLDINP
                                      | 
$HTPWR                                | Start of input block HTPWR (heat transport)
ETATH=0.35                            | Thermal to electric conversion efficiency
$END                                  | End of input block HTPWR
                                      | 
$HTTINP                               | Start of input block HTTINP (heat transport)
FMGDMW = 0.                           | Power to MGF units
$END                                  | End of input block HTTINP
                                      | 
$HTRINP                               | Start of input block HTRINP (heat transport)
BASEEL=5.e6                           | Base plant electric load
$END                                  | End of input block HTRINP
                                      | 
$EST                                  | Start of input block EST (energy storage)
ISCENR=2                              | Energy store option
$END                                  | End of input block EST
                                      | 
$VACCY                                | Start of input block VACCY (vacuum system)
NTYPE = 1                             | Use cryopump
$END                                  | End of input block VACCY
                                      | 
$OSECTS                               | Start of input block OSECTS (output sections)
SECT01 = 1,                           | }
SECT02 = 1,                           | }
SECT03 = 1,                           | }
SECT04 = 1,                           | }
SECT05 = 1,                           | }
SECT06 = 1,                           | }
SECT07 = 1,                           | }
SECT08 = 1,                           | }
SECT09 = 1,                           | }
SECT10 = 1,                           | } Turn on all output sections
SECT11 = 1,                           | }
SECT12 = 1,                           | }
SECT13 = 1,                           | }
SECT14 = 1,                           | }
SECT15 = 1,                           | }
SECT16 = 1,                           | }
SECT17 = 1,                           | }
SECT18 = 1,                           | }
SECT19 = 1,                           | }
SECT20 = 1                            | }
SECT21 = 1                            | }
$END                                  | End of input block OSECTS
\end{verbatim}
\normalsize
