\myappendix{Optimisation Input File}
\label{app:infile2}

The following is a typical input file used to run \process\ in optimisation
mode. Comments in [\ldots] have been added to the right of each line.

\footnotesize
\begin{verbatim}
* Numerics information                [Comment]

boundl(1) = 2.5,                      [Bound on iteration variable 1 (aspect)]
BOUNDU(10) = 3.,                      [Bound on iteration variable 10 (hfact)]
BOUNDU(60) = 4.d4,                    [Bound on iteration variable 60 (cpttf)]
NEQNS = 15,                           [Number of active constraint equations]
NVAR = 25                             [Number of active iteration variables]
ICC =   1,  2, 10, 11, 7, 16, 8, 24, 31, 32, 33, 34, 35, 36, 14, [Constraint eqns]
ixc =   5, 10, 12, 3,  7,        36, 48, 49, 50, 51, 53, 54, 19, [Corresponding...]
        1, 2, 4, 6, 13, 16, 29, 56, 57, 58, 59, 60,              [...iteration variables]
IOPTIMZ = 1,                          [Turn on optimisation]
MINMAX = 6,                           [Minimise cost of electricity]

ISWEEP=7,                             [Seven point scan]
NSWEEP=6,                             [Use WALALW as scanning variable]
SWEEP= 6.0,5.5,4.5,4.0,3.5,3.0,2.5    [Values of WALALW for each scan point]

* F-values and limits

FBETATRY = 1.0                        [N.B. active iteration variable 36]

* Physics parameters

ASPECT = 3.5,                         [N.B. active iteration variable 1]
BETA = 0.042,                         [N.B. active iteration variable 5]
BT = 6.,                              [N.B. active iteration variable 2]
DENE = 1.5e20,                        [N.B. active iteration variable 6]
FVSBRNNI = 1.0,                       [Non-inductive volt seconds fraction]
DNBETA = 3.5,                         [Beta g coefficient]
HFACT = 2.,                           [N.B. active iteration variable 10]
ICURR = 4,                            [Use ITER current scaling]
ISC = 6,                              [Use ITER 89-P confinement time scaling law]
IINVQD = 1,                           [Use inverse quadrature]
IITER = 1,                            [Use ITER fusion power calculations]
ISHAPE = 0,                           [Use input values for KAPPA and TRIANG]
KAPPA = 2.218,                        [Plasma elongation]
Q = 3.0,                              [Edge safety factor]
RMAJOR = 7.0,                         [N.B. active iteration variable 3]
RNBEAM = 0.0002,                      [N.B. active iteration variable 7]
TBURN = 227.9                         [Burn time]
TE = 15.,                             [N.B. active iteration variable 4]
TRIANG = 0.6                          [Plasma triangularity]

* Current drive parameters

IRFCD = 1,                            [Use current drive]
IEFRF = 5                             [Use ITER neutral beam current drive]
FEFFCD = 3.,                          [Artificially enhance efficiency]

* Divertor parameters

ANGINC=0.262,                         [Angle of incidence of field lines on plate]
PRN1=0.285                            [Density ratio]

* Machine build

BORE = 0.12,                          [N.B. active iteration variable 29]
OHCTH = 0.1,                          [N.B. active iteration variable 16]
GAPOH = 0.08,                         [Inboard gap]
TFCTH = 0.9,                          [N.B. active iteration variable 13]
DDWI = 0.07,                          [Vacuum vessel thickness]
SHLDITH = 0.69,                       [Inboard shield thickness]
BLNKITH = 0.115,                      [Inboard blanket thickness]
FWITH = 0.035,                        [Inboard first wall thickness]
SCRAPLI = 0.14,                       [Inboard scrape-off layer thickness]
SCRAPLO = 0.15,                       [Outboard scrape-off layer thickness]
FWOTH = 0.035,                        [Outboard first wall thickness]
BLNKOTH = 0.235,                      [Outboard blanket thickness]
SHLDOTH = 1.05,                       [Outboard shield thickness]
GAPOMIN = 0.21,                       [Outboard gap]
VGAPTF = 0,                           [Vertical gap]

* First wall, blanket, shield parameters

LBLNKT=0                              [Use old blanket model]
DENSTL=7800.                          [Steel density]

* TF coil parameters

OACDCP = 1.4e7,                       [N.B. active iteration variable 12]
ITFSUP = 1,                           [Use superconducting TF coils]
RIPMAX = 5.,                          [Maximum TF ripple]

* PF coil parameters

NGRP = 3,                             [Three groups of PF coils]
IPFLOC = 1,2,3,                       [Locations for each group]
NCLS = 2,2,2,1,                       [Number of coils in each group]
COHEOF = 1.85e7,                      [central solenoid current at End Of Flat-top]
FCOHBOP = 0.9,                        [central solenoid current at Begin. Of Pulse / COHEOF]
ROUTR = 1.5,                          [Radial position for group 3]
ZREF(3) = 2.5,                        [Z position for group 3]
OHHGHF = .71                          [Height ratio central solenoid / TF coil]

* Vacuum system parameters

NTYPE = 1                             [Use cryopump]

* Heat transport parameters

ETATH=0.35                            [Thermal to electric conversion efficiency]
FMGDMW = 0.                           [Power to MGF units]
BASEEL=5.e6                           [Base plant electric load]
ISCENR=2                              [Energy store option]

* Buildings

FNDT = 2.                             [Foundation thickness]
EFLOOR=1.d5                           [Effective total floor space]

* Costs

IREACTOR = 1,                         [Calculate cost of electricity]
IFUELTYP = 0                          [Treat blanket, first wall etc as capital cost]
UCHRS = 87.9,                         [Unit cost of heat rejection system]
UCCPCL1 = 250,                        [Unit cost of high strength tapered copper]
UCCPCLB = 150                         [Unit cost of TF outer leg plate coils]
\end{verbatim}
\normalsize

