\myappendix{The Input File}
\label{app:infile}

The input file \indat\ is used to change the values of the physics,
engineering and other code parameters from their default values, and to set up
the numerics (constraint equations, iteration variables etc.) required to
define the problem to be solved.  The user interface writes the input file, so it is not necessary to edit it directly.

\subsection{Tokamak, stellarator, RFP or IFE?}
\label{sec:device}

The default model is the tokamak.  To select a stellarator, reversed field pinch or inertial fusion energy plant, an additional input file is required, \texttt{device.dat}, which should contain a single character in the first
line, which is interpreted as follows:
\begin{tabbing}
\hspace{15mm}\= \texttt{0} : use tokamak model \\
\> \texttt{1} : use stellarator model \\
\> \texttt{2} : use reversed field pinch model \\
\> \texttt{3} : use inertial fusion energy model
\end{tabbing}

\subsection{File format}

Variables can be specified in any order in the input file.  Comment lines start with a \texttt{*} character.  Data lines are of the form
\begin{verbatim}
variable = value
\end{verbatim}
where \texttt{variable} is the name of one of the input parameters or
iteration variables listed in the variable descriptor file, and \texttt{value}
is the (usually numerical) initial value required for that variable. (Arrays,
as opposed to scalar quantities, are treated differently --- see below.) All input data are screened for non-sensible values.

The following rules must be obeyed when writing an input file:

\begin{enumerate}

\item Each variable must be on a separate line.

\item Variable names can be upper case, lower case, or a mixture of both.

\item Spaces may not appear within a variable name or data value.

\item Other spaces within a line, and trailing spaces, are ignored.

\item Commas are not necessary between variables (but see below).

\item Data can extend over more than one line.

\item One-dimensional arrays can be explicitly subscripted, or unscripted, in
  which case the following element order is assumed: \texttt{A(1), A(2),
    A(3),...}

\item At present, multiple dimension arrays can only be handled without
  reference to explicit subscripts, in which case the following element order
  is assumed: \texttt{B(1,1), B(2,1), B(3,1),...} The use of the input file to
  specify multiple dimension array elements is prone to error.

\item Unscripted array elements must be separated by commas.

\item Blank lines are allowed anywhere in the input file.

\item Lines starting with a \texttt{*} are assumed to be comments.

\item Comment lines starting with five or more asterisks (i.e.\
  \texttt{*****}) are reproduced verbatim in the output file. This feature is not recommended, as these comments are likely to become out of date.  The user interface does not support this feature.

\item In-line comments are \textit{usually}\/ ignored, but there can be
  problems if one contains a comma (\texttt{,}). If this is the case, there
  must also be a comma after the variable's value and before the comment.

\end{enumerate}

It is useful to divide the input file into sections, using suitable comment
lines, to help the user keep related variables together.

The following is a valid fragment of an input file (the vertical lines are
simply to help show the column alignment):
\begin{center}
\begin{tabular}{||l}
$\!\!$\texttt{* This line is a comment that will not appear in the output} \\
$\!\!$\texttt{***** This line is a comment that will appear in the output} \\
$\!\!$\texttt{boundl(1) = 2.5,} \\
$\!\!$\texttt{BOUNDU(10) = 3.,} \\
$\!\!$\texttt{BOUNDU(45) = 1,} \\
$\!\!$\texttt{* Another comment... Note that real values can be entered as if} \\
$\!\!$\texttt{* they were integers, and vice versa (but it's not recommended...)} \\
$\!\!$\texttt{epsfcn = 10.e-4,} \\
$\!\!$\texttt{Ftol = 1.D-4,} \\
$\!\!$\texttt{* The next line sets the first five elements of array icc:} \\
$\!\!$\texttt{ICC =   2, 10, 11, 24, 31} \\
$\!\!$\texttt{* The next line sets the first ten elements of array ixc:} \\
$\!\!$\texttt{ixc =   10, 12, 3, 36, 48,} \\
$\!\!$\hspace{15mm}\texttt{1, 2, 6, 13, 16,} \\
$\!\!$\texttt{IOPTIMZ = 1,} \\
$\!\!$\texttt{maxcal = 200} \\
$\!\!$\texttt{ nsweep = 7} \\
$\!\!$\texttt{NEQNS = 5,    This is an in-line comment} \\
$\!\!$\texttt{NVAR = 10,    Another, but successfully containing a comma!} \\
\end{tabular}
\end{center}

The following are \textit{invalid}\/ entries in the input file
(Q: Why?!):
\begin{center}
\begin{tabular}{||l}
$\!\!$\texttt{boundl(1,1) = 2.5,} \\
$\!\!$\texttt{BOUNDU(N) = 3.,} \\
$\!\!$\texttt{A line of `random' characters like this will clearly wreak havoc} \\
$\!\!$\texttt{eps fcn = 10.e-4, ftol = 1.D-4} \\
$\!\!$\texttt{epsvmc = 1.0 e-4} \\
$\!\!$\texttt{ICC =   2  10  11  24  31} \\
$\!\!$\texttt{IOPTIMZ = 1.0,  This will in fact be okay - but is not recommended} \\
$\!\!$\texttt{NEQNS = 5    An in-line comment on a line with only one comma (,) character} \\
\end{tabular}
\end{center}

If the code encounters a problem reading the input file, it will stop immediately
with a (hopefully) useful error message. It may be worth looking at the
contents of the output file as well, to help narrow down on which line of the
input file the problem might lie.
\normalsize

