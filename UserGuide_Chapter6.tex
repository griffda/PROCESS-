\mychapter{Utility Programs}
\label{chap:utilities}

A number of Python utilities for \process\ are available which allow the user
to perform a number of useful actions, for instance to modify the
input file \indat, run \process\ until a feasible solution is found, or to
extract and plot data from the \process\ output. All utilities are available
from \texttt{~pknight/process/bin/utilities}.

For \process\ \textit{users}, only the Python executables described in
Section~\ref{sec:py_exec} are relevant. For anyone interested in modifying the
existing code or developing new Python utilities the \process\ Python
libraries are described in Section~\ref{sec:py_lib}.

All executables are using Python library functions either from the publicly
available \texttt{numpy}, \texttt{scipy} and \texttt{matplotlib} libraries or
the \process\ Python libraries documented in Section~\ref{sec:py_lib}. To use
the \process\ Python libraries you need to make sure their directory is in
your Python path. Use the commands given in Section~\ref{sec:run_environment} to
do this.

All Python code has been written for Python 3 and adheres to the PEP8 standard.

\section{Executables}
\label{sec:py_exec}

All executables can be run by using any of the following commands:
\begin{quote}
\begin{verbatim}
python executable_name.py
python3 executable_name.py
executable_name.py
\end{verbatim}
\end{quote}
and they typically have a short description of their usage when putting
\texttt{-h} or \texttt{--help} as arguments:
\begin{quote}
\begin{verbatim}
python executable_name.py -h
\end{verbatim}
\end{quote}

Some executables come with a configuration file that typically follows the
naming convention \texttt{executable\_name.conf}. To run the executable please
copy the config file into your own work directory. Do \textbf{NOT} try to edit
the config file in the \process\ directory.

\subsection{write\_new\_in\_dat.py}

This program modifies a given \indat\ file such that all iteration variables
are given by their values in \outdat. The final values of the iteration
variables in \outdat\ are set in a block at the end of \indat. Any other
definitions of these variables are commented out.

\begin{description}
\item{\textbf{Input:}}
\indat, \outdat\ 
                                
\item{\textbf{Output:}}
\texttt{new\_IN.DAT}
\end{description}

\subsection{write\_constraints.py}

This Python script modifies the \process\ input file \indat\ to include the
constraints marked ``\texttt{Y}'' or ``\texttt{y}'' in a configuration file
called \texttt{write\_constraints.conf}. It automatically adds the required
iteration variable for each constraint, and any additional iteration variables
selected. The script also creates blank upper and lower bound equations; these
need to be completed manually or deleted. All files used need to be in the
working directory. The original \indat\ is renamed \texttt{OLD.IN.DAT}.

\begin{description}
\item{\textbf{Input:}}
\indat, \texttt{write\_constraints.conf}
                                
\item{\textbf{Output:}}
\indat, \texttt{OLD.IN.DAT}
\end{description}

\paragraph{Configuration Options:}

The configuration file \texttt{write\_constraints.conf} has the following style:
\begin{framed}
\begin{verbatim}
Constraints and associated iteration variables
y ICC=1 plasma beta consistency ixc=5 beta plasma beta 0.001d0 1.0d0
y ICC=2 global power balance    ixc=10 hfact confinement time h-factor 0.1d0 3.0d0
  ICC=3 ion power balance                                            
  ICC=4 electron power balance 
...
Remaining iteration variables
  IXC=1 aspect plasma aspect ratio 1.100D0 10.00D0
y IXC=2 bt toroidal field on axis 0.010D0 100.0D0
  IXC=3 rmajor plasma major radius 0.100D0 10.00D0
...
\end{verbatim}
\end{framed}
An example version that can be modified is available in the utilities directory.

\subsection{run\_process.py}

This program runs \process, if necessary, with varying iteration variable
start parameters until a feasible solution is found.

\begin{description}
\item{\textbf{Input:}}
\texttt{run\_process.conf}, an \indat\ file as specified in the config file
                                
\item{\textbf{Output (in the specified work directory):}} All the standard
  PROCESS output, \texttt{process.log} (log file), \texttt{README.txt}
  (contains comments from config file)
\end{description}

\paragraph{Configuration Options:}

The configuration file \texttt{run\_process.conf} has the following style:
\begin{framed}
\begin{verbatim}
* This is a comment in the config file!

* Path to working directory in which PROCESS is run.
WDIR = Run1
* original IN.DAT name 
ORIGINAL_IN_DAT = demo1a_10_sep_13.IN.DAT
* PATH to PROCESS binary
PROCESS = ~pknight/process/bin/process
* ONE line comment to be put into README.txt
COMMENT = This is a test DEMO run! ;-)
* Max. no iterations
NITER = 5
* integer seed for random number generator; use None for random seed
SEED = 2
* factor within which the iteration variables are changed
FACTOR = 1.5
* Number of allowed unfeasible points that do not trigger rerunning.
NO_ALLOWED_UNFEASIBLE = 0 
* include a summary file with the iteration variables at each stage.
INCLUDE_ITERVAR_DIFF = True
* add iteration variables - comma separated
ADD_IXC = 99, 77
* remove iteration variables - comma separated
DEL_IXC = 
* add constraint equations  - comma separated 
ADD_ICC = 57,
* remove constraint equations - comma separated
DEL_ICC =
* set any variable to a new value
* the variable does not have to exist in IN.DAT
VAR_TFTORT = 1.9
VAR_EPSVMC = 1e-4
* remove variables
DEL_VAR_IITER
\end{verbatim}
\end{framed}
A configuration file with an alternative name can be specified using the optional argument
\begin{quote}
\begin{verbatim}
run_process.py -f CONFIGFILE
\end{verbatim}
\end{quote}

\subsection{build\_index.py}

Creates an index of all \process\ run comments in a series of subfolders.

\begin{description}
\item{\textbf{Input:}}
None
                                
\item{\textbf{Output:}}
\verb|Index.txt|
\end{description}

\paragraph{Configuration Options:}

Optional arguments are:
\begin{quote}
\begin{verbatim}
# change the name of the file containing the folder description
build_index.py -r README.txt 
# append the results to Index.txt instead of creating a new file
build_index.py -m a
# change the name of the subfolder Base - default=Run
build_index.py -b Base 
# give a list of subfolder suffixes - default=all
build_index.py -s 1-4,6,8,9-12 
# increase verbosity
build_index.py -v
\end{verbatim}
\end{quote}

An example \texttt{Index.txt} file might look like this
\begin{framed}
\begin{verbatim}
Run1:
 Original run

Run2:
 Changed the no. TF coils
...
\end{verbatim}
\end{framed}

\subsection{make\_plot\_dat.py}

Creates a \plotdat-type file from \mfile.

\begin{description}
\item{\textbf{Input:}}
\verb|make_plot_dat.conf|, \mfile\
                                
\item{\textbf{Output:}}
\verb|make_plot_dat.out|
\end{description}

\paragraph{Configuration Options:}

Optional arguments are:
\begin{quote} 
\begin{verbatim}
# new variables for output
make_plot_dat.py -p rmajor
# writes make_plot_dat.out in columns 
make_plot_dat.py --columns
# resets make_plot_dat.conf to PLOT.DAT layout
make_plot_dat.py --reset-config
# file to read as input
make_plot_dat.py -f MFILE.DAT 
# run with default parameters
make_plot_dat.py --defaults
\end{verbatim}
\end{quote}

An example version of \texttt{make\_plot\_dat.conf} might look like this:
\begin{framed}
\begin{verbatim}
# make_plot_dat.out config file.
rmajor
aspect
rminor
bt
powfmw
pnetelmw
te
pdivt
strtf1
strtf2
\end{verbatim}
\end{framed}

\subsection{plot\_proc.py}

A utility to produce the standard \process\ summary output, covering the major
parameters, provenance data, and a machine cross-section. It allows a wide
range of graphical output formats and can easily be customised using the
function \texttt{gather\_info()} in \texttt{plot\_proc\_func.py}.

\begin{description}
\item{\textbf{Input:}}
 \mfile
                                
\item{\textbf{Output:}}
\verb|proc_plot_out.pdf| (or as specified by user)
\end{description}

\paragraph{Configuration Options:}

Optional arguments are:
\begin{quote}
\begin{verbatim}
# change the input file name
python plot_proc.py -f MFILE.DAT 
# change the output file name
python plot_proc.py -o out.pdf
# show the plot as well as saving the figure
python plot_proc.py -s 
\end{verbatim}
\end{quote}

\subsection{plot\_sweep.py}

This utility plots normalised values from \plotdat\ and allows comparisons of
changes in values from a sweep compared to the first point in the sweep. It is
now somewhat deprecated due to the existence of more flexible sweep-generating
and output-handling utilities.

\begin{description}
\item{\textbf{Input:}}
 \plotdat\
                                
\item{\textbf{Output:}}
\texttt{PLOT.DAT.eps} (default or as specified by user)
\end{description}

\paragraph{Configuration Options:}

Optional arguments are:
\begin{quote}
\begin{verbatim}
# creates PLOT.DAT.eps with R0, IP, beta
python plot_sweep.py -g 11 13 18 
# creates demo1.png with Te, n
python plot_sweep.py -o demo1.png 21 22
\end{verbatim}
\end{quote}

\subsection{plot\_mfile\_sweep.py}

This utility plots normalised values from \mfile\ and allows comparisons of
changes in values from a sweep compared to the first point in the sweep.

\begin{description}
\item{\textbf{Input:}}
 \mfile
                                
\item{\textbf{Output:}}
\texttt{sweep\_fig.pdf} (default or as specified by user)
\end{description}

Optional arguments are:
\begin{quote}
\begin{verbatim}
# creates sweep_fig.pdf with R0, te, aspect (same variable names as in MFILE.DAT)
python plot_mfile_sweep.py -p rmajor te aspect
# creates demo1.png with Te, n
python plot_mfile_sweep.py -o demo1.png -p te dene
# creates a sweep_fig.pdf with R0, aspect with a different MFILE.DAT
python plot_mfile_sweep.py -f diff_mfile.dat -p rmajor aspect
# Show plot to screen instead of saving with R0 and aspect
python plot_mfile_sweep.py -p rmajor aspect --show

Use -h or --help for help

\end{verbatim}
\end{quote}

\subsection{diagnose\_process.py}

This utility aids the user to interpret (unsuccessful) \process\/ runs, unless
\process\/ has terminated prematurely. It takes an existing \mfile\ and plots
the normalised iteration variables, i.e.\ the iteration variable values
normalised to their bounds such that 0 indicates an iteration variable at its
lower bound and 1 an iteration variable at its upper bound. Furthermore, it
shows the normalised constraint residuals.

\begin{description}
\item{\textbf{Input:}}
 \mfile
                                
\item{\textbf{Output:}}
  Displays plots on screen, still need to be saved by the user! (Remember to
  use -Y or -X, if \texttt{ssh}ing into a remote machine!)

\end{description}

Optional arguments are:
\begin{quote}
\begin{verbatim}
# allows to specify another location/name for the MFILE
python diagnose_process.py -f MFILE.DAT

Use -h or --help for help

\end{verbatim}
\end{quote}

\subsection{2D\_scan.py}

\Red{N.B.\ This utility is currently not working --- please do not try to use it.}

Runs a two-dimensional scan of \process\ variables. This code does not yet use
the new \process\ Python libraries.

\begin{description}
\item{\textbf{Input:}}
 \texttt{2D\_scan.conf}, \indat\
                                
\item{\textbf{Output:}} All the standard PROCESS output files,
  \texttt{ERROR.DAT} (Error matrix for all iterations), \texttt{EDGETABLE.DAT}
  (Marks if variables at bounds), \texttt{MATRICES.DAT} (Solution for each
  variable), \texttt{EDGEVARS.DAT} (Names of variables at bounds)
\end{description}

\paragraph{Configuration Options:}

The configuration file \texttt{2D\_scan.conf} has the following style
\begin{framed}
\begin{verbatim}
* This is a comment

* No evaluations of the first variable
N1= 10
* lower bound of first variable
LB1= 1
* upper bound of first variable
UB1= 3
* name of first variable
NAME1= abktflnc

* No evaluations of the second variable
N2= 15
* lower bound of the second variable
LB2= 3
* upper bound of the second variable
UB2= 8
* name of the second variable
NAME2= bktlife

* Output variables as given by their variable description
OUT_VARS
Major Radius
Pdivt
/OUT_VARS

* Directory path to IN.DAT
PATH=/DIR2INDAT
* Boolean flag to store OUT.DAT and PLOT.DAT for each run
* in the Data/ subdir
DATA=0
\end{verbatim}
\end{framed}

\subsection{a\_to\_b.py}

Takes an initial \indat\ and a target \indat\ and runs \process\ repeatedly
with the aim of walking the solution to the initial input file to a solution
using the values of the variables in the target input file.  After each step,
the output values of the iteration variables are used as the input values for
the next step.

\begin{description}
\item{\textbf{Input:}}
\texttt{a\_to\_b.conf}

\item{\textbf{Output}} 

When the program finishes, the .DAT files from the last run of \process\ can
be found in the specified working directory.

If option keep\_output = True: \indat, \outdat, \mfile\ and logged process
output of each step are stored. Files are prefaced with their step number
eg. \texttt{003.IN.DAT} and copied to the specified output directory.

\end{description}

\paragraph{Configuration Options:}

The configuration file \texttt{a\_to\_b.conf} has the following style:
\begin{framed}
\begin{verbatim}
*Comment line

*Working directory to store temporary files, default = wdir
wdir = wdir

*Switch to keep .DAT files for every step, default = True
keep_output = True

*Directory for output if keep_output = True, default = steps
outdir = steps

*IN.DAT file for A, default = A.DAT
a_filename = A.DAT

*IN.DAT file for B, default = B.DAT
b_filename = B.DAT

*Path to process binary
*path_to_process = /home/pknight/process/bin/process
path_to_process = /home/pknight/process/bin/process

*Number of iterations of vary_iteration_variables to run, default = 20
vary_niter = 20

*Number of steps to go from A to B, default = 10
nsteps = 10

*Factor to vary iteration variables within, default = 1.2
factor = 1.2

*Gap between upper and lower bounds to narrow to, default = 1.001
bound_gap = 1.001
\end{verbatim}
\end{framed}

\subsection{create\_dicts.py}

\Red{To do}

\section{Libraries}
\label{sec:py_lib}

All library modules and functions are documented using docstrings. These can
be accessed by reading the code directly or via the \texttt{help()} function
in Python.

\subsection{in\_dat.py}

A set of Python classes to read, modify and write an \indat\ file.

\begin{description}

\item{\verb|INVariable(name, value, comment="")| } Initialises an \indat\
  variable class which requires a name and a value.

\end{description}

\rule{\textwidth}{0.4pt}

\begin{description}

\item{\verb|INModule(name)|} Initialises an \indat\ module class which
  requires a name. This class stores information for an \indat\ file which has
  modules separated with lines with \$MODULE\_NAME and \$END.

\item{\verb|INModule.add_variable(var)|} Adds an \verb|INVariable| object to
  the \verb|INModule| class.

\item{\verb|INModule.remove_variable(variable_name)|} Removes an
  \verb|INVariable| object from the \verb|INModule| class.

\item{\verb|INModule.add_line(line)|} Adds a line from the \indat\ that isn't
  a variable line (e.g. a comment) to the \verb|INModule| class.

\item{\verb|INModule.get_var(var_name)|} Returns an \verb|INVariable| object
  from the \verb|INModule| class.

\item{\verb|INModule.add_constraint_eqn(eqn_number)|} Adds constraint
  equation \verb|eqn_number| to the list of constraint equations
  \verb|InVariable| class if the equation is not already in the list.

\item{\verb|INModule.remove_constraint_eqn(eqn_number)|} Removes
  constraint equation \verb|eqn_number| from the list of constraint
  equations \verb|InVariable| class if the equation is already in the list.

\item{\verb|INModule.add_iteration_variable(var_number)|} Adds iteration
  variable \verb|var_number| to the list of iteration variables
  \verb|InVariable| class if the variable is not already in the list.

\item{\verb|INModule.remove_iteration_variable(var_number)|} Removes
  iteration variable \verb|var_number| to the list of iteration variables
  \verb|InVariable| class if the variable is already in the list.

\end{description}

\rule{\textwidth}{0.4pt}

\begin{description}

\item{\verb|INDATClassic(filename="IN.DAT")| } Initialises an \indat\ class
  which can be given a different filename. This class is used for an \indat\
  with a module structure.

\item{\verb|INDATClassic.read_in_dat()| } Reads an \indat\ file with modular
  structure.

\item{\verb|INDATClassic.write_in_dat(filename="new_IN.DAT")| } Writes a new
  \indat\ with modular structure.

\item{\verb|INDATNew.read_in_dat()| } Reads an \indat\ file without modular
  structure.

\item{\verb|INDATNew.write_in_dat(filename="new_IN.DAT")| } Writes a new
  \indat\ without modular structure.

\item{\verb|INDATNew.add_variable(var)| } Adds an \verb|INVariable| object to
  the \verb|INDATNew| class.

\item{\verb|INDATNew.remove_variable(var_name)| } Removes an
  \verb|INVariable| object from the \verb|INDATNew| class.

\item{\verb|INDATNew.add_constraint_eqn(eqn_number)|} Adds constraint
  equation \verb|eqn_number| to the list of constraint equations
  \verb|InVariable| class if the equation is not already in the list.

\item{\verb|INDATNew.remove_constraint_eqn(eqn_number)|} Removes
  constraint equation \verb|eqn_number| from the list of constraint
  equations \verb|InVariable| class if the equation is already in the list.

\item{\verb|INDATNew.add_iteration_variable(var_number)| } Adds iteration
  variable \verb|var_number| to the list of iteration variables
  \verb|InVariable| class if the variable is not already in the list.

\item{\verb|INDATNew.remove_iteration_variable(var_number)| } Removes
  iteration variable \verb|var_number| to the list of iteration variables
  \verb|InVariable| class if the variable is already in the list.

\end{description}

\rule{\textwidth}{0.4pt}

\begin{description}

\item{\verb|clear_lines(lines)|} Removes comment only lines and replaces
  multiple empty lines with a single empty line.

\item{\verb|variable_type(var_name, var_value)|} Checks the type of an \indat\
  variable using \verb|DICT_VAR_TYPE|

\item{\verb|fortran_python_scientific(var_value)|} Changes from FORTRAN double
  precision notation \verb|D| to Python's \verb|e| notation.

\end{description}

\subsection{mfile.py}

A set of Python classes to read and extract data from \mfile.

\begin{description}

\item{\verb|MFileVariable(var_name, var_description)|} Object to contain
  information for a single \mfile\ variable.

\item{\verb|MFileVariable.set_scan(scan_number, scan_value)|} Sets the
  variable value for scan number \verb|scan_number|

\item{\verb|MFileVariable.get_scan(scan_number)|} Returns the variable value
  for scan number \verb|scan_number|

\item{\verb|MFileVariable.get_scans()|} Returns the variable value for all
  scans.

\end{description}

\rule{\textwidth}{0.4pt}

\begin{description}

\item{\verb|MFile(filename="MFILE.DAT")|} Object to contain information for
  all variables in an \mfile\ for all scans.

\item{\verb|MFile.search_keys(variable)|} Search for an MFILE variable in the
  data dictionary.

\item{\verb|MFile.search_des(description)|} Search for an MFILE description in
  the data dictionary.

\item{\verb|MFile.make_plot_dat(custom_keys, filename="make_plot_dat.out", file_format="row")|}
  Create a \plotdat\ equivalent for the variables in \verb|custom_keys| in
  either row or column format.

\item{\verb|MFile.get_num_scans()|} Returns the number of scans in the
  \mfile

\end{description}

\rule{\textwidth}{0.4pt}

\begin{description}

\item{\verb|read_mplot_conf(filename="make_plot_dat.conf")|} Reads the config
  file \verb|make_plot_dat.conf| and fills the list \verb|custom_keys| with
  these variables.

\item{\verb|write_mplot_conf(filename="make_plot_dat.conf")|} Writes a new
  \verb|make_plot_dat.conf| adding any additional variables that the user
  \verb|gave make_plot_dat.py| at runtime.

\end{description}

\subsection{process\_funcs.py}

This library contains a collection of functions used by various \process\
utilities.

\begin{description}

\item{\verb|get_neqns_itervars(wdir='.')| } Returns the number of equations
  and a list of variable names of all iteration variables.

\item{\verb|update_ixc_bounds(wdir='.')|} Updates the lower and upper bounds
  in \verb|DICT_IXC_BOUNDS| from \indat.

\item{\verb|get_variable_range(itervars, factor, wdir='.')|} Returns the lower
  and upper bounds of the variable range for each iteration variable.

  \texttt{itervars}: string list of all iteration variable names

  \texttt{factor}: defines the variation range for non-f-values by setting
  them to value * factor and value / factor respectively while taking their
  \process\ bounds into account.

  For f-values the allowed range is equal to their \process\ bounds.

\item{\verb|check_logfile(logfile='process.log')|} Checks the log file of the
  \process\ output.  Stops if an error occurred that needs to be fixed before
  rerunning.

\item{\verb|process_stopped(logfile='process.log')|} Checks the \process\
  logfile whether it has prematurely stopped.

\item{\verb|process_warnings(logfile='process.log')|} Checks the \process\
  logfile whether any warnings have occurred.

\item{\verb|mfile_exists()|} Checks whether \mfile\ exists.

\item{\verb|no_unfeasible_mfile(wdir='.')|} Returns the number of unfeasible
  points in a scan in \mfile.

\item{\verb|no_unfeasible_outdat(wdir='.')|} Returns the number of unfeasible
  points in a scan in \outdat.

\item{\verb|vary_iteration_variables(itervars, lbs, ubs) |} Changes the
  iteration variables in \indat\ within given bounds.

  \texttt{itervars}: string list of all iteration variable names

  \texttt{lbs}: float list of lower bounds for variables

  \texttt{ubs}: float list of upper bounds for variables

\item{\verb|get_solution_from_mfile(neqns, nvars, wdir='.')|} Returns:-

\begin{itemize}
\item \texttt{ifail}: error value of \process
\item the objective functions
\item the square root of the sum of the squares of the constraints
\item a list of the final iteration variable values
\item a list of the final constraint residue values
\item If the run was a scan, the values of the last scan point will be returned.
\end{itemize}

\item{\verb|get_solution_from_outdat(neqns, nvars)|} Returns:-

\begin{itemize}
\item \texttt{ifail}: error value of \process
\item the objective functions
\item the square root of the sum of the squares of the constraints
\item a list of the final iteration variable values
\item a list of the final constraint residue values
\item If the run was a scan, the values of the last scan point will be returned.
\end{itemize}

\end{description}

\subsection{process\_config.py}

A collection of Python classes for configuration files used by various
\process\ utilities, e.g. \texttt{run\_process.py} or
\texttt{test\_process.py}. It contains a base class \texttt{ProcessConfig} and
two derived classes \texttt{RunProcessConfig} and \texttt{TestProcessConfig}.

\begin{description}

\item{\verb|ProcessConfig()|} Object that contains the configuration
  parameters for PROCESS runs.

  \verb|filename| : Configuration file name

  \verb|wdir| : Working directory

  \verb|or_in_dat| : Original \indat\/ file

  \verb|process| : \process\/ binary

  \verb|niter| : (Maximum) number of iterations

  \verb|u_seed| : User specified seed value for the random number generator

  \verb|factor| : Multiplication factor adjusting the range in which the
  original iteration variables should be varied

  \verb|comment| : Additional comment to be written into \verb|README.txt|

\item{\verb|ProcessConfig.echo_base()|} Echos the attributes of the base class
  to standard output.

\item{\verb|ProcessConfig.echo()|} Echos the values of the current class to
  standard output.

\item{\verb|ProcessConfig.prepare_wdir()|} Prepares the work directory for the
  run.

\item{\verb|ProcessConfig.create_readme(directory='.')|} Creates a file called
  \texttt{README.txt} containing \texttt{ProcessConfig.comment}.

\item{\verb|ProcessConfig.modify_in_dat()|} Modifies the original \indat\/
  file.

\item{\verb|ProcessConfig.setup()|} Sets up the program for running.

\item{\verb|ProcessConfig.run_process()|} Runs \process\/ binary.

\item{\verb|ProcessConfig.get_comment()|} Gets the comment line from the
  configuration file.

\item{\verb|ProcessConfig.get_attribute(attributename)|} Gets a class
  attribute from the configuration file.

\item{\verb|ProcessConfig.set_base_attributes()|} Sets the attributes of the
  base class.

\end{description}

\rule{\textwidth}{0.4pt}

\begin{description}

\item{\verb|TestProcessConfig(filename='test_process.conf')|} Object that
  contains the configuration parameter of the \verb|test_process.py| program.

  \verb|ioptimz| : sets \verb|ioptimz| (optimisation solver) in \indat.

  \verb|epsvmc| : sets \verb|epsvmc| (VMCON error tolerance) in \indat.

  \verb|epsfcn| : sets \verb|epsfcn| (finite diff. steplength) in \indat.

  \verb|minmax| : sets \verb|minmax| (figure of merit switch) in \indat.

\item{\verb|TestProcessConfig.echo()|} Echos the values of the current class
  to std out.

\item{\verb|TestProcessConfig.modify_in_dat()|} Modifies \indat\/ using the
  configuration parameters.

\end{description}

\rule{\textwidth}{0.4pt}

\begin{description}

\item{\verb|RunProcessConfig(filename='run_process.conf')|} Configuration
  parameters of the \verb|run_process.py| program.

  \verb|no_allowed_unfeasible| : The number of allowed unfeasible points in a
  sweep

  \verb|create_itervar_diff| : Boolean to indicate the creation of a summary
  file of the iteration variable values at each stage

  \verb|add_ixc| : List of iteration variables to be added to \indat.

  \verb|del_ixc| : List of iteration variables to be deleted from \indat.

  \verb|add_icc| : List of constrained equations to be added to \indat.

  \verb|del_icc| : List of constrained equations to be deleted from \indat.

  \verb|dictvar| : Dictionary mapping variable name to new value (replaces old
  or gets appended)

  \verb|del_var| : List of variables to be deleted from \indat.

\item{\verb|RunProcessConfig.get_attribute_csv_list(attributename)|} Get a
  class attribute list from the configuration file; expects comma separated
  values.

\item{\verb|RunProcessConfig.set_del_var()|} Sets the
  \verb|RunProcessConfig.del_var| attribute from the config file.

\item{\verb|RunProcessConfig.set_dictvar()|} Sets the
  \verb|RunProcessConfig.dictvar| attribute from config file.

\item{\verb|RunProcessConfig.echo()|} Echos the values of the current class.

\item{\verb|RunProcessConfig.modify_in_dat()|} Modifies \indat\/ using the
  configuration parameters.

\item{\verb|RunProcessConfig.modify_vars()|} Modifies \indat\/ by adding,
  deleting and modifiying variables.

\item{\verb|RunProcessConfig.modify_ixc()|} Modifies the array of iteration
  variables in \indat.

\item{\verb|RunProcessConfig.modify_icc()|} Modifies the array of constraint
  equations in \indat.

\end{description}

% Its class structure is described in Figure \ref{fig:uml_config}.
%\begin{figure}
%\includegraphics[height=0.9\textheight]{process_config_classes.pdf}
%\caption{UML diagram of the class structure in \texttt{process\_config.py}.}
%\includegraphics[width=0.9\textwidth]{process_config_classes_limited.pdf}
%\caption{Class structure in \texttt{process\_config.py}.}
%\label{fig:uml_config}
%\end{figure}

\subsection{process\_dicts.py}

A collection of dictionaries and lists used by various \process\ utilities.

\begin{description}

\item{\verb|IFAIL_SUCCESS|} This is the \process\ error code of a successful
  run.

\item{\verb|PARAMETER_DEFAULTS|} Default values for making a \plotdat\ file
  from \mfile.

\item{\verb|DICT_VAR_TYPE|} Maps the \process\ variable name to its value
  type. The value type can be one of \verb|int_array|, \verb|int_variable|,
  \verb|real_array| or \verb|real_variable|.

\item{\verb|DICT_IXC_SIMPLE|} Maps the string number of the iteration variable
  to its variable name.

\item{\verb|DICT_IXC_FULL|} Maps the string number of the iteration variable
  to a dictionary that contains the variable name under 'name', the default
  lower variable bound (float) under 'lb' and the default upper variable bound
  (float) under 'ub'.

\item{\verb|DICT_IXC_BOUNDS|} Maps each iteration variable name to a
  dictionary that contains the default lower variable bound (float) under 'lb'
  and the default upper variable bound (float) under 'ub'.

\item{\verb|NON_F_VALUES|} List of iteration variable names that start with an
  f, but are not f-values.

\item{\verb|DICT_NSWEEP2IXC|} Maps the sweep variable number \texttt{nsweep}
  to the respective iteration variable number, if applicable.

\item{\verb|DICT_IX2NSWEEPC|} Maps the iteration variable number to the
  respective sweep variable number \texttt{nsweep}, if applicable.

\item{\verb|DICT_TF_TYPE|} Maps the \process\ TF coil type number to its type.

\item{\verb|DICT_OPTIMISATION_VARS|} Maps each figure of merit number to its
  description.

\item{\verb|DICT_IXC_DEFAULT|} Maps each iteration variable name to its
  default value.

\end{description}

\Red{Add new stuff used by GUI etc.}

\subsection{a\_to\_b\_config.py}

\Red{To do...}

\subsection{proc\_plot\_func.py}

A collection of functions and lists used by \texttt{plot\_proc.py}.

\begin{description}

\item{\verb|RADIAL_BUILD|} A list of radial build variables.

\item{\verb|VERTICAL_BUILD|} A list of vertical build variables.

\item{\verb|FILL_COLS|} A list of plotting colours

\end{description}

For all of the functions below the arguments are:

\hspace{1cm} \texttt{axis} : Matplotlib axis object to plot to

\hspace{1cm} \texttt{mfile\_data} : MFILE data object \verb|MFile|

\hspace{1cm} \texttt{scan} : Scan number to plot

\begin{description}

\item{\verb|plot_plasma(axis, mfile_data,scan)|} Function to plot plasma

\item{\verb|plot_machine_pic(axis, mfile_data, scan)|} Function to plot machine build

\item{\verb|plot_tf_coils(axis, mfile_data, scan)|} Function to plot the TF coils

\item{\verb|plot_pf_coils(axis, mfile_data, scan)|} Function to plot the PF coils

\item{\verb|plot_geometry_info(axis, mfile_data, scan)|} Function to plot the
  geometry info block

\item{\verb|plot_physics_info(axis, mfile_data, scan)|} Function to plot the
  physics info block

\item{\verb|plot_magnetics_info(axis, mfile_data, scan)|} Function to plot the
  magnetics info block

\item{\verb|plot_power_info(axis, mfile_data, scan)|} Function to plot the
  power info block

\item{\verb|plot_current_drive_info(axis, mfile_data, scan)|} Function to plot
  the current drive info block

\item{\verb|plot_geometry_info(axis, mfile_data, scan)|} Function to plot the
  geometry info block

\end{description}

\section{User Interface}
\label{sec:gui}
The user interface is still being developed, but currently allows for viewing
and editing of \indat\ files in a web browser. It can be found in the
\texttt{utilities/gui} directory.

\subsection{Launching}
The interface uses Django for its web framework. Currently it can only be
launched by running the Django development server locally. This is performed
automatically if you launch the GUI by typing \verb+start_gui+ from the
command line. For instructions on starting the server manually, see the
\verb|README| file in the \texttt{utilities/gui} directory.

\subsection{Editing variables}
Every variable should be listed along with its current value, a 'reference'
value and a short description. Hovering over the short description will
provide a longer description. All values begin with the default value
assigned by \process. 

The reference values are used to compare differences between two files.
Differences between the reference value and the current input value will be
highlighted in red.

The 'Meta' section allows editing of the run description, which will be placed
at the top of the created \indat. Changes to the iteration variables and
constraint equations that are enabled can be done using the checkboxes in the
relevant section.

\subsection{Opening and saving files}
An input file and a reference file can be opened using the buttons at the top
of the screen. The files opened must be compatible with the \process\ version
number shown in the top right of the screen.

The 'Save' button should produce a \indat\ file with a similar
layout to the GUI. Variables whose values are set different from the
\process\ default are listed under module headings, along with a comment
describing the variable. For integer variables, a description of the value
taken may also be included. The variables \texttt{neqns} and \texttt{nvar} are automatically
calculated.

\subsection{Searching}
There is currently no search feature, but searching for a particular
variable can be done using the browser's in-built search. Use the button
in the top-right of the screen to expand every module heading and use Ctrl-f
to search through the page.

