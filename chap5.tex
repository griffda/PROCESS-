%%-*-Latex-*-
\mychapter{Inclusion of Additional Variables and Equations}
\label{chap:modify}

It is often useful to add extra features to the code in order to model
new situations. This chapter provides instructions on how to add
various numerics related items to \PSD

\section{Input Parameters}

Input parameters (see Section~\ref{sec:inpars}) are added to the code
in the following way:
\begin{enumerate}
\item
Choose the most relevant \INCLUDE file, and, keeping everything in
alphabetical order, add the parameter to
\begin{enumerate}
\item
the correct type declaration block, and
\item
the corresponding \COMMON block.
\end{enumerate}
\item
Ensure that all the routines that use the new variable reference the
relevant \INCLUDE file.
\item
Add the parameter to the relevant section in routine {\tt INITIAL} in
source file {\tt initial.f}, giving it a ``sensible'' default value.
Keep to alphabetical order.
\item
Add the parameter to the relevant routine in source file {\tt
input.f}, including the comments at the start of the routine. The
comments in routine {\tt READNL} provide full instructions on how to
do this. Note that real (i.e.\ double precision) and integer variables
are treated differently, as are scalar quantities and arrays. Keep to
alphabetical order.
\item
Add the details of the parameter to the relevant section of the
variable descriptor file {\tt var.des}, keeping to alphabetical order.
\end{enumerate}

\section{Iteration Variables}

New iteration variables (see Section~\ref{sec:itvars}) are added in the
same way as input parameters, with the following additions:
\begin{enumerate}
\item
Increment the parameter {\tt ipnvars} in \INCLUDE file {\tt param.h}
to accommodate the new iteration variable.
\item
Add the variable to routines {\tt LOADXC} and {\tt CONVXC} in source
file {\tt xc.f}, mimicking the way that the existing iteration
variables are coded. Remember to ensure that these routines reference
the relevant \INCLUDE file.
\item
Assign sensible values for the variable's bounds to the relevant
elements in arrays {\tt boundl} and {\tt boundu} in routine {\tt
INITIAL} in source file {\tt initial.f}.
\item
Assign the relevant element of character array {\tt lablxc} to the
name of the variable, in routine {\tt INITIAL} in source file {\tt
initial.f}.
\item
Document the changes to {\tt ipnvars} and {\tt ixc} in the variable
descriptor file {\tt var.des}.
\end{enumerate}
If an existing input parameter is now required to be an iteration
variable, then simply carry out the tasks mentioned here.

It should be noted that iteration variables must not be reset elsewhere in the
code. That is, they may only be assigned new values when originally
initialised (in {\tt INITIAL}, or in the input file if required), and in
routine {\tt CONVXC} where the iteration process itself is performed.
Otherwise, the numerical procedure cannot adjust the value as it requires, and
the program will fail.

\section{Other Global Variables}

This type of variable embraces all those present in the \INCLUDE files
which do not need to be given initial values or to be input, as they
are calculated within the code. These should be added to the code in
the following way:
\begin{enumerate}
\item
Choose the most relevant \INCLUDE file, and, keeping everything in
alphabetical order, add the parameter to
\begin{enumerate}
\item
the correct type declaration block, and
\item
the corresponding \COMMON block.
\end{enumerate}
\item
Ensure that all the routines that use the new variable reference the
relevant \INCLUDE file.
\item
Add the parameter to the relevant section in routine {\tt INITIAL} in
source file {\tt initial.f}, giving it a default value of zero. This
is done to ensure that the variable is defined immediately, preventing
possible problems later. Keep to alphabetical order, as always.
\item
Add the details of the parameter to the relevant section of the
variable descriptor file {\tt var.des}.
\end{enumerate}

\section{Constraint Equations}

Constraint equations (see Section~\ref{sec:constraints}) are added to
\PS in the following way:
\begin{enumerate}
\item
Increment the parameter {\tt ipeqns} in \INCLUDE file {\tt param.h} to
accommodate the new constraint.
\item
Add the constraint equation to routine {\tt CON1} in source file {\tt
eqns.f}, ensuring that all the variables used in the formula are
contained in the \INCLUDE files present at the start of this routine.
Use a similar formulation to that used for the existing constraint
equations, remembering that the code will try to force {\tt cc(i)} to
be zero.
\item
Assign a description of the new constraint to the relevant element of
array {\tt lablcc} in routine {\tt INITIAL} in source file {\tt
initial.f}, using 34 characters or less.
\item
Document the changes to {\tt ipeqns} and {\tt icc} in the variable
descriptor file {\tt var.des}.
\end{enumerate}
Remember that if a limit equation is being added, a new f-value
iteration variable may also need to be added to the code.

\section{Figures of Merit}

New figures of merit (see Section~\ref{sec:foms}) are added to \PS in
the following way:

\begin{enumerate}
\item
Increment the parameter {\tt ipnfoms} in \INCLUDE file {\tt param.h}
to accommodate the new figure of merit.
\item
Add the new figure of merit equation to routine {\tt FUNFOM} in source
file {\tt optimiz.f}, following the method used in the existing
examples. The value of {\tt fc} should be of order unity, so select a
reasonable scaling factor if necessary.
\item
Ensure that all the variables used in the formula are contained in the
\INCLUDE files present at the start of this routine.
\item
Add a short description of the new figure of merit to the {\tt minmax}
entry in routine {\tt RDNL01} in source file {\tt input.f}.
\item
Assign a description of the new figure of merit to the relevant
element of array {\tt lablmm} in routine {\tt INITIAL} in source file
{\tt initial.f}, using 22 characters or less.
\item
Document the changes to {\tt ipnfoms} and {\tt minmax} in the variable
descriptor file {\tt var.des}.
\end{enumerate}

\section{Scanning Variables}

Scanning variables (see Section~\ref{sec:scans}) are added to \PS in
the following way:

\begin{enumerate}
\item
Increment the parameter {\tt ipnscnv} in \INCLUDE file {\tt param.h}
to accommodate the new scanning variable.
\item
Add a new assignment to the relevant part of routine {\tt SCAN} in
source file {\tt aamain.f}, following the examples already present,
including the inclusion of a short description of the new scanning
variable in variable {\tt xlabel}.
\item
Ensure that the scanning variable used in the assignment is contained
in one of the \INCLUDE files present at the start of this routine.
\item
Add a short description of the new scanning variable to the {\tt nsweep}
entry in routine {\tt RDNL15} in source file {\tt input.f}.
\item
Document the changes to {\tt ipnscnv} and {\tt nsweep} in the variable
descriptor file {\tt var.des}.
\end{enumerate}
