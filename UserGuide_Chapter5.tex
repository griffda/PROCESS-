\mychapter{Inclusion of New Models, Additional Variables and Equations}
\label{chap:modify}

It is often useful to add extra features to the code in order to model new
situations. This chapter provides instructions on how to add various numerics
related items to \process.

Please remember to modify the relevant Table(s) in this User Guide if changes
are made to the source code!

\section{Input Parameters}

Input parameters (see Section~\ref{sec:inpars}) are added to the code in the
following way:

\begin{enumerate}

\item Choose the most relevant module (usually one of those in source file
  \texttt{global\_variables.f90}). Keeping everything in alphabetical order,
  add a declaration statement for the new variable, specifying a ``sensible''
  default value, and a correctly formatted comment line to describe the
  variable (copying the examples already present).

\item Ensure that all the modules that use the new variable reference the
  relevant module via the Fortran \texttt{use} statement.

\item Add the parameter to routine \texttt{PARSE\_INPUT\_FILE} in source file
  \texttt{input.f90} in a suitable place --- keep to alphabetical order. The
  existing examples provide guidance on how to do this. Note that real (i.e.\
  double precision) and integer variables are treated differently, as are
  scalar quantities and arrays.

\item Modify \texttt{process\_dicts.py} and \texttt{write\_constraints.conf}
  accordingly.

\end{enumerate}

\section{Iteration Variables}

New iteration variables (see Section~\ref{sec:itvars}) are added in the
same way as input parameters, with the following additions:

\begin{enumerate}

\item Increment the parameter \texttt{ipnvars} in module \texttt{numerics} in
  source file \texttt{numerics.f90} to accommodate the new iteration variable.

\item Add an additional line to the initialisation of the array \texttt{ixc}
  in module \texttt{numerics} in source file \texttt{numerics.f90}.

\item Assign sensible values for the variable's bounds to the relevant
  elements in arrays \texttt{boundl} and \texttt{boundu} in module
  \texttt{numerics} in source file \texttt{numerics.f90}.

\item Assign the relevant element of character array \texttt{lablxc} to the
  name of the variable, in module \texttt{numerics} in source file
  \texttt{numerics.f90}.

\item Add the variable to routines \texttt{LOADXC} and \texttt{CONVXC} in
  source file \texttt{iteration\_variables.f90}, mimicking the way that the
  existing iteration variables are coded.

\item Modify \texttt{process\_dicts.py} and \texttt{write\_constraints.conf}
  accordingly.

\end{enumerate}

Don't forget to add suitable correctly-formatted comment lines to
\texttt{numerics.f90} to document the above changes.

If an existing input parameter is now required to be an iteration variable,
then simply carry out the tasks mentioned here.

It should be noted that iteration variables must not be reset elsewhere in the
code. That is, they may only be assigned new values when originally
initialised (in the relevant module, or in the input file if required), and in
routine \texttt{CONVXC} where the iteration process itself is performed.
Otherwise, the numerical procedure cannot adjust the value as it requires, and
the program will fail.

\section{Other Global Variables}

This type of variable embraces all those present in the modules in
\texttt{global\_variables.f90} (and some others elsewhere) which do not need
to be given initial values or to be input, as they are calculated within the
code. These should be added to the code in the following way:

\begin{enumerate}

\item Choose the most relevant module (usually one of those in source file
  \texttt{global\_variables.f90}). Keeping everything in alphabetical order,
  add a declaration statement for the new variable, specifying the initial
  value \texttt{0.0D0}, and a correctly formatted comment line to describe the
  variable (copying the examples already present).

\item Ensure that all the modules that use the new variable reference the
  relevant module via the Fortran \texttt{use} statement.

\end{enumerate}

\section{Constraint Equations}

Constraint equations (see Section~\ref{sec:constraints}) are added to
\process\ in the following way:

\begin{enumerate}

\item Increment the parameter \texttt{ipeqns} in module \texttt{numerics} in
  source file \texttt{numerics.f90} to accommodate the new constraint.

\item Add an additional line to the initialisation of the array \texttt{icc}
  in module \texttt{numerics} in source file \texttt{numerics.f90}.

\item Assign a description of the new constraint to the relevant element of
  array \texttt{lablcc}, in module \texttt{numerics} in source file
  \texttt{numerics.f90}.

\item Add the constraint equation to routine \texttt{CONSTRAINTS} in source
  file \texttt{constraint\_equations.f90}, ensuring that all the variables
  used in the formula are contained in the modules specified via \texttt{use}
  statements present at the start of this routine.  Use a similar formulation
  to that used for the existing constraint equations, remembering that the
  code will try to force \texttt{cc(i)} to be zero.

\end{enumerate}

Don't forget to add suitable correctly-formatted comment lines to
\texttt{numerics.f90} to document the above changes.

Remember that if a limit equation is being added, a new f-value iteration
variable may also need to be added to the code.

\section{Figures of Merit}

New figures of merit (see Section~\ref{sec:foms}) are added to \process\ in
the following way:

\begin{enumerate}

\item Increment the parameter \texttt{ipnfoms} in module \texttt{numerics} in
  source file \texttt{numerics.f90} to accommodate the new figure of merit.

\item Assign a description of the new figure of merit to the relevant element
  of array \texttt{lablmm} in module \texttt{numerics} in source file
  \texttt{numerics.f90}.

\item Add the new figure of merit equation to routine \texttt{FUNFOM} in
  source file \texttt{evaluators.f90}, following the method used in the
  existing examples. The value of \texttt{fc} should be of order unity, so
  select a reasonable scaling factor if necessary. Ensure that all the
  variables used in the new equation are contained in the modules specified
  via \texttt{use} statements present at the start of this file.

\item Modify \texttt{process\_dicts.py} accordingly.

\end{enumerate}

Don't forget to add suitable correctly-formatted comment lines to
\texttt{numerics.f90} to document the above changes.

\section{Scanning Variables}

Scanning variables (see Section~\ref{sec:scans}) are added to \process\ in
the following way:

\begin{enumerate}

\item Increment the parameter \texttt{ipnscnv} in module \texttt{scan\_module}
  in source file \texttt{scan.f90} to accommodate the new scanning variable.

\item Add a short description of the new scanning variable to the
  \texttt{nsweep} entry in source file \texttt{scan.f90}.

\item Add a new assignment to the relevant part of routine \texttt{SCAN} in
  source file \texttt{scan.f90}, following the examples already present,
  including the inclusion of a short description of the new scanning variable
  in variable \texttt{xlabel}.

\item Ensure that the scanning variable used in the assignment is contained in
  one of the modules specified via \texttt{use} statements present at the
  start of this routine.

\item Modify \texttt{process\_dicts.py} if necessary.

\end{enumerate}

\section{Submission of New Models}

The \process\ source code is maintained by CCFE, and resides in a
\textit{Git}~\cite{git} repository on the CCFE servers. We welcome
contributions of alternative or improved models and algorithms.

The Fortran~90/95 source code has a uniform visual style and structural
layout. CCFE will transfer any contributed models into \process, in order for
us to maintain its present coding standard. To simplify this task, we request
that contributors provide the following information for any new models that
they provide:

\begin{itemize}

\item A comprehensive description of the model; please provide a full list of
  references.

\item A list of all inputs and outputs: descriptions, default (input) values,
  allowed ranges, units.

\item If possible, please cross-reference any input/output variables to
  existing global variables listed in the variable descriptor file (see
  Section~\ref{sec:vardes}).

\item A description of any new numerics requirements (input parameters,
  iteration variables, constraint equations, figures of merit etc.).

\item A definition of any pre-requisites.

\item A description of any side-effects.

\item Any available test data, code examples or test programs (in Fortran or
  other language) would be useful.

\end{itemize}

